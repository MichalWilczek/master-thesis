
In this thesis, an electro-thermal multidimensional model is constructed in ANSYS APDL to simulate self-protectability scenarios of superconducting magnets. The analysis takes the longitudinal and the turn-to-turn quench propagation into account. The developed method has been illustrated with a case of the skew quadrupole, being one of the high-order corrector magnets for the High-Luminosity LHC upgrade project.

The quench propagation in superconducting magnets is a magneto-thermal effect. This is a challenging task to simulate due to multi-physics, multi-rate, multi-domain, and multi-scale problems. Moreover, high material nonlinearities must be studied as a function of temperature and, in some cases, magnetic field strength.

The possibility of conducting the quench studies in ANSYS APDL is verified first. The models, simulating the 1D quench propagation, are prepared in ANSYS and compared with STEAM-BBQ, being a tool for the 1D quench propagation implemented in COMSOL and developed at TE-MPE-PE at CERN. 

The standard solution of the heat balance equation is non-feasible due to steep temperature gradients and high nonlinearities. Therefore, to avoid the explicit solution of the quench front propagation, the ANSYS model relies on an external routine that imposes a constant quench velocity along the conductor based on simplified 1D numerical studies. The methodology is verified with standard numerical analyses performed in ANSYS. This thesis also covers algorithms and a Python code that were developed to implement the proposed methodology in ANSYS APDL. Advantages and disadvantages of the developed methodology are thoroughly discussed.