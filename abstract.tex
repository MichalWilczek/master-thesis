
In this thesis, an electro-thermal multidimensional model is constructed in ANSYS APDL to simulate a discharge of the skew quadrupole being one of the high-order corrector magnets for the High-Luminosity LHC project. The analysis takes into consideration longitudinal and turn-to-turn quench propagation. The model is validated against available measurements of current and resistive voltage obtained during the discharge of the magnet. 

The quench propagation in superconducting magnets is a challenging task from the simulation point of view including multi-physics, multi-rate, multi-domain, and multi-scale problems. One has to take into consideration high material non-linearities being a function of temperature and, in some cases, magnetic field. 

A possibility of conducting the quench studies in ANSYS APDL is verified first. The models simulating a 1D quench propagation are prepared in ANSYS and compared with STEAM-BBQ being a tool for quench simulations implemented in COMSOL and developed at TE-MPE-PE at CERN. 

In order to simulate a multidimensional quench propagation in the skew quadrupole in a relatively short time, the ANSYS model relies on an external routine that imposes a constant quench velocity along the conductor based on simplified 1D numerical studies. The methodology is verified with standard numerical analyses performed in ANSYS. The thesis also covers the algorithms and a Python code that are developed to implement the given methodology in ANSYS APDL. Advantages and disadvantages of the proposed methodology are thoroughly discussed.