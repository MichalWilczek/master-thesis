
\subsection{Geometry, Material Properties and Mesh}

Since copper and Nb-Ti are characterised by a~different volumetric heat capacity, the total volumetric heat capacity of a strand is given by

\begin{equation}
    c_\text{v, strand} = f_\text{Cu} ~ c_\text{v, Cu}(T) + f_\text{Nb-Ti} ~ c_\text{v, Nb-Ti}(T,B),
    \label{eqn: cv_equiv}
\end{equation}
where $c_\text{v, Cu}$ -- volumetric heat capacity of copper as a function of temperature, $c_\text{v, Nb-Ti}$ -- volumetric heat capacity of Nb-Ti as a function of temperature and magnetic field, $f_\text{Cu}$ -- volumetric fraction of copper, $f_\text{Nb-Ti}$ -- volumetric fraction of Nb-Ti. The~volumetric heat capacity of the composite strand for a given value of $B$ and $r_\text{Cu/sc}$ is shown in Fig. \ref{fig:eq_wind_cp}.

\begin{figure}[H]
\centering
    \begin{tikzpicture}
        \begin{axis}[
          no markers,
          width=0.7\linewidth, 
          height = 4.5cm,
          xlabel={$T,~\text{K}$},
          ylabel={$c_\text{v, strand},~\frac{\text{J}}{\text{m}^3 \cdot \text{K}}$},
          xmin=0.0,
          ymin=0.0,
          xmax=300.0
          ]
          \addplot table[x=temperature,y=cv_equiv,col sep=comma] {sections/1D_quench_modelling/figures/other/cv_equivalent.csv}; 
        \end{axis}
    \end{tikzpicture}
    \caption{Strand volumetric heat capacity as a function of temperature for $B=2~\text{T}$ and $r_\text{Cu/sc}=2.2$.}
    \label{fig:eq_wind_cp}
\end{figure}

According to~\cite[p.~46]{material_props_for_heat_transfer_modelling_in_nb3sn_magnets}, the thermal conductivity of Nb-Ti changes from 1 to 9~$\frac{\text{W}}{\text{m K}}$ in the temperature range of $T \in (20, 200)~\text{K}$. At temperatures below 20~K, the thermal conductivity of Nb-Ti is lower than~$1~\frac{\text{W}}{\text{m K}}$. By comparing the thermal conductivity of Nb-Ti and copper, whose thermal conductivity is presented in Appendix~\ref{subsection:copper_thermal_conductivity}, one can conclude that the thermal conductivity of Nb-Ti is negligible and only the part of copper contributes to the longitudinal heat propagation, as 
\begin{equation}
    k_\text{strand} = f_\text{Cu} ~ k_\text{Cu} + f_\text{Nb-Ti} ~ k_\text{Nb-Ti} \approx  f_\text{Cu} ~ k_\text{Cu},
    \label{eqn: k_equiv}
\end{equation}
where $k_\text{strand}$ -- thermal conductivity of a strand, $k_\text{Cu}$ -- thermal conductivity of copper, $k_\text{Nb-Ti}$ -- thermal conductivity of Nb-Ti. It is important to highlight that only in Chapter~\ref{chapter: 1d_quench_propagation_modelling}, the thermal conductivity of copper is calculated according to the Wiedemann-Franz formula, as
\begin{equation}
    k_\text{Cu} = 2.45 \cdot 10^{-8} ~ \frac{T}{\rho_\text{Cu}}.
    \label{eqn: k_cu_wiedemann_franz}
\end{equation}
Wiedeman-Franz formula is used for the sake of comparison of the results with COMSOL in which only such material property is available. In the remainder of the thesis, the thermal conductivity of copper and all other material properties are calculated according to the fits provided by NIST, as described in Appendix~\ref{appendix_material_properties_description}. 

In ANSYS, the element LINK33 is used to the 1D quench propagation. It is a uniaxial 2-node linear element with the ability to conduct heat between its nodes suitable for steady-state and transient analyses~\cite{ansys_element_manual}. Unlike ANSYS, COMSOL does not use explicit element types to define physical equations in a domain. Instead, one has to choose from a list of available physics modules, in this case thermal physics. Therefore, the algebraic physical equations related to heat conduction in COMSOL are applied to the nodes belonging to the composite strand. In both tools, a uniformly distributed mesh is used with mesh size equal to 0.1 mm. The mesh size is based on~\cite[p.~40]{paudel_thesis} in which it is recommended to discretise the domain in the scale less than or equal to 1~mm for an adiabatic analysis of a 1D quench propagation.

\subsection{Initial and Boundary Conditions, and Solver Settings}

Over the entire length of the strand, the power density is applied according to~(\ref{eqn: p_dens_equiv}). In addition, a Gaussian profile of the initial temperature is assumed according to 
\begin{equation}
    T(x) = T_\text{init} + (T_\text{max} - T_\text{init}) ~ e^{-(\frac{x}{\alpha})^2},
    \label{eqn: gaussian_temp_ic}
\end{equation}
where $T(x)$ -- temperature profile along $x$-axis, $T_\text{init}$ -- initial bath temperature of a strand, $T_\text{max}$ -- the maximum temperature of the Gaussian profile, $\alpha$ -- shape parameter of the Gaussian profile proportional to the standard deviation. The input parameters corresponding to the Gaussian profile, initial transport current, and initially quenched zone are depicted in Table~\ref{table: 1d_quench_propagation_analysis_init_temp_input_parameters}. 

\begin{table}[H]
    \caption{Input parameters for the 1D analysis.} 
    \vspace{-1.em} 
    \fontsize{10}{10}
    \selectfont 
    \renewcommand{\arraystretch}{1.5}
    \begin{center}
        \begin{tabular}{ ccc }  
        \hline
        parameter & value & unit \\
        \hline
        $I$ & 100 & [A] \\
        $B$ & 2 & [T] \\
        $T_\text{init}$ & 1.9 & [K] \\
        $T_\text{max}$ & 20.0 & [K] \\
        $T_\text{c}$ & 8.429 & [K] \\
        $L_\text{quench, init}$ & 0.1 & [m] \\ 
        $\alpha$ & 0.223 & [m] \\   
        \hline 
        \end{tabular}
    \end{center}  
     \label{table: 1d_quench_propagation_analysis_init_temp_input_parameters} 
 \end{table}

The initially quenched zone is equal to $L_\text{quench, init}= 0.1~\text{m}$ when symmetry is not taken into consideration. It~means that at $x=0.05~\text{m}$, the strand is at the critical temperature for the given magnetic field strength. The shape parameter, \textalpha~in (\ref{eqn: gaussian_temp_ic}) is calculated accordingly. The critical temperature $T_\text{c}$ is calculated as in (\ref{eqn:critical_temperature_appendix}) in Appendix~\ref{appendix:nbti_material_properties}. A symmetry condition is applied at the position $x=0~\text{m}$. Thus, a one metre-long strand represents a half of the analysed domain, as presented in Fig. \ref{fig: init_gauss_temp_distr}.

\begin{figure}[H]
\centering
    \begin{tikzpicture}
        \begin{axis}[
          no markers,
          width=0.7\linewidth, 
          height = 4.5cm,
          xlabel={$L_\text{strand},~\text{m}$},
          ylabel={$T,~\text{K}$},
          xmin=0.0,
          ymin=0.0,
          xmax=1.0
          ]
          \addplot table[x=posx,y=temperature,col sep=comma] {sections/1D_quench_modelling/figures/other/gaus_init_distr.csv}; 
        \end{axis}
    \end{tikzpicture}
    \caption{Initial Gaussian temperature distribution.}
    \label{fig: init_gauss_temp_distr}
\end{figure}

The solution in both ANSYS and COMSOL are obtained with a strict time stepping, i.e. the time step during the solution is fixed. The input parameters related to the time stepping algorithm and the total simulation time are presented in Table \ref{table: 1d_quench_propagation_analysis_time_stepping_input_parameters}. ANSYS uses a default sparse solver with an automatic solution control setting. In COMSOL, the Multifrontal Massively Parallel Sparse Direct Solver (MUMPS) is applied. Both tools use direct solvers with an implicit Backward Euler time stepping method~\cite{comsol_webpage, ansys_command_reference}. In a time-dependent domain, default values for time stepping convergence are applied in both ANSYS and COMSOL. The geometry is built based on the \nth{1} order elements in each tool.

\begin{table}[H]
    \caption{Analysis time stepping input parameters.} 
    \vspace{-1.em} 
    \fontsize{10}{10}
    \selectfont 
    \renewcommand{\arraystretch}{1.5}
    \begin{center}
        \begin{tabular}{ ccc }  
        \hline
        parameter & value & unit \\
        \hline
        $t_\text{total}$ & $10^{-1}$ & [s] \\   
        $t_\text{step, max}$ & $10^{-4}$ & [s] \\   
        \hline 
        \end{tabular}
    \end{center}  
     \label{table: 1d_quench_propagation_analysis_time_stepping_input_parameters} 
 \end{table}

\subsection{Results}
\label{subsubsection:1d_quench_propagation_analysis_results_no_insulation}

The temperature distribution profiles are compared at three time steps $t=\{0.03, 0.06, 0.1\}$~s, as presented in Fig.~\ref{fig: 1d_no_insulation_temp_along_strand_comparison}. One can conclude that the initial heat pulse is sufficient to initiate quench. In fact, relatively high thermal conductivity and low heat capacity of the strand allow for a longitudinal quench propagation. At each time window, the hot-spot is placed at $x=0~\text{m}$. In general, COMSOL predicts faster quench propagation.

\begin{figure}[H]
\centering
    \begin{tikzpicture}
        \begin{axis}[
          no markers,
          width=0.7\linewidth, 
          height = 5.0cm,
          xlabel={$\bar{x},~\text{m}$},
          ylabel={$T,~\text{K}$},
          xmin=0.0,
          ymin=0.0,
          xmax=1.0,
          legend pos=outer north east
          ]
        %   Initial temperature curve
          \addplot[smooth, black] table[x=posx,y=t_0_0_ans,col sep=comma] {sections/1D_quench_modelling/figures/results_no_insulation/Temp_tstep_10ms_1e4elems_f2_2.csv};
          
        %   COMSOL plots
          \addplot[smooth, red] table[x=posx,y=t_0_03_com,col sep=comma] {sections/1D_quench_modelling/figures/results_no_insulation/Temp_tstep_10ms_1e4elems_f2_2.csv};
          \addplot[smooth, red] table[x=posx,y=t_0_06_com,col sep=comma] {sections/1D_quench_modelling/figures/results_no_insulation/Temp_tstep_10ms_1e4elems_f2_2.csv};
          \addplot[smooth, red] table[x=posx,y=t_0_1_com,col sep=comma] {sections/1D_quench_modelling/figures/results_no_insulation/Temp_tstep_10ms_1e4elems_f2_2.csv};

        %   ANSYS plots
          \addplot[smooth, blue] table[x=posx,y=t_0_03_ans,col sep=comma] {sections/1D_quench_modelling/figures/results_no_insulation/Temp_tstep_10ms_1e4elems_f2_2.csv};
          \addplot[smooth, blue] table[x=posx,y=t_0_06_ans,col sep=comma] {sections/1D_quench_modelling/figures/results_no_insulation/Temp_tstep_10ms_1e4elems_f2_2.csv};
          \addplot[smooth, blue] table[x=posx,y=t_0_1_ans,col sep=comma] {sections/1D_quench_modelling/figures/results_no_insulation/Temp_tstep_10ms_1e4elems_f2_2.csv};
          
          \legend{
          $T_\text{init}$,
          COMSOL,,,
          ANSYS
          }
        \end{axis}
        
        \draw[black, thick, ->] (2,3) -- (3,3);
        \node[scale = 1] at (3.8, 3) {$\vec{v}_\text{quench}$};   
        
    \end{tikzpicture}
    \caption{Temperature distribution calculated in COMSOL and ANSYS for three time steps: $t=\{0.03, 0.06, 0.1\}$~s with a specified direction of the quench propagation, $\vec{v}_\text{quench}$.}
    \label{fig: 1d_no_insulation_temp_along_strand_comparison}
\end{figure}

The quench velocity is calculated by comparing the position of the quench front at $t=0.06~\text{s}$ and $t=0.1~\text{s}$, assuming that at this moment, the quench propagation velocity reaches a steady-state value. The relative error is calculated as
\begin{equation}
    E_\text{r} = \frac{r_\text{ANSYS}-r_\text{COMSOL}}{r_\text{COMSOL}}~100\%,
    \label{eqn:relative_error_comsol_ansys_benchmarking}
\end{equation}
where $r_\text{ANSYS}$ -- a given result from ANSYS, $r_\text{COMSOL}$ -- a given result from COMSOL. The~relative error is estimated for:

\begin{itemize}
    \item quench velocity, as presented in Table \ref{table: 1d_no_insulation_v_quench_comparison},
    \item temperature along the strand length at $t=0.1~\text{s}$, as presented in Fig. \ref{fig: ans_comsol_comparison_rel_error_temp_f_2_2}.
\end{itemize}

\begin{table}[H]
    \caption{Quench velocity comparison between COMSOL and ANSYS.} 
    \vspace{-1.em} 
    \fontsize{10}{10}
    \selectfont 
    \renewcommand{\arraystretch}{1.5}
    \begin{center}
        \begin{tabular}{ ccc }  
        \hline
        parameter & value & unit \\
        \hline
        $v_\text{quench, COMSOL}$ & 7.075 & [m/s] \\
        $v_\text{quench, ANSYS}$ & 6.968 & [m/s] \\
        $E_\text{r}$ & -1.519 & [\%] \\
        \hline 
        \end{tabular}
    \end{center}  
     \label{table: 1d_no_insulation_v_quench_comparison} 
 \end{table}

\begin{figure}[H]
\centering
    \begin{tikzpicture}
        \begin{axis}[
          width=0.7\linewidth, 
          height = 4.5cm,
          xlabel={$\bar{x},~\text{m}$},
          ylabel={$E_\text{r}$, \%},
          xmin=0.0,
          xmax=1.0
          ]
          \addplot[blue, mark=*] table[x=posx,y=error_0_1,col sep=comma] {sections/1D_quench_modelling/figures/results_no_insulation/Temp_tstep_10ms_1e4elems_f2_2_error.csv};
        \end{axis}
    \end{tikzpicture}
    \caption{Relative error along the strand with respect to the temperature distribution for $t=0.1~\text{s}$.}
    \label{fig: ans_comsol_comparison_rel_error_temp_f_2_2}
\end{figure}

The difference in the hot-spot temperature at $x=0~\text{m}$ does not exceed 0.06 \%. The quench velocity in ANSYS is lower than the one in COMSOL by less than 2~\% which results in the increase of the relative error corresponding to the nodal temperature distribution up to 20~\% close to the quench front at $t=0.1~\text{s}$.