
This analysis presents the case for a bare strand, i.e. when neither insulation nor epoxy resin are considered. Therefore, the strand cross-section is a yellow part of the domain presented in Fig. \ref{fig: 1d_strand_geometry}. The strand is a composite with a specified non-superconductor to superconductor ratio described as 

\begin{equation}
    \left\{ \begin{array}{ll}
    f_{Cu/Nb-Ti} = \frac{f_\text{Cu}}{f_\text{Nb-Ti}} = \frac{A_\text{Cu}}{A_\text{Nb-Ti}}\\ \\
    f_\text{Cu} + f_\text{Nb-Ti} = 1,
    \end{array} \right.
    \label{eqn: non_super_to_super_ratio}
\end{equation}
where $f_\text{Cu}$ -- volumetric ratio of copper, $f_\text{Nb-Ti}$ -- volumetric ratio of Nb-Ti, $A_\text{Cu}$ -- cross-sectional area of copper, $A_\text{Nb-Ti}$ -- cross-sectional area of Nb-Ti. Since both materials are characterised by a different volumetric heat capacity, the total volumetric heat capacity of a strand is given by

\begin{equation}
    C_\text{v, strand} = f_\text{Cu} ~ C_\text{v, Cu} + f_\text{Nb-Ti} ~ C_\text{v, Nb-Ti},
    \label{eqn: cv_equiv}
\end{equation}
where $C_\text{v, Cu}$ -- volumetric heat capacity of copper, $C_\text{v, Nb-Ti}$ -- volumetric heat capacity of Nb-Ti. The graphic representation of of this formula is shown in Fig. \ref{fig:eq_wind_cp}.

\begin{figure}[h!]
\centering
    \begin{tikzpicture}
        \begin{axis}[
          no markers,
          width=0.7\linewidth, 
          height = 4.5cm,
          xlabel={$T,~\text{K}$},
          ylabel={$C_\text{v, strand},~\frac{\text{J}}{\text{m}^3 \cdot \text{K}}$},
          xmin=0.0,
          ymin=0.0,
          xmax=50.0
          ]
          \addplot table[x=temperature,y=cv_equiv,col sep=comma] {sections/1D_quench_modelling/figures/cv_equivalent.csv}; 
        \end{axis}
    \end{tikzpicture}
    \caption{Strand volumetric heat capacity as a function of temperature for $B=2~\text{T}$ and $f_{Cu/Nb-Ti}=2.2$}
    \label{fig:eq_wind_cp}
\end{figure}

At temperatures above the critical temperature, the current flows both through a superconductor and a copper matrix. Both strand components can be represented as a parallel connection of two resistors, $R_\text{Nb-Ti}$ and $R_\text{Cu}$. Above the critical temperature, $R_\text{Nb-Ti} \gg R_\text{Cu}$. Therefore, it is assumed that the current only flows through copper stabiliser and only this part of the strand contributes to the Joule heating. Then, the average power density in the strand cross-section is calculated as 
\begin{equation}
    q_\text{Joule} = J_\text{strand}^2~\rho_\text{strand} = J_\text{Cu}^2~\rho_\text{Cu}~f_\text{Cu} = \frac{I_\text{Cu}^2}{f_\text{Cu}^2~A_\text{strand}^2}~\rho_\text{Cu}~f_\text{Cu} = \frac{I_{Cu}^2}{A_\text{strand}^2}~\frac{\rho_\text{Cu}}{f_\text{Cu}}, 
    \label{eqn: p_dens_equiv}
\end{equation}
where $J_{strand}$ -- current density in a strand, $\rho_{strand}$ -- resistivity of a strand, $J_\text{Cu}$ -- current density in copper, $\rho_\text{Cu}$ -- resistivity of copper, $I_\text{Cu}$ -- current in copper, $A_\text{strand}$ -- cross-sectional area of a strand. One can notice that, in order to apply power density over the entire strand domain, the copper resistivity $\rho_\text{Cu}$ should be divided by the non-superconductor ratio, $f_\text{Cu}$.

Since the materials properties of electrical and thermal conductivity are often strictly connected, in this case $k_\text{Cu} \gg k_\text{Nb-Ti}$. Then, it is assumed that the thermal conductivity of Nb-Ti is negligible and only the part of copper contributes to longitudinal heat propagation, as 
\begin{equation}
    k_\text{strand} = f_\text{Cu} ~ k_\text{Cu} + f_\text{Nb-Ti} ~ k_\text{Nb-Ti} =  f_\text{Cu} ~ k_\text{Cu},
    \label{eqn: k_equiv}
\end{equation}
where $k_\text{strand}$ -- thermal conductivity of a strand, $k_\text{Cu}$ -- thermal conductivity of copper, $k_\text{Nb-Ti}$ -- thermal conductivity of Nb-Ti. It is important to highlight that only in Chapter \ref{section: 1d_quench_propagation_modelling}, the thermal conductivity of copper is calculated according to the Wiedemann-Franz, as
\begin{equation}
    k_\text{Cu} = 2.45 \cdot 10^{-8} ~ \frac{T}{\rho_\text{Cu}},
    \label{eqn: k_cu_wiedemann_franz}
\end{equation}
where $T$ -- local strand temperature. Wiedeman-Franz formula is used for the sake of comparison of the results with COMSOL in which only such a material property repository was created for the internal use of my working group at CERN. In the remainder of the thesis, thermal conductivity of copper and all other material properties are calculated according to NIST standards, as described in Appendix \ref{appendix_material_properties_description}. 

The Gaussian profile of initial temperature is assumed according to 
\begin{equation}
    T(x) = T_\text{init} + (T_\text{max} - T_\text{init}) ~ e^{-(\frac{x}{\alpha})^2},
    \label{eqn: gaussian_temp_ic}
\end{equation}
where $T(x)$ -- temperature profile along x-axis, $T_\text{init}$ -- initial bath temperature of a strand, $T_\text{max}$ -- maximum temperature of in Gaussian profile, $\alpha$ -- shape factor of Gaussian profile. A symmetry condition is applied at the position $x=0~\text{m}$. As presented in Fig. \ref{fig: init_gauss_temp_distr}, one-metre cable represents a half of the analysed domain. The input parameters for the 1D analysis are depicted in Table~\ref{table: 1d_quench_propagation_analysis_input_parameters}. The initially quenched zone is equal to $L_\text{quench, init}= 0.1~\text{m}$ when symmetry is not taken into consideration. It~means that at $x=0.05~\text{m}$, the strand is at critical temperature for the given magnetic field strength. Therefore, the $\alpha$ parameter in (\ref{eqn: gaussian_temp_ic}) is calculated accordingly. RRR and $f$~are assumed to be equal as in the case of the skew quadrupole further described in Section \ref{section:skew_quadrupole_quench_detection_analysis}. 

\begin{figure}[h!]
\centering
    \begin{tikzpicture}
        \begin{axis}[
          no markers,
          width=0.7\linewidth, 
          height = 4.5cm,
          xlabel={$L_\text{strand},~\text{m}$},
          ylabel={$T,~\text{K}$},
          xmin=0.0,
          ymin=0.0,
          xmax=1.0
          ]
          \addplot table[x=posx,y=temperature,col sep=comma] {sections/1D_quench_modelling/figures/gaus_init_distr.csv}; 
        \end{axis}
    \end{tikzpicture}
    \caption{Initial Gaussian temperature distribution}
    \label{fig: init_gauss_temp_distr}
\end{figure}

\begin{table}[h!]
    \caption{Analysis input parameters} 
    \vspace{-1.em} 
    \fontsize{10}{10}
    \selectfont 
    \renewcommand{\arraystretch}{1.5}
    \begin{center}
        \begin{tabular}{ ccc }  
        \hline
        $T_\text{init}$ & 1.9 & [K] \\
        $T_\text{max}$ & 20.0 & [K] \\
        $T_\text{c}$ & 8.429 & [K] \\
        $B$ & 2 & [T] \\
        $L_\text{quench, init}$ & 0.1 & [m] \\ 
        $\alpha$ & 0.223 & [m] \\   
        $I$ & 100 & [A] \\   
        RRR & 193 & [-] \\   
        f & 2.2 & [-] \\   
        time total & 0.1 & [s] \\   
        time step & 10 & [\textmu s] \\   
        number of elements & $10^4$ & [-] \\   
        \hline 
        \end{tabular}
    \end{center}  
     \label{table: 1d_quench_propagation_analysis_input_parameters} 
 \end{table}

Three time steps were compared for $t=\{0.03, 0.06, 0.1\}$ s, as presented in Fig. \ref{fig: ans_comsol_comparison_f_2_2}. In each of the software, the quench velocity was calculated by comparing the position of the quench front at $t=0.06~\text{s}$ and $t=0.1~\text{s}$. 
The relative error was estimated for: 
\begin{itemize}
    \item quench velocity, as presented in Table \ref{table: 1d_no_insulation_v_quench_comparison},
    \item temperature along the strand length at $t=0.1~\text{s}$, as presented in Fig. \ref{fig: 1d_no_insulation_temp_along_strand_comparison}.
\end{itemize}

The difference in hot spot temperature at $x=0~\text{m}$ does not exceed 0.06 \%. The quench velocity is slower in case of ANSYS by less than 2~\% which results in the increase of relative error up to 20~\% at the quench front at $t=0.1~\text{s}$, as presented in Fig. \ref{fig: 1d_no_insulation_temp_along_strand_comparison}.

\begin{table}[h!]
    \caption{Quench velocity comparison in COMSOL and ANSYS} 
    \vspace{-1.em} 
    \fontsize{10}{10}
    \selectfont 
    \renewcommand{\arraystretch}{1.5}
    \begin{center}
        \begin{tabular}{ ccc }  
        \hline
        $v_\text{quench, COMSOL}$ & 7.075 & [m/s] \\
        $v_\text{quench, ANSYS}$ & 6.968 & [m/s] \\
        Relative error & -1.519 & [\%] \\
        \hline 
        \end{tabular}
    \end{center}  
     \label{table: 1d_no_insulation_v_quench_comparison} 
 \end{table}

\begin{figure}[h!]
\centering
    \begin{tikzpicture}
        \begin{axis}[
          no markers,
          width=0.7\linewidth, 
          height = 4.5cm,
          xlabel={$L_\text{strand},~\text{m}$},
          ylabel={$T,~\text{K}$},
          xmin=0.0,
          ymin=0.0,
          xmax=1.0
          ]
        %   Initial temperature curve
          \addplot[smooth, black] table[x=posx,y=t_0_0_ans,col sep=comma] {sections/1D_quench_modelling/figures/results_no_insulation/Temp_tstep_10ms_1e4elems_f2_2.csv};
          
        %   COMSOL plots
          \addplot[smooth, red] table[x=posx,y=t_0_03_com,col sep=comma] {sections/1D_quench_modelling/figures/results_no_insulation/Temp_tstep_10ms_1e4elems_f2_2.csv};
          \addplot[smooth, red] table[x=posx,y=t_0_06_com,col sep=comma] {sections/1D_quench_modelling/figures/results_no_insulation/Temp_tstep_10ms_1e4elems_f2_2.csv};
          \addplot[smooth, red] table[x=posx,y=t_0_1_com,col sep=comma] {sections/1D_quench_modelling/figures/results_no_insulation/Temp_tstep_10ms_1e4elems_f2_2.csv};

        %   ANSYS plots
          \addplot[smooth, blue] table[x=posx,y=t_0_03_ans,col sep=comma] {sections/1D_quench_modelling/figures/results_no_insulation/Temp_tstep_10ms_1e4elems_f2_2.csv};
          \addplot[smooth, blue] table[x=posx,y=t_0_06_ans,col sep=comma] {sections/1D_quench_modelling/figures/results_no_insulation/Temp_tstep_10ms_1e4elems_f2_2.csv};
          \addplot[smooth, blue] table[x=posx,y=t_0_1_ans,col sep=comma] {sections/1D_quench_modelling/figures/results_no_insulation/Temp_tstep_10ms_1e4elems_f2_2.csv};
          
        \end{axis}
    \end{tikzpicture}
    \caption{Temperature distribution calculated in COMSOL (red) and ANSYS (blue) for three time frames: $t=\{0.03, 0.06, 0.1\}$ s.}
    \label{fig: 1d_no_insulation_temp_along_strand_comparison}
\end{figure}

\begin{figure}[h!]
\centering
    \begin{tikzpicture}
        \begin{axis}[
          width=0.7\linewidth, 
          height = 4.5cm,
          xlabel={$L_\text{strand},~\text{m}$},
          ylabel={Relative error, \%},
          xmin=0.0,
          xmax=1.0
          ]
          \addplot[blue, mark=*] table[x=posx,y=error_0_1,col sep=comma] {sections/1D_quench_modelling/figures/results_no_insulation/Temp_tstep_10ms_1e4elems_f2_2_error.csv};
        \end{axis}
    \end{tikzpicture}
    \caption{Relative error along the strand for $t=0.1~\text{s}$.}
    \label{fig: ans_comsol_comparison_f_2_2}
\end{figure}

In ANSYS, the element LINK33 was used to solve this task. It is a standard uniaxial element with the ability to conduct heat between its nodes suitable for steady-state and transient analyses~\cite{ansys_element_manual}. Power density equation described in (\ref{eqn: p_dens_equiv}), was defined as a heat source. In COMSOL elements are not common. Therefore, the algebraic physical equations related to heat conduction problems between the nodes had to be specified.