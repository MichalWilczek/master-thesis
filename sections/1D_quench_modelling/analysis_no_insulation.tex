
\subsection{Geometry, Material Properties and Mesh}

The strand is a composite with a specified non-superconductor to superconductor ratio given as 

\begin{equation}
    \left\{ \begin{array}{ll}
    r_\text{Cu/Nb-Ti} = \frac{f_\text{Cu}}{f_\text{Nb-Ti}} = \frac{A_\text{Cu}}{A_\text{Nb-Ti}}\\ \\
    f_\text{Cu} + f_\text{Nb-Ti} = 1,
    \end{array} \right.
    \label{eqn: non_super_to_super_ratio}
\end{equation}
where $f_\text{Cu}$ -- volumetric fraction of copper, $f_\text{Nb-Ti}$ -- volumetric fraction of Nb-Ti, $A_\text{Cu}$ -- cross-sectional area of copper, $A_\text{Nb-Ti}$ -- cross-sectional area of Nb-Ti. Since both materials are characterised by a~different volumetric heat capacity, the total volumetric heat capacity of a strand given by

\begin{equation}
    C_\text{v, strand} = f_\text{Cu} ~ C_\text{v, Cu} + f_\text{Nb-Ti} ~ C_\text{v, Nb-Ti},
    \label{eqn: cv_equiv}
\end{equation}
where $C_\text{v, Cu}$ -- volumetric heat capacity of copper, $C_\text{v, Nb-Ti}$ -- volumetric heat capacity of Nb-Ti is shown in Fig. \ref{fig:eq_wind_cp}.

\begin{figure}[H]
\centering
    \begin{tikzpicture}
        \begin{axis}[
          no markers,
          width=0.7\linewidth, 
          height = 4.5cm,
          xlabel={$T,~\text{K}$},
          ylabel={$C_\text{v, strand},~\frac{\text{J}}{\text{m}^3 \cdot \text{K}}$},
          xmin=0.0,
          ymin=0.0,
          xmax=300.0
          ]
          \addplot table[x=temperature,y=cv_equiv,col sep=comma] {sections/1D_quench_modelling/figures/other/cv_equivalent.csv}; 
        \end{axis}
    \end{tikzpicture}
    \caption{Strand volumetric heat capacity as a function of temperature for $B=2~\text{T}$ and $r_\text{Cu/Nb-Ti}=2.2$.}
    \label{fig:eq_wind_cp}
\end{figure}

According to the polynomial interpolation fit used in~\cite[p.~46]{material_props_for_heat_transfer_modelling_in_nb3sn_magnets}, thermal conductivity of Nb-Ti increases linearly in the temperature range of $T \in (20, 200)~\text{K}$ and its value changes from 1 to 9~$\frac{\text{W}}{\text{m K}}$. At temperatures below 20~K, the thermal conductivity of Nb-Ti is lower than~$1~\frac{\text{W}}{\text{m K}}$. By comparing the thermal conductivity of Nb-Ti and copper (see in Appendix~\ref{appendix_material_properties_description}), one can assume that the thermal conductivity of Nb-Ti is negligible and only the part of copper contributes to the longitudinal heat propagation, as 
\begin{equation}
    k_\text{strand} = f_\text{Cu} ~ k_\text{Cu} + f_\text{Nb-Ti} ~ k_\text{Nb-Ti} \approx  f_\text{Cu} ~ k_\text{Cu},
    \label{eqn: k_equiv}
\end{equation}
where $k_\text{strand}$ -- thermal conductivity of a strand, $k_\text{Cu}$ -- thermal conductivity of copper, $k_\text{Nb-Ti}$ -- thermal conductivity of Nb-Ti. It is important to highlight that only in Chapter~\ref{chapter: 1d_quench_propagation_modelling}, the thermal conductivity of copper is calculated according to the Wiedemann-Franz formula, as
\begin{equation}
    k_\text{Cu} = 2.45 \cdot 10^{-8} ~ \frac{T}{\rho_\text{Cu}},
    \label{eqn: k_cu_wiedemann_franz}
\end{equation}
where $T$ -- local strand temperature. Wiedeman-Franz formula is used for the sake of comparison of the results with COMSOL in which only such a material property was available. In the remainder of the thesis, the thermal conductivity of copper and all other material properties are calculated according to fits provided by NIST, as described in Appendix~\ref{appendix_material_properties_description}. 

In ANSYS, the element LINK33 was used to solve the task of 1D thermal quench propagation. It is a uniaxial linear element with the ability to conduct heat between its nodes suitable for steady-state and transient analyses~\cite{ansys_element_manual}. Unlike ANSYS, COMSOL does not use explicit element types to define physical equations in a domain. Instead, one has to choose from a list of available physics modules, in this case thermal physics. Therefore, the algebraic physical equations related to heat conduction in COMSOL were applied externally to the nodes belonging to the composite strand. In both tools, a uniformly distributed mesh was used with mesh size equal to 0.1 mm.

\subsection{Initial and Boundary Conditions}

In the superconducting state, the current flows entirely through a superconductor characterised by zero-resistivity. When a quench occurs, the current starts commuting and strand components can be represented as a parallel connection of two resistors, $R_\text{Nb-Ti}$ and $R_\text{Cu}$. Above the critical temperature, the superconductor resistance is much larger than the one of copper, i.e. $R_\text{Nb-Ti} \gg R_\text{Cu}$. Therefore, it is assumed that the current only flows through a copper stabiliser and only this part of the strand contributes to the Joule heating. Over the entire domain of a strand, a heat source is applied as a power density as
\begin{equation}
    q_\text{Joule} = J_\text{strand}^2~\rho_\text{strand} = J_\text{Cu}^2~\rho_\text{Cu}~f_\text{Cu} = \frac{I_\text{Cu}^2}{f_\text{Cu}^2~A_\text{strand}^2}~\rho_\text{Cu}~f_\text{Cu} = \frac{I_\text{Cu}^2}{A_\text{strand}^2}~\frac{\rho_\text{Cu}}{f_\text{Cu}}, 
    \label{eqn: p_dens_equiv}
\end{equation}
where $J_{strand}$ -- current density in a strand, $\rho_\text{strand}$ -- resistivity of a strand, $J_\text{Cu}$ -- current density in copper, $\rho_\text{Cu}$ -- resistivity of copper, $I_\text{Cu}$ -- current in copper stabiliser, $A_\text{strand}$ -- cross-sectional area of a strand. One should notice that the copper resistivity $\rho_\text{Cu}$ should be divided by the non-superconductor fraction as 
\begin{equation}
    \rho_\text{strand} = \frac{\rho_\text{Cu}}{f_\text{Cu}}.
    \label{eqn:strand_resistivity}
\end{equation}

The Gaussian profile of initial temperature is assumed according to 
\begin{equation}
    T(x) = T_\text{init} + (T_\text{max} - T_\text{init}) ~ e^{-(\frac{x}{\alpha})^2},
    \label{eqn: gaussian_temp_ic}
\end{equation}
where $T(x)$ -- temperature profile along x-axis, $T_\text{init}$ -- initial bath temperature of a strand, $T_\text{max}$ -- maximum temperature of in Gaussian profile, $\alpha$ -- shape parameter of Gaussian profile. The initial input parameters for the 1D analysis are depicted in Table~\ref{table: 1d_quench_propagation_analysis_init_temp_input_parameters}. 

\begin{table}[H]
    \caption{Initial temperature input parameters.} 
    \vspace{-1.em} 
    \fontsize{10}{10}
    \selectfont 
    \renewcommand{\arraystretch}{1.5}
    \begin{center}
        \begin{tabular}{ ccc }  
        \hline
        parameter & value & unit \\
        \hline
        $I$ & 100 & [A] \\
        $B$ & 2 & [T] \\
        $T_\text{init}$ & 1.9 & [K] \\
        $T_\text{max}$ & 20.0 & [K] \\
        $T_\text{c}$ & 8.429 & [K] \\
        $L_\text{quench, init}$ & 0.1 & [m] \\ 
        $\alpha$ & 0.223 & [m] \\   
        \hline 
        \end{tabular}
    \end{center}  
     \label{table: 1d_quench_propagation_analysis_init_temp_input_parameters} 
 \end{table}

The initially quenched zone is equal to $L_\text{quench, init}= 0.1~\text{m}$ when symmetry is not taken into consideration. It~means that at $x=0.05~\text{m}$, the strand is at critical temperature for the given magnetic field strength. The shape parameter, \textalpha~in (\ref{eqn: gaussian_temp_ic}) is calculated accordingly. The critical temperature is calculated as
\begin{equation}
    T_\text{c}(B) = T_\text{c0}\cdot(1-\frac{B}{B_\text{c20}})^{0.59},
\end{equation}
where $T_\text{c0}$ -- maximum critical temperature for $B=0~\text{T}$, $B_\text{c20}$ -- maximum upper critical magnetic field for $T=0~\text{K}$. Their numerical values for Nb-Ti are presented in Table~\ref{table: 1d_quench_propagation_analysis_crit_temp_params}. In both parameters, the current density is equal to zero. 

\begin{table}[H]
    \caption{Critical temperature parameters for Nb-Ti~\cite[p.~755]{empirical_scaling_formulas_for_critical_current}.} 
    \vspace{-1.em} 
    \fontsize{10}{10}
    \selectfont 
    \renewcommand{\arraystretch}{1.5}
    \begin{center}
        \begin{tabular}{ ccc }  
        \hline
        parameter & value & unit \\
        \hline
        $T_\text{c0}$ & 9.2 & [K] \\
        $B_\text{c20}$ & 14.5 & [T] \\
        \hline 
        \end{tabular}
    \end{center}  
     \label{table: 1d_quench_propagation_analysis_crit_temp_params} 
 \end{table}

A symmetry condition is applied at the position $x=0~\text{m}$. As presented in Fig. \ref{fig: init_gauss_temp_distr}, one-metre cable represents a half of the analysed domain. 

\begin{figure}[H]
\centering
    \begin{tikzpicture}
        \begin{axis}[
          no markers,
          width=0.7\linewidth, 
          height = 4.5cm,
          xlabel={$L_\text{strand},~\text{m}$},
          ylabel={$T,~\text{K}$},
          xmin=0.0,
          ymin=0.0,
          xmax=1.0
          ]
          \addplot table[x=posx,y=temperature,col sep=comma] {sections/1D_quench_modelling/figures/other/gaus_init_distr.csv}; 
        \end{axis}
    \end{tikzpicture}
    \caption{Initial Gaussian temperature distribution.}
    \label{fig: init_gauss_temp_distr}
\end{figure}

Input parameters related to the time stepping algorithm and the total simulation time are presented in Table \ref{table: 1d_quench_propagation_analysis_time_stepping_input_parameters}. In a time-dependent domain, default values for time stepping convergence were applied in both software.

\begin{table}[H]
    \caption{Analysis time stepping input parameters.} 
    \vspace{-1.em} 
    \fontsize{10}{10}
    \selectfont 
    \renewcommand{\arraystretch}{1.5}
    \begin{center}
        \begin{tabular}{ ccc }  
        \hline
        parameter & value & unit \\
        \hline
        time total & 0.1 & [s] \\   
        max time step size & 10 & [\textmu s] \\   
        \hline 
        \end{tabular}
    \end{center}  
     \label{table: 1d_quench_propagation_analysis_time_stepping_input_parameters} 
 \end{table}

\subsection{Results}
\label{subsubsection:1d_quench_propagation_analysis_results_no_insulation}

The results are compared at three time steps $t=\{0.03, 0.06, 0.1\}$ s, as presented in Fig.~\ref{fig: 1d_no_insulation_temp_along_strand_comparison}.

\begin{figure}[H]
\centering
    \begin{tikzpicture}
        \begin{axis}[
          no markers,
          width=0.8\linewidth, 
          height = 5.0cm,
          xlabel={$L_\text{strand},~\text{m}$},
          ylabel={$T,~\text{K}$},
          xmin=0.0,
          ymin=0.0,
          xmax=1.0,
          legend pos=north east
          ]
        %   Initial temperature curve
          \addplot[smooth, black] table[x=posx,y=t_0_0_ans,col sep=comma] {sections/1D_quench_modelling/figures/results_no_insulation/Temp_tstep_10ms_1e4elems_f2_2.csv};
          
        %   COMSOL plots
          \addplot[smooth, red] table[x=posx,y=t_0_03_com,col sep=comma] {sections/1D_quench_modelling/figures/results_no_insulation/Temp_tstep_10ms_1e4elems_f2_2.csv};
          \addplot[smooth, red] table[x=posx,y=t_0_06_com,col sep=comma] {sections/1D_quench_modelling/figures/results_no_insulation/Temp_tstep_10ms_1e4elems_f2_2.csv};
          \addplot[smooth, red] table[x=posx,y=t_0_1_com,col sep=comma] {sections/1D_quench_modelling/figures/results_no_insulation/Temp_tstep_10ms_1e4elems_f2_2.csv};

        %   ANSYS plots
          \addplot[smooth, blue] table[x=posx,y=t_0_03_ans,col sep=comma] {sections/1D_quench_modelling/figures/results_no_insulation/Temp_tstep_10ms_1e4elems_f2_2.csv};
          \addplot[smooth, blue] table[x=posx,y=t_0_06_ans,col sep=comma] {sections/1D_quench_modelling/figures/results_no_insulation/Temp_tstep_10ms_1e4elems_f2_2.csv};
          \addplot[smooth, blue] table[x=posx,y=t_0_1_ans,col sep=comma] {sections/1D_quench_modelling/figures/results_no_insulation/Temp_tstep_10ms_1e4elems_f2_2.csv};
          
          \legend{
          $T_\text{init}$ profile,
          COMSOL,,,
          ANSYS
          }
        \end{axis}
        
        \draw[black, thick, ->] (2,3) -- (3,3);
        \node[scale = 1] at (3.8, 3) {$\vec{v}_\text{quench}$};   
        
    \end{tikzpicture}
    \caption{Temperature distribution calculated in COMSOL and ANSYS for three time steps: $t=\{0.03, 0.06, 0.1\}$~s with a specified direction of quench velocity, $\vec{v}_\text{quench}$.}
    \label{fig: 1d_no_insulation_temp_along_strand_comparison}
\end{figure}

The quench velocity was calculated by comparing the position of the quench front at $t=0.06~\text{s}$ and $t=0.1~\text{s}$. 
The relative error was calculated as
\begin{equation}
    E_\text{r} = \frac{r_\text{ANSYS}-r_\text{COMSOL}}{r_\text{COMSOL}}~100\%,
    \label{eqn:relative_error_comsol_ansys_benchmarking}
\end{equation}
where $r_\text{ANSYS}$ -- nodal results from ANSYS, $r_\text{COMSOL}$ -- nodal results from COMSOL. The relative error was estimated for:

\begin{itemize}
    \item quench velocity, as presented in Table \ref{table: 1d_no_insulation_v_quench_comparison},
    \item temperature along the strand length at $t=0.1~\text{s}$, as presented in Fig. \ref{fig: ans_comsol_comparison_rel_error_temp_f_2_2}.
\end{itemize}

\begin{table}[H]
    \caption{Quench velocity comparison in COMSOL and ANSYS.} 
    \vspace{-1.em} 
    \fontsize{10}{10}
    \selectfont 
    \renewcommand{\arraystretch}{1.5}
    \begin{center}
        \begin{tabular}{ ccc }  
        \hline
        parameter & value & unit \\
        \hline
        $v_\text{quench, COMSOL}$ & 7.075 & [m/s] \\
        $v_\text{quench, ANSYS}$ & 6.968 & [m/s] \\
        relative error & -1.519 & [\%] \\
        \hline 
        \end{tabular}
    \end{center}  
     \label{table: 1d_no_insulation_v_quench_comparison} 
 \end{table}

\begin{figure}[H]
\centering
    \begin{tikzpicture}
        \begin{axis}[
          width=0.7\linewidth, 
          height = 4.0cm,
          xlabel={$L_\text{strand},~\text{m}$},
          ylabel={Relative error, \%},
          xmin=0.0,
          xmax=1.0
          ]
          \addplot[blue, mark=*] table[x=posx,y=error_0_1,col sep=comma] {sections/1D_quench_modelling/figures/results_no_insulation/Temp_tstep_10ms_1e4elems_f2_2_error.csv};
        \end{axis}
    \end{tikzpicture}
    \caption{Relative error along the strand for $t=0.1~\text{s}$.}
    \label{fig: ans_comsol_comparison_rel_error_temp_f_2_2}
\end{figure}

The difference in hot spot temperature at $x=0~\text{m}$ does not exceed 0.06 \%. The quench velocity is slower the the ANSYS model by less than 2~\% which results in the increase of relative error up to 20~\% at the quench front at $t=0.1~\text{s}$.