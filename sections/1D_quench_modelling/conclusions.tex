
Both simulation environments show the similar results when it comes to the problem of 1D quench thermal propagation at cryogenic temperatures. The relative error of different parameters for two software oscillated in the range of 2\%. In every analysis, a relatively dense mesh was applied of 10 nodes/mm. In  \cite{paudel_thesis}, it is recommended to discretise the domain in the scale less than or equal to a millimetre when an adiabatic analysis is conducted. 
With 10 times less dense mesh, all the analyses with insulation and without showed less than 0.5\% of a relative error. 

As concluded, in order to solve a one-metre domain, at least 1000 elements should be applied. At this point, it is interesting to estimate the scale of the thermal problem to be solved for 3-dimensional magnet geometries in which the cable length of all windings together can reach several hundreds of metres. By adding to this the influence of insulation and material non-linearities, the computing time of such analyses will easily reach at least a couple of days. Therefore, a numerical method aiming at reducing this nodal domain is strongly desired.
