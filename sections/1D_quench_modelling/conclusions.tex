
The comparison of the quench velocity and the hot-spot temperature is presented in Table~\ref{table: 1d_quench_propagation_modelling_conclusion_comparison}. The maximum relative error is not larger than 2\% in the analyses with insulation and epoxy resin, and 2.5\% in the simulations of a bare strand. In both ANSYS and COMSOL, the mesh density study is conducted to verify its influence on the variation of the nodal temperature along the strand. The analysis indicates that a 10 times coarser longitudinal mesh leads in each case to a relative error of a nodal temperature distribution along the strand lower than 0.5\% with respect to the results in which a denser meshes is applied. The remaining error between COMSOL and ANSYS is a consequence of the convergence discrepancy. Moreover, both software are commercial in which the access to various parameters related to the solver settings is either limited or difficult to compare between one and another. Despite the error, it is concluded that the ANSYS model is verified with respect to STEAM-BBQ tool. 

\begin{table}[H]
    \caption{Comparison of the quench velocity and the hot-spot temperature between COMSOL and ANSYS.} 
    \vspace{-1.em} 
    \fontsize{10}{10}
    \selectfont 
    \renewcommand{\arraystretch}{1.5}
    \begin{center}
        \begin{tabular}{ | c | cL | cL | }  
        \hline
          & \multicolumn{2}{c|}{bare strand} & \multicolumn{2}{c|}{strand with insulation and resin} \\ 
         & $v_\text{quench},~\frac{\text{m}}{\text{s}}$ & $T_\text{hot-spot}$ at $t=0.1~\text{s}$,~K & $v_\text{quench},~\frac{\text{m}}{\text{s}}$ & $T_\text{hot-spot}$ at $t=0.1~\text{s}$,~K \\
         \hline
        COMSOL & 7.075 & 28.188 & 2.308 & 23.408 \\
        ANSYS & 6.968& 28.174 & 2.310 & 23.926 \\
        \hline
        $E_\text{r}$, \% & -1.519 & -0.050 & 0.108 & 2.213 \\
        \hline 
        \end{tabular}
    \end{center}  
     \label{table: 1d_quench_propagation_modelling_conclusion_comparison} 
 \end{table}

As advised in~\cite[p.~40]{paudel_thesis} the mesh size of 1~mm should be used along the strand to simulate the 1D adiabatic quench propagation. If the insulation layer is considered, the total number of nodes is estimated as
\begin{equation}
    n_\text{nodes, total} = n_\text{nodes, strand}~(1+ 4~n_\text{nodes, insulation}),
    \label{eqn:tot_number_of_nodes}
\end{equation}
where $n_\text{nodes, strand}$ -- total number of nodes along the strand, $n_\text{nodes, insulation}$ -- number of nodes across the insulation layer. With 6 nodes across the insulation, the total number of nodes for an 812~m-long cable of the skew quadrupole accounts for approximately 20 million nodes. Taking into account that this is not even a fully 3D problem, the number is relatively large. As presented in Fig. \ref{fig: quench_propagation_conclusion_computing_time_estimation}, the computing time is estimated for the analysis with the mesh size of 1 mm and 6 elements across the insulation. Three analyses are conducted with a varying length of the strand. The analysis of a 10 metre-long domain lasts more than 7 hours. The linear interpolation of the results indicates that, the simulation of a 100~m-long cable would approximately last 72 hours\footnote{The analysis was performed on the following calculation unit: Intel(R) Xeon(R) CPU E5-2667 V4 @~3.20 GHz (2~processors) with RAM 128 GB.}.

\begin{figure}[H]
\centering
    \begin{tikzpicture}
        \begin{axis}[
          width=0.7\linewidth, 
          height = 4.0cm,
          xlabel={$\bar{x},~\text{m}$},
          ylabel={$t$, s},
          xticklabel style={/pgf/number format/fixed},
          xtick={0.1, 1, 10, 100},
          xmode=log,
          ymode=log,
          legend pos=outer north east
          ]
          \addplot[blue, mark=*] table[x=nodes,y=time,col sep=comma] {sections/1D_quench_modelling/figures/other/computing_time_estimation_vs_strand_length.csv};
          \addplot[red, dashed] table[x=nodes,y=time,col sep=comma] {sections/1D_quench_modelling/figures/other/computing_time_estimation_trendline.csv};
          \legend{
          computing time,
          linear approximation
          }
        \end{axis}
    \end{tikzpicture}
    \caption{Computing time as a function of number of nodes.}
    \label{fig: quench_propagation_conclusion_computing_time_estimation}
\end{figure}

One of the methods for reducing the discretised domain is the application of an adaptive mesh which locally refines space remaining in a close contact with the moving quench front. Such a solution would be applicable if a 1D analysis was taken into consideration. The aim of this work is to conduct a multidimensional analysis with both longitudinal and transverse quench propagation. This requires a 3D adaptive mesh refinement, which would be very difficult to perform.
