
In this chapter, there are two studies conducted for a quench propagation in a 1D strand: 

\begin{enumerate}
    \item Analysis of a bare composite strand
    \item Analysis of the composite strand with insulation and epoxy resin
\end{enumerate}

For both cases, the mesh density study is conducted. The following assumptions are made in the simulations: 

\begin{enumerate}
    \item There is no helium cooling.
    \item Temperature in the cross-section of a composite strand is uniform.
    \item Turn-to-turn propagation does not occur between different windings of a coil.
    \item When the insulation and epoxy resin are considered, the longitudinal heat transfer outside of the bare strand is neglected.
\end{enumerate}

The second and third assumptions allow one to consider the quench simulation as a 1D longitudinal heat propagation. When the strand is analysed with an external insulation and epoxy resin, the analysis becomes a 1D+1D heat conduction problem because of the fourth assumption. It is further explained in Section \ref{section: 1D_quench_propagation_with_insulation}. In presented thermal problems, the heat balance partial differential equation is solved as
\begin{equation}
    \gamma c_p(T,B) \frac{\partial T}{\partial t} = \frac{\partial}{\partial \vec{r}}[k(T,B) \frac{\partial T}{\partial \vec{r}}] + q_0(T,B),
    \label{eqn:PDE_heat_balance}
\end{equation}
where $T$ -- temperature varying in time and space defined by a position vector $\vec{r}$, $\gamma$ -- mass density of a material, $c_p$ -- specific heat of a material, $k$ -- thermal conductivity of a material, $q_v$~-- external heat source (Joule heating). Specific heat capacity, thermal conductivity, and resistivity are a function of both temperature and magnetic field strength in the strand. The insulation of the strand and, optionally, resin are only a function of temperature. The domain is thermally anisotropic when the superconducting strand is analysed with an external insulation layer and, optionally, epoxy resin.