
% NbTi mass density 
\subsection{Mass Density}
The mass density is assumed to be constant and equal to $\rho = 6000~\text{kg/m}^{3}$.

% NbTi specific heat capacity
\subsection{Volumetric Heat Capacity}
Volumetric heat capacity of Nb-Ti is estimated by means of a piece-wise polynomial function which takes into account a linear magnetic dependence below the critical temperature. The critical temperature is calculated as: 

\begin{equation}
    T_\text{c}(B) = T_\text{c0}\cdot(1-\frac{B}{B_\text{c20}})^{0.59},
    \label{eqn:critical_temperature_appendix}
\end{equation}
where $T_\text{c0}$ -- maximum critical temperature for $B=0~\text{T}$, $B_\text{c20}$ -- maximum upper critical magnetic field for $T=0~\text{K}$. Their numerical values for Nb-Ti are presented in Table~\ref{table: appendix_nb_ti_crit_temp_params}. In both parameters, the current density is equal to zero. 

\begin{table}[H]
    \caption{Critical temperature parameters for Nb-Ti~\cite{empirical_scaling_formulas_for_critical_current}.} 
    \vspace{-1.em} 
    \fontsize{10}{10}
    \selectfont 
    \renewcommand{\arraystretch}{1.5}
    \begin{center}
        \begin{tabular}{ ccc }  
        \hline
        parameter & value & unit \\
        \hline
        $T_\text{c0}$ & 9.2 & [K] \\
        $B_\text{c20}$ & 14.5 & [T] \\
        \hline 
        \end{tabular}
    \end{center}  
     \label{table: appendix_nb_ti_crit_temp_params} 
 \end{table}

The volumetric heat capacity is calculated in the following manner: 
\begin{equation}
    c_\text{v, Nb-Ti}(T, T_\text{c}) = \frac{\text{a}_0 + \text{a}_1\cdot T^{1} + \text{a}_2\cdot T^{2} + \text{a}_3\cdot T^{3}+ \text{a}_4\cdot T^{4}} {\rho_{Nb-Ti}},
\end{equation}
where $a$ -- fit parameters presented in Table \ref{table:nbti_parameters}.

\begin{table}[H]
    \caption{Polynomial parameters for Nb-Ti volumetric heat capacity.} 
    \vspace{-1.em} 
    \fontsize{10}{10}
    \selectfont 
    \renewcommand{\arraystretch}{1.5}
    \begin{center}
    \begin{tabular}{ lccccc }  
    \hline
    & $\text{a}_0$ & $\text{a}_1$ & $\text{a}_2$ & $\text{a}_3$ & $\text{a}_4$ \\
    \hline
    $T \in (0, T_\text{c})$~K & 0 & $64 \cdot B$ & 0 & 49.1 & 0 \\        
    $T \in [T_\text{c}, 28.358)$~K & 0 & 928 & 0 & 16.24 & 0 \\        
    $T \in [28.358, 50.99)$~K & 41383 & -7846.1 & 553.71 & 11.9838 & -0.2177 \\        
    $T \in [50.99, 165.8)$~K & $-1.53\cdot10^{6}$ & 83022 & -716.3 & 2.976 & -0.00482 \\ 
    $T \in [165.8, 496.54)$~K & $1.24\cdot10^{6}$ & 13706 & -51.66 & 0.09296 & $-6.29\cdot10^{-5}$ \\        
    $T \in [496.54, 1000)$~K & $2.45\cdot10^{6}$ & 955.5 & -0.257 & 0 & 0 \\       
    \hline
     \end{tabular} 
    \end{center}  
     \label{table:nbti_parameters} 
\end{table}

As presented in Fig. \ref{fig:nbti_cv_plot}, there occurs a step jump in Nb-Ti heat capacity at critical temperature. Since the critical temperature decreases with rise of magnetic field strength, the transition occurs at lower temperatures when a higher B-field is applied.
 
\begin{figure}[H]
    \centering
    \begin{tikzpicture}[spy using outlines=
	{circle, magnification=8, connect spies}]
    \begin{axis}[
      no markers,
      width=0.7\linewidth, 
      height = 4cm,
      xlabel={$T,~\text{K}$},
      ylabel={$c_\text{v},~\frac{\text{J}}{\text{m}^3~\text{K}}$},
      xtick={8.5, 50, 100, 150, 200, 250, 300},
      xmin=0.0,
      ymin=0.0,
      xmax=300.0,
      legend pos=outer north east
      ]
      \addplot[smooth, blue] table[x=temperature,y=B0,col sep=comma] {sections/appendices/material_properties/niobium_titanium/figures/nb_ti_cv.csv}; 
      \addplot[smooth, red] table[x=temperature,y=B3,col sep=comma] {sections/appendices/material_properties/niobium_titanium/figures/nb_ti_cv.csv}; 
      
      \coordinate (spypoint) at (8.5,20000);
      \coordinate (magnifyglass) at (100, 1000000);
    
      \legend{
      \textit{B}=0 T,
      \textit{B}=3 T}
    
    \end{axis}
    \spy [black, size=1.5cm] on (spypoint)
      in node[fill=white] at (magnifyglass);
    \end{tikzpicture}
    \caption{Volumetric heat capacity of Ni-Ti as a function of temperature.}
    \label{fig:nbti_cv_plot}
\end{figure}