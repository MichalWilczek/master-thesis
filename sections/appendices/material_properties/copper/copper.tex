
\subsection{RRR Definition}
\label{appendix:subsection_rrr_definition}
Residual resistivity ratio, RRR is a parameter comparing copper resistivity at two different temperatures when no magnetic field is applied. It is calculated as: 
\begin{equation}
RRR = \frac{\rho(T_\text{high}, B=0~\text{T})}{\rho(T_\text{low}, B=0~\text{T})},
\end{equation}
where in NIST $T_\text{high}=273~\text{K}$ whereas $T_\text{low}=4~\text{K}$.

\subsection{Resistivity}
Copper resistivity depends on temperature and RRR:
\begin{equation}
    \rho_\text{N}(T, \text{RRR}) = \rho_\text{0}+\rho_\text{i}+\rho_\text{i0},
\end{equation}
\\
where:
\begin{equation}
    \rho_\text{0} = \frac{1.5553\cdot10^{-8}}{\text{RRR}},
\end{equation}

\begin{equation}
    \rho_\text{i} = \frac{\text{P}_\text{1}}{1+\text{P}_\text{1}  \text{P}_\text{3}, T^{\text{P}_\text{2} - \text{P}_\text{4}}, \exp{[-(\frac{\text{P}_\text{5}}{T}}^{\text{P}_\text{6}})]},
\end{equation}

\begin{equation}
    \rho_\text{i0} = \text{P}_\text{7} \frac{\rho_\text{i} \rho_\text{0}}{\rho_\text{i} + \rho_\text{0}},
\end{equation}
with fit parameters depicted in Table \ref{table:nist_resistivity_parameters}. 

\begin{table}[H]
    \caption{Fit P-parameters for copper electrical resistivity} 
    \vspace{-1.em} 
    \fontsize{10}{10}
    \selectfont 
    \renewcommand{\arraystretch}{1.5}
    \begin{center}
    \begin{tabular}{ ccccccc }  
    \hline
    $\text{P}_1$ & $\text{P}_2$ & $\text{P}_3$ & $\text{P}_4$ & $\text{P}_5$ & $\text{P}_6$ & $\text{P}_7$ \\
    \hline
    $1.171\cdot10^{-17}$ & 4.49 & $3.841\cdot10^{10}$ & 1.14 & 50 & 6.428 & 0.4531 \\
    \hline 
    \end{tabular}
    \end{center}  
     \label{table:nist_resistivity_parameters} 
 \end{table}
 
In order to apply the resistivity dependence on magnetic field strength, the following formula is applied:

\begin{equation}
    \rho_\text{N}(T, B, \text{RRR}) = \rho_\text{N}(T, \text{RRR}) \cdot (1 + 10^{a(x)}),  
\end{equation}
where:
\begin{equation}
    a(x) = \sum_{n=0}^{N} a_\text{n}(\log_\text{10}x)^{n},
\end{equation}

\begin{equation}
    x \approx \frac{1.553 \cdot 10^{-8}}{\rho_\text{N}(T, \text{RRR})} \cdot B,
\end{equation}
with $B$ -- magnetic field strength, $a$ -- fit parameters described in Table \ref{table:nist_resistivity_parameters2}.

\begin{table}[h!]
    \caption{Fit a-parameters for copper electrical resistivity} 
    \vspace{-1.em} 
    \fontsize{10}{10}
    \selectfont 
    \renewcommand{\arraystretch}{1.5}
    \begin{center}
    \begin{tabular}{ ccccc }  
    $\text{a}_0$ & $\text{a}_1$ & $\text{a}_2$ & $\text{a}_3$ & $\text{a}_4$ \\
    \hline
    -2.662 & 0.3168 & 0.6229 & -0.1839 & 0.001827 \\
    \hline
    \end{tabular}
    \end{center}  
     \label{table:nist_resistivity_parameters2} 
 \end{table}

As presented in Fig. \ref{fig:cu_resistivity_plot}, the resistivity of copper rises with rise of temperature. It rises with increase of magnetic fields strength as well as with decrease of RRR.

\begin{figure}[H]
\centering
\begin{tikzpicture}
\begin{axis}[
  no markers,
  legend style={at={(1,0)},anchor=south east},
  grid style={dashed,gray!30},
  width=0.85\linewidth, 
  height = 6cm,
  xlabel={$T,~\text{K}$},
  ylabel={$\rho,~\Omega~\text{m}$},
  xlabel style={below right},
  ylabel style={above left},
  xmin=0.0,
  ymin=0.0,
  xmax=100.0,
  legend pos=north west
  ]
  \addplot table[x=Time,y=B_0_0_RRR_200,col sep=comma] {sections/appendices/material_properties/copper/figures/cu_resistivity.csv}; 
  \addplot table[x=Time,y=B_3_0_RRR_200,col sep=comma] {sections/appendices/material_properties/copper/figures/cu_resistivity.csv}; 
  
  \legend{
  \textit{B}(RRR=200)=0 T,
  \textit{B}(RRR=200)=3 T}

\end{axis}
\end{tikzpicture}
\caption{Copper resistivity temperature dependence for two values of magnetic field.}
    \label{fig:cu_resistivity_plot}
\end{figure}

%  thermal conductivity
\subsection{Thermal Conductivity}
Copper thermal conductivity depends on temperature and RRR:
\begin{equation}
    k_\text{N}(T, \text{RRR}) = \frac{1}{\text{W}_\text{0} + \text{W}_\text{i} + \text{W}_\text{i0}}, 
\end{equation}
where:
\begin{equation}
    \text{W}_\text{0} = \frac{\beta}{T},
\end{equation}

\begin{equation}
    \text{W}_\text{i} = \frac{\text{P}_\text{1} T^{\text{P}_\text{2}}}{1+\text{P}_\text{1}  \text{P}_\text{3}  T^{\text{P}_\text{2} + \text{P}_\text{4}}  \exp{[-(\frac{\text{P}_\text{5}}{T}}^{\text{P}_\text{6}})]},
\end{equation}

\begin{equation}
    \text{W}_\text{i0} = \text{P}_\text{7} \frac{\text{W}_\text{i} \text{W}_\text{0}}{\text{W}_\text{i} + \text{W}_\text{0}},
\end{equation}
\\
with fit parameters presented in Table \ref{table:nist_cu_k_parameters}.

\begin{table}[H]
    \caption{Fit \textbeta- and P-parameters for copper thermal conductivity} 
    \vspace{-1.em} 
    \fontsize{10}{10}
    \selectfont 
    \renewcommand{\arraystretch}{1.5}
    \begin{center}
    \begin{tabular}{ cc }  
    $\upbeta$ & $\upbeta_\text{r}$ \\
    \hline
    $0.634/\text{RRR}$ & $\upbeta/0.0003$ \\
    \hline
    \end{tabular}
    
    \begin{tabular}{ ccccccc }  
    $\text{P}_1$ & $\text{P}_2$ & $\text{P}_3$ & $\text{P}_4$ & $\text{P}_5$ & $\text{P}_6$ & $\text{P}_7$ \\
    \hline
    $1.754\cdot10^{-8}$ & 2.763 & 1102 & -0.165 & 70 & 1.756 & $0.838/\upbeta_\text{r}^{0.1661}$ \\
    \hline 
    \end{tabular}
    \end{center}  
     \label{table:nist_cu_k_parameters} 
 \end{table}
 
 In order to include the magnetic induction dependence in the function of thermal conductivity, the following formula is applied: 
\begin{equation}
    k_\text{N}(T, B, \text{RRR}) = \frac{\rho\text{N}(T, B=0, \text{RRR})}{\rho\text{N}(T, B=0, \text{RRR})} \cdot k_\text{N}(T, \text{RRR})
\end{equation}

As presented in Fig. \ref{fig:cu_k_plot}, the thermal conductivity of copper decreases with rise of magnetic field strength. One can also notice that the peak value of thermal conductivity is higher with increase of RRR. 
 
\begin{figure}[H]
\centering
\begin{tikzpicture}
\begin{axis}[
  no markers,
  legend style={at={(1,0)},anchor=south east},
  grid style={dashed,gray!30},
  width=0.85\linewidth, 
  height = 6cm,
  xmode=log,
  xlabel={$T,~\text{K}$},
  ylabel={$k,~\frac{\text{W}}{\text{m K}}$},
  xlabel style={below right},
  ylabel style={above left},
  xmax=300.0,
  ymin=0.0,
  ymax=4200.0,
  legend pos=north west
  ]
  \addplot[smooth, blue] table[x=temperature,y=B0,col sep=comma] {sections/appendices/material_properties/copper/figures/cu_thermal_conductivity_rrr_200.csv}; 
  \addplot[smooth, red] table[x=temperature,y=B3,col sep=comma] {sections/appendices/material_properties/copper/figures/cu_thermal_conductivity_rrr_200.csv};

  \addplot[smooth, dashed, blue] table[x=temperature,y=B0,col sep=comma] {sections/appendices/material_properties/copper/figures/cu_thermal_conductivity_rrr_50.csv};
  \addplot[smooth, dashed, red] table[x=temperature,y=B3,col sep=comma] {sections/appendices/material_properties/copper/figures/cu_thermal_conductivity_rrr_50.csv};

  \legend{
  \textit{B}(RRR=200)=0 T,
  \textit{B}(RRR=200)=3 T,
  \textit{B}(RRR=50)=0 T,
  \textit{B}(RRR=50)=3 T}

\end{axis}
\end{tikzpicture}
\caption{Thermal conductivity of copper as a function of temperature for two values of magnetic field and RRR.}
    \label{fig:cu_k_plot}
\end{figure}

% mass density 
\subsection{Mass Density}
The mass density of copper is assumed to be constant and equal to $\rho = 8960~\text{kg/m}^{3}$.

% Volumetric heat capacity 
\subsection{Volumetric Heat Capacity}
Copper specific heat capacity is approximated with a polynomial interpolation as: 
\begin{equation}
    x(T) = 10^{\sum_{n=0}^{N} a_\text{n}(\log_\text{10}T)^{n}},
\end{equation}
\\
where $N$ -- order of polynomial, a -- fit parameters described in Table \ref{table:nist_cu_cp_parameters}. 

\begin{table}[H]
    \caption{Fit parameters for copper specific heat capacity} 
    \vspace{-1.em} 
    \fontsize{10}{10}
    \selectfont 
    \renewcommand{\arraystretch}{1.5}
    \begin{center}
    \begin{tabular}{ cccccccc }  
    $\text{a}_0$ & $\text{a}_1$ & $\text{a}_2$ & $\text{a}_3$ & $\text{a}_4$ & $\text{a}_5$ & $\text{a}_6$ & $\text{a}_7$ \\
    \hline
    -1.91844 & -0.15973 & 8.61013 & -18.996 & 21.9661 & -12.7328 & 3.54322 & -0.3797 \\
    \hline 
    \end{tabular}
    \end{center}  
     \label{table:nist_cu_cp_parameters} 
 \end{table}
 
In order to convert specific into volumetric heat capacity, the polynomial function is divided by copper density. The volumetric heat capacity of copper is presented in Fig. \ref{fig:cu_cv_plot}.

\begin{figure}[H]
\centering
\begin{tikzpicture}
\begin{axis}[
  no markers,
  legend style={at={(1,0)},anchor=south east},
  grid style={dashed,gray!30},
  width=0.85\linewidth, 
  height = 6cm,
  xlabel={$T,~\text{K}$},
  ylabel={$C_\text{v},~\frac{\text{J}}{\text{m}^3~\text{K}}$},
  xlabel style={below right},
  ylabel style={above left},
  xmin=0.0,
  ymin=0.0,
  xmax=300.0,
  legend pos=north west
  ]
  \addplot[smooth, blue] table[x=temperature,y=cvcu,col sep=comma] {sections/appendices/material_properties/copper/figures/cu_cv.csv}; 

\end{axis}
\end{tikzpicture}
    \caption{Volumetric heat capacity of copper as a function of temperature.}
    \label{fig:cu_cv_plot}
\end{figure}