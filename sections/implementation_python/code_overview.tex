
As explained in Chapter~\ref{chapter:quench_velocity_modelling}, the co-simulation in a quench velocity-based approach requires using an external routine that controls the quench propagation in time. This routine was developed in Python. The routine has four main functions: 

\begin{enumerate}
    \item Translation of a Python code into ANSYS APDL internal commands in order to allow the external routine to communicate with the software. 
    \item Implementation of pre-processing, solution and post-processing steps in the thermo-electric simulations conducted in ANSYS APDL.
    \item Implementation of all algorithms explained in Chapter~\ref{chapter:algorithms}.
    \item Execution of a quench velocity-based analysis by controlling the propagation of quench in time. 
\end{enumerate}

The first three points allow the Python code to conduct not only a quench velocity-based analysis, but also a standard thermal quench simulation. Therefore, in this thesis, all simulations are controlled by one external Python script. The exemplary code depicting how the Python communicates with ANSYS APDL is described in Appendix~\ref{appendix:python_ansys_compilation}. In order to run the programme, Python 3.7 is required as well as ANSYS APDL~2019~R1. The~programme is open source and available on GitLab\footnote{GitLab is an official code repository at CERN.} under the link given in~\cite{my_python_code}.

As presented in Fig.~\ref{fig:block_diagram_python_architecture}, the script is divided into four main components: execution script, general purpose classes, workflow classes and main classes. The programme uses a~text file in which all input parameters required for the analysis are specified. The execution script is a core of the programme performing consecutive steps of the analysis. The~workflow classes are created to execute the code similarly to ANSYS APDL, i.e. the procedures corresponding to individual methods are collected in the steps of preprocessing, solving and postprocessing as it is done in GUI\footnote{GUI -- Graphical User Interface} of ANSYS APDL. Therefore, the script remains intuitive with respect to the ANSYS commercial tool. The workflow classes use the main classes in which the core methods of the Python script are implemented. The general purpose classes are used by most of the remaining classes. The analysis engine chooses a specific case for the analysis based on the input text file filled by a user. 

\begin{figure}[H]
    \renewcommand{\baselinestretch}{0.7} 
    \centering
    \begin{tikzpicture}[node distance = 1.5cm, auto]
        \tikzstyle{decision} = [diamond, draw, fill=blue!20, text width=3cm, text badly centered, node distance=2.5cm, inner sep=0pt, scale=0.8]
        \tikzstyle{block} = [rectangle, draw, fill=blue!20, text width=3.2cm, text centered, rounded corners, node distance=0.85cm, scale=0.8, minimum height=0.5cm]
        \tikzstyle{line} = [draw, -latex']
        \tikzstyle{cloud} = [draw, ellipse,fill=red!20, node distance=1.5cm, minimum height=2em, scale=0.8]
  
    \node [block] (analysis_engine) {analysis\_engine};
    \node [block, below of=analysis_engine] (common functions) {common\_functions};
    \node [block, right of=analysis_engine, node distance=5cm] (preprocessor) {preprocessor};
    \node [block, below of=preprocessor] (solver) {solver};
    \node [block, below of=solver] (postprocessor) {postprocessor};
     
    \node [block, above of=preprocessor, node distance=1.5cm, fill=red!20] (execution_script) {execution script};
    \node [cloud, above of=execution_script] (json_file) {Input text file};
    \node [block, below of=postprocessor, node distance=1.5cm] (ansys_interface) {ansys\_interface};
    \node [block, below of=ansys_interface] (geometry) {geometry};
    \node [block, below of=geometry] (materials) {materials};
    \node [block, below of=materials] (physics) {physics};

    \node [block, right of=physics, node distance=5cm] (magnetic field) {magnetic\_field};
    \node [block, above of=magnetic field] (initial temperature) {initial\_temperature};
    \node [block, above of=initial temperature] (circuit) {circuit};
    \node [block, below of=magnetic field] (quench velocity) {quench\_velocity};
    \node [block, below of=quench velocity] (time stepping) {time\_stepping};

    \path [line] (circuit) -| (6,-4) |- (physics);
    \path [draw] (initial temperature) -- (6,-3.9);
    \path [draw] (magnetic field) -- (6,-4.6);
    \path [draw] (quench velocity) -- (6,-5.3);
    \path [draw] (time stepping) -| (6,-4.6);
    
    \node [rectangle, draw, dashed, rounded corners, node distance=0.85cm, below of=preprocessor, minimum height=2.5cm, text width=3cm, node distance=0.75cm] (solver_block) { };
    \node [rectangle, draw, dashed, rounded corners, minimum height=4.2cm, text width=7cm] at (6, -4.3) { };
    \node [rectangle, draw, dashed, rounded corners, minimum height=1.8cm, text width=3cm] at (0, -0.35) { };

    \path [line] (json_file) -- (execution_script);
    \path [line] (execution_script) -- (solver_block);

    \node [scale=0.8, text width=3.25cm] at (7.1,0) {workflow classes};
    \node [scale=0.8, text width=3.25cm] at (9,-1.9) {main classes};
    \node [scale=0.8, text width=3.25cm] at (-0.25,1) {general purpose classes};

    \end{tikzpicture}
    \caption{Schematic of the Python script.}
    \label{fig:block_diagram_python_architecture}
\end{figure}

As described in Table~\ref{table:python_analysis_options}, the user can conduct a standard analysis and a quench velocity-based one. The Python script is able to create two types of geometries: a high-order corrector magnet as well a stack of extruded strands. The analysis may be executed with or without the insulation layer. The initial temperature is a~Gaussian profile or a~step function. The script has a repository of thermal and electric material properties of copper, Nb-Ti and G10 based on fits provided by NIST (see Appendix~\ref{appendix_material_properties_description}). When a quench velocity-based approach is considered, the user can specify quench velocity as a~constant value or a~function of current and magnetic field. The multi-strand simulations are dedicated to use a quench velocity-based approach. The quench velocity-based case is equipped with a tool able to simulate the discharge of a magnet with a varying inductance as a function of current and the dump resistor connected in series with a magnet. Since some of the material properties of copper and Nb-Ti (see in Appendix~\ref{appendix_material_properties_description}) are a function of the magnetic field, the user can specify whether the magnetic field is constant or -- a function of a position in ($x$, $y$)-plane from the magnet cross-section, as presented in Fig.~\ref{fig: 3d_coil_illustation_with_2d_b_field}. One can specify multiple maps for different values of current.

\begin{table}[H]
    \caption{Analysis options in the analysis} 
    \vspace{-1.em} 
    \fontsize{10}{10}
    \selectfont 
    \renewcommand{\arraystretch}{1.5}
    \begin{center}
        \begin{tabular}{ c | c | m{5cm} }  
        \hline
        \textbf{analysis type} & \textbf{standard} & \textbf{quench-velocity based} \\
        \hline
        geometry & \multicolumn{2}{c}{3D high-order corrector,  extruded (multi-)strand} \\
        \hline
        analysis with insulation & \multicolumn{2}{c}{available} \\
        \hline
        initial temperature profile & \multicolumn{2}{c}{step function, Gaussian profile} \\
        \hline
        available material properties & \multicolumn{2}{c}{Cu, Nb-Ti, G10 (all based on NIST)} \\
        \hline
        quench velocity model & N/A & constant, function of current and B-field \\
        \hline
        circuit analysis & constant current & constant current, RL discharge with dump resistor \\ 
        \hline
        B-field analysis & constant & constant, function of current and position in 2D space
        \end{tabular}
    \end{center}  
     \label{table:python_analysis_options} 
 \end{table}
