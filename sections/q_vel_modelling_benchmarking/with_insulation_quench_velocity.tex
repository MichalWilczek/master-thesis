
Similarly to Section~\ref{subsection:quench_velocity_benchmarking_no_insulation_quench_velocity}, the electro-thermal element, LINK68 is used along the strand. The insulation layer is represented by a thermal element LINK33 since no Joule effect occurs in this zone. Two time step ranges were chosen: $t=\{[10, 100], [100, 1000]\}~\upmu \text{s}$. The study over a 1 metre-long domain was conducted with a varying number of nodes, $n=\{50, 100, 1000\}$. The geometric assumptions as well as initial conditions were the same as in Section \ref{subsection: 1D_quench_propagation_with_insulation} without a heat source. The additional parameters related to the co-simulation are shown in Table \ref{table:1d_qv_benchmarking_geometry__with_insulation_parameters_quench_velocity}. In order to observe convergence of relative errors, the total simulation time was extended to $t=0.3~\text{s}$

\begin{table}[H]
    \caption{Analysis input parameters.} 
    \vspace{-1.em} 
    \fontsize{10}{10}
    \selectfont 
    \renewcommand{\arraystretch}{1.5}
    \begin{center}
        \begin{tabular}{ ccc }  
        \hline
        communication time step & 0.0025 & [s] \\
        average quench velocity & 2.18 & [m/s] \\
        total simulation time & 0.3 & [s] \\
        \hline 
        \end{tabular}
    \end{center}  
    \label{table:1d_qv_benchmarking_geometry__with_insulation_parameters_quench_velocity} 
\end{table}

Fig.~\ref{fig:q_vel_benchmarking_temp_distr_over_strand_with_insulation} compares the temperature distribution over a strand domain when 50 nodes are applied with a quench velocity-based methodology. The quench front lags behind the benchmark solution analogously to the case without the insulation layer.

\begin{figure}[H]
\centering
    \begin{tikzpicture}
        \begin{axis}[
          no markers,
          width=0.8\linewidth, 
          height = 5.0cm,
          xlabel={$L_\text{strand},~\text{m}$},
          ylabel={$T,~\text{K}$},
          xmin=0.0,
          ymin=0.0,
          xmax=1.0,
          legend pos=north east
          ]
          
        %   Initial temperature curve for the mesh used in quench velocity modelling 
          \addplot[black] table[x=position,y=t_0,col sep=comma] {sections/q_vel_modelling_benchmarking/figures/results_with_insulation/quench_velocity_50_nodes.csv};
          
        %   Heat Balance Equation plots
          \addplot[red] table[x=position,y=t_0_03,col sep=comma] {sections/q_vel_modelling_benchmarking/figures/results_with_insulation/heat_balance_1000_nodes_benchmark.csv};
          \addplot[red] table[x=position,y=t_0_06,col sep=comma] {sections/q_vel_modelling_benchmarking/figures/results_with_insulation/heat_balance_1000_nodes_benchmark.csv};
          \addplot[red] table[x=position,y=t_0_1,col sep=comma] {sections/q_vel_modelling_benchmarking/figures/results_with_insulation/heat_balance_1000_nodes_benchmark.csv};
        %   \addplot[red] table[x=position,y=t_0_3,col sep=comma] {sections/q_vel_modelling_benchmarking/figures/results_with_insulation/heat_balance_1000_nodes_benchmark.csv};

        %   Quench Velocity Modelling plots
          \addplot[blue] table[x=position,y=t_0_03,col sep=comma] {sections/q_vel_modelling_benchmarking/figures/results_with_insulation/quench_velocity_50_nodes.csv};
          \addplot[blue] table[x=position,y=t_0_06,col sep=comma] {sections/q_vel_modelling_benchmarking/figures/results_with_insulation/quench_velocity_50_nodes.csv};
          \addplot[blue] table[x=position,y=t_0_1,col sep=comma] {sections/q_vel_modelling_benchmarking/figures/results_with_insulation/quench_velocity_50_nodes.csv};
        %   \addplot[blue] table[x=position,y=t_0_3,col sep=comma] {sections/q_vel_modelling_benchmarking/figures/results_with_insulation/quench_velocity_50_nodes.csv};
          
          \legend{
          $T_\text{init}$ profile,
          heat balance,,,,
          quench velocity
          }
        \end{axis}
    \end{tikzpicture}
    \caption{Temperature distribution of a heat balance-based benchmark solution and a quench velocity-based solution with 50 nodes along the domain and 20 nodes across the insulation layer for three time steps: $t=\{0.03, 0.06, 0.1\}$~s.}
    \label{fig:q_vel_benchmarking_temp_distr_over_strand_with_insulation}
\end{figure}

As presented in Fig.~\ref{fig: q_vel_modelling_res_volt_benchmarking_with_insulation}, the resistive voltage rises monotonically for the quench velocity-based analysis and follows the curve similarly to the case without the insulation layer. 

\begin{figure}[H]
\centering
    \begin{tikzpicture}
        \begin{axis}[
          width=0.7\linewidth, 
          height = 4.5cm,
          xlabel={Time, $\text{s}$},
          ylabel={Resistive Voltage, $\text{V}$},
          xticklabel style={/pgf/number format/fixed},
          yticklabel style={/pgf/number format/fixed},
          xmin=0.0,
          xmax=0.3,
          legend pos = north west
          ]
          \addplot[blue, mark=*] table[x=time,y=50_nodes_quench_velocity,col sep=comma] {sections/q_vel_modelling_benchmarking/figures/results_with_insulation/quench_velocity_res_volt_benchmarking.csv};

          \addplot[red] table[x=time,y=heat_balance_benchmark,col sep=comma] {sections/q_vel_modelling_benchmarking/figures/results_with_insulation/quench_velocity_res_volt_benchmarking.csv};

          \legend{
          quench velocity,
          heat balance,
          }
          
        \end{axis}
    \end{tikzpicture}
    \caption{Resistive voltage comparison for standard numerical solution and quench velocity-based simulation with with 50 nodes along the domain.}
    \label{fig: q_vel_modelling_res_volt_benchmarking_with_insulation}
\end{figure}

A double increase in number of nodes from 50 to 100 nodes/m results in a better convergence of a quench velocity-based resistive voltage with respect to the benchmark solution, as presented in Fig.~\ref{fig: q_vel_modelling_res_volt_rel_error_with_insulation}. In such a case, the relative error remains in the range of -10\% and converges to -2.5\% as the quench propagates. When a coarser mesh with 50 nodes is used, the relative error converges to the value of -5\%. 

\begin{figure}[H]
\centering
    \begin{tikzpicture}
        \begin{axis}[
          width=0.7\linewidth, 
          height = 6.0cm,
          xlabel={Time, $\text{s}$},
          ylabel={Relative error, \%},
          xticklabel style={/pgf/number format/fixed},
          xmin=0.0,
          xmax=0.3,
          legend pos=south east
          ]
          \addplot[blue, mark=*] table[x=time,y=50_nodes,col sep=comma] {sections/q_vel_modelling_benchmarking/figures/results_with_insulation/quench_velocity_res_volt_rel_error.csv};
          \addplot[red, mark=*] table[x=time,y=100_nodes,col sep=comma] {sections/q_vel_modelling_benchmarking/figures/results_with_insulation/quench_velocity_res_volt_rel_error.csv};
          \addlegendimage{/pgfplots/refstyle=plot_resistive_voltage}\addlegendentry{50 nodes}
          \addlegendimage{/pgfplots/refstyle=plot_resistive_voltage}\addlegendentry{100 nodes}

        \end{axis}
    \end{tikzpicture}
    \caption{Relative error of resistive voltage for 50 and 100 nodes used for quench velocity-based simulation.}
    \label{fig: q_vel_modelling_res_volt_rel_error_with_insulation}
\end{figure}

Fig.~\ref{fig: q_vel_modelling_hot_spot_rel_error_with_insulation} shows the evolution of relative error at the hot spot. At $t=0.3~\text{s}$, the relative error reaches -1.2\% with the mesh density of 100 nodes. Unfortunately, the clear convergence cannot be deduced which might lead to the conclusion that the error will increase with time. 

\begin{figure}[H]
\centering
    \begin{tikzpicture}
        \begin{axis}[
          width=0.7\linewidth, 
          height = 6.0cm,
          xlabel={Time, $\text{s}$},
          ylabel={Relative error, \%},
          xticklabel style={/pgf/number format/fixed},
          xmin=0.0,
          xmax=0.3,
          legend pos=north east
          ]
          \addplot[blue, mark=*] table[x=time,y=50_nodes,col sep=comma] {sections/q_vel_modelling_benchmarking/figures/results_with_insulation/quench_velocity_hot_spot_rel_error.csv};
          \addplot[red, mark=*] table[x=time,y=100_nodes,col sep=comma] {sections/q_vel_modelling_benchmarking/figures/results_with_insulation/quench_velocity_hot_spot_rel_error.csv};
        %   \addplot[green, mark=*] table[x=time,y=1000_nodes,col sep=comma] {sections/q_vel_modelling_benchmarking/figures/results_with_insulation/quench_velocity_hot_spot_rel_error.csv};
          \addlegendimage{/pgfplots/refstyle=plot_resistive_voltage}\addlegendentry{50 nodes}
          \addlegendimage{/pgfplots/refstyle=plot_resistive_voltage}\addlegendentry{100 nodes}
        %   \addlegendimage{/pgfplots/refstyle=plot_resistive_voltage}\addlegendentry{1000 nodes}
          
        \end{axis}
    \end{tikzpicture}
    \caption{Relative error of hot spot temperature for 50 and 100 nodes used for quench velocity-based simulation.}
    \label{fig: q_vel_modelling_hot_spot_rel_error_with_insulation}
\end{figure}


