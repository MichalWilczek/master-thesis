
Similarly to the analysis of a bare strand employing the quench velocity-based approach, the electro-thermal element LINK68 is used to model a strand. The insulation with epoxy resin is represented by the LINK33 element since no heat is generated in that region. In~order to observe the relevant evolution of a relative error of the resistive voltage and the hot-spot temperature, the total simulation time is extended to $t=0.5~\text{s}$. The study of a~1.5~metre-long cable is performed with a varying longitudinal mesh size, $m=\{1, 10, 20\}~\text{mm}$. The cable is longer than in the analysis a bare strand so that the quench front could propagate during the entire simulation time without reaching the end of the domain. The time step range in the quench velocity-based approach is $t_\text{step range}=[0.1, 1]~\text{ms}$, as deduced in Section~\ref{section:quench_velocity_benchmarking_no_insulation_quench_velocity}. The geometric assumptions, the initial conditions, and the analysis settings are identical with the ones discussed in Section \ref{section: 1D_quench_propagation_with_insulation}. The parameters related to the co-simulation are shown in Table \ref{table:1d_qv_benchmarking_geometry__with_insulation_parameters_quench_velocity}. 

\begin{table}[H]
    \caption{Input parameters in the quench velocity-based approach.} 
    \vspace{-1.em} 
    \fontsize{10}{10}
    \selectfont 
    \renewcommand{\arraystretch}{1.5}
    \begin{center}
        \begin{tabular}{ ccc }  
        \hline
        parameter & value & unit \\
        \hline
        $t_\text{com}$ & 2.5 & [ms] \\
        $v_\text{quench, average}$ & 2.18 & [m/s] \\
        $t_\text{total}$ & 0.5 & [s] \\
        \hline 
        \end{tabular}
    \end{center}  
    \label{table:1d_qv_benchmarking_geometry__with_insulation_parameters_quench_velocity} 
\end{table}

Fig.~\ref{fig:q_vel_benchmarking_temp_distr_over_strand_with_insulation} compares the temperature distribution along the strand with the mesh size of 20~mm between the standard analysis and the quench velocity-based approach. The quench front remains behind the~reference solution analogously to the case of a bare strand, as presented in Fig.~\ref{fig: q_vel_benchmarking_temp_distr_over_strand_no_insulation}.

\begin{figure}[H]
\centering
    \begin{tikzpicture}
        \begin{axis}[
          no markers,
          width=0.7\linewidth, 
          height = 5.0cm,
          xlabel={$\bar{x},~\text{m}$},
          ylabel={$T,~\text{K}$},
          xmin=0.0,
          ymin=0.0,
          xmax=1.5,
          legend pos=outer north east
          ]
          
        %   Initial temperature curve for the mesh used in quench velocity modelling 
          \addplot[black] table[x=position,y=t_0,col sep=comma] {sections/q_vel_modelling_benchmarking/figures/results_with_insulation/quench_velocity_50_nodes.csv};
          
        %   Heat Balance Equation plots
          \addplot[red] table[x=position,y=t_0_1,col sep=comma] {sections/q_vel_modelling_benchmarking/figures/results_with_insulation/heat_balance_1000_nodes_benchmark.csv};
          \addplot[red] table[x=position,y=t_0_3,col sep=comma] {sections/q_vel_modelling_benchmarking/figures/results_with_insulation/heat_balance_1000_nodes_benchmark.csv};
          \addplot[red] table[x=position,y=t_0_5,col sep=comma] {sections/q_vel_modelling_benchmarking/figures/results_with_insulation/heat_balance_1000_nodes_benchmark.csv};

        %   Quench Velocity Modelling plots
          \addplot[blue] table[x=position,y=t_0_1,col sep=comma] {sections/q_vel_modelling_benchmarking/figures/results_with_insulation/quench_velocity_50_nodes.csv};
          \addplot[blue] table[x=position,y=t_0_3,col sep=comma] {sections/q_vel_modelling_benchmarking/figures/results_with_insulation/quench_velocity_50_nodes.csv};
          \addplot[blue] table[x=position,y=t_0_5,col sep=comma] {sections/q_vel_modelling_benchmarking/figures/results_with_insulation/quench_velocity_50_nodes.csv};
          
          \legend{
          $T_\text{init}$,
          standard,,,,
          quench velocity
          }
        \end{axis}
        
        \draw[black, thick, ->] (1.5,3) -- (2.5,3);
        \node[scale = 1] at (3.3, 3) {$\vec{v}_\text{quench}$}; 
        
    \end{tikzpicture}
    \caption{Temperature distribution with a specified direction of quench velocity, $\vec{v}_\text{quench}$ of for the reference solution and the quench velocity-based approach with the longitudinal mesh size of 20~mm and the mesh size of 8~$\upmu \text{m}$ (20 nodes) across the insulation layer at three time steps: $t=\{0.1, 0.3, 0.5\}$~s.}
    \label{fig:q_vel_benchmarking_temp_distr_over_strand_with_insulation}
\end{figure}

As presented in Fig.~\ref{fig: q_vel_modelling_res_volt_benchmarking_with_insulation}, the resistive voltage rises monotonically in the quench velocity-based approach and follows the reference solution similarly to the example of a bare strand. 

\begin{figure}[H]
\centering
    \begin{tikzpicture}
        \begin{axis}[
          width=0.7\linewidth, 
          height = 4.0cm,
          xlabel={$t$, $\text{s}$},
          ylabel={$V_\text{res}$, $\text{V}$},
          xticklabel style={/pgf/number format/fixed},
          yticklabel style={/pgf/number format/fixed},
          xmin=0.0,
          xmax=0.5,
          legend pos =outer north east
          ]
          \addplot[red] table[x=time,y=heat_balance_benchmark,col sep=comma] {sections/q_vel_modelling_benchmarking/figures/results_with_insulation/quench_velocity_res_volt_benchmarking.csv};
          
          \addplot[blue, mark=*] table[x=time,y=50_nodes_quench_velocity,col sep=comma] {sections/q_vel_modelling_benchmarking/figures/results_with_insulation/quench_velocity_res_volt_benchmarking.csv};

          \legend{
          standard,
          quench velocity,
          }
          
        \end{axis}
    \end{tikzpicture}
    \caption{Comparison of the resistive voltage between the reference solution and the quench velocity-based approach with the longitudinal mesh size of 20~mm.}
    \label{fig: q_vel_modelling_res_volt_benchmarking_with_insulation}
\end{figure}

A double decrease in the mesh size from 20 to 10~mm results in a better convergence of the resistive voltage in the quench velocity-based approach with respect to the reference solution, as presented in Fig.~\ref{fig: q_vel_modelling_res_volt_rel_error_with_insulation}. For the mesh size of 10~mm, the relative error remains in the range of -10\% and converges to -2.5\% as the quench propagates. With a coarser mesh of 20~mm, the relative error converges to the approximate value of -5\%. It is also worth mentioning that the decrease of the mesh size to 1~mm of does not decrease the relative error with respect to the solution with a larger mesh size. The mesh size of 1~mm corresponds to the case of a reference standard solution. Similarly to the example without the insulation layer, one has to accept the difference between two approaches.

\begin{figure}[H]
\centering
    \begin{tikzpicture}
        \begin{axis}[
          width=0.7\linewidth, 
          height = 4.0cm,
          xlabel={$t$, $\text{s}$},
          ylabel={$E_\text{r}$, \%},
          xticklabel style={/pgf/number format/fixed},
          xmin=0.0,
          xmax=0.5,
          legend pos=outer north east
          ]
          \addplot[blue, mark=*] table[x=time,y=50_nodes,col sep=comma] {sections/q_vel_modelling_benchmarking/figures/results_with_insulation/quench_velocity_res_volt_rel_error.csv};
          \addplot[red, mark=*] table[x=time,y=100_nodes,col sep=comma] {sections/q_vel_modelling_benchmarking/figures/results_with_insulation/quench_velocity_res_volt_rel_error.csv};
          \addplot[green, mark=*] table[x=time,y=1000_nodes,col sep=comma] {sections/q_vel_modelling_benchmarking/figures/results_with_insulation/quench_velocity_res_volt_rel_error.csv};
          
          \addlegendimage{/pgfplots/refstyle=plot_resistive_voltage}\addlegendentry{20 mm}
          \addlegendimage{/pgfplots/refstyle=plot_resistive_voltage}\addlegendentry{10 mm}
          \addlegendimage{/pgfplots/refstyle=plot_resistive_voltage}\addlegendentry{1 mm}

        \end{axis}
    \end{tikzpicture}
    \caption{Relative error of the resistive voltage for the varying mesh size along the strand in the quench velocity-based approach.}
    \label{fig: q_vel_modelling_res_volt_rel_error_with_insulation}
\end{figure}

Figure~\ref{fig: q_vel_modelling_hot_spot_rel_error_with_insulation} presents the evolution of the relative error with respect to the hot-spot temperature. One can notice the stabilisation of an error after approximately $t=0.2~\text{s}$. After that moment, the error starts diverging with an approximate rate of -2\%/s by assuming a linear change error drop with time. The divergence occurs for every mesh size. At $t=0.5~\text{s}$, the result with the mesh size of 20~mm is characterised by a relative error 0.5\% lower with respect to higher mesh densities. 

\begin{figure}[H]
\centering
    \begin{tikzpicture}
        \begin{axis}[
          width=0.7\linewidth, 
          height = 4.0cm,
          xlabel={$t$, $\text{s}$},
          ylabel={$E_\text{r}$, \%},
          xticklabel style={/pgf/number format/fixed},
          xmin=0.0,
          xmax=0.5,
          legend pos=outer north east
          ]
          \addplot[blue, mark=*] table[x=time,y=50_nodes,col sep=comma] {sections/q_vel_modelling_benchmarking/figures/results_with_insulation/quench_velocity_hot_spot_rel_error.csv};
          \addplot[red, mark=*] table[x=time,y=100_nodes,col sep=comma] {sections/q_vel_modelling_benchmarking/figures/results_with_insulation/quench_velocity_hot_spot_rel_error.csv};
          \addplot[green, mark=*] table[x=time,y=1000_nodes,col sep=comma] {sections/q_vel_modelling_benchmarking/figures/results_with_insulation/quench_velocity_hot_spot_rel_error.csv};
          \addlegendimage{/pgfplots/refstyle=plot_resistive_voltage}\addlegendentry{20 mm}
          \addlegendimage{/pgfplots/refstyle=plot_resistive_voltage}\addlegendentry{10 mm}
          \addlegendimage{/pgfplots/refstyle=plot_resistive_voltage}\addlegendentry{1 mm}

        \end{axis}
    \end{tikzpicture}
    \caption{Relative error of the hot-spot temperature for the varying mesh size along the strand used for the quench velocity-based approach.}
    \label{fig: q_vel_modelling_hot_spot_rel_error_with_insulation}
\end{figure}


