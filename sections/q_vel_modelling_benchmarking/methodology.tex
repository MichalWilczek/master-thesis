The benchmark study is conducted for two cases separately: 
\begin{enumerate}
    \item 1D strand analysis without insulation layer,
    \item 1D strand analysis with an external insulation layer.
\end{enumerate}

In order to conduct a quench velocity-based analysis, the value of quench velocity in time is required. The standard analysis serves for two purposes:
\begin{enumerate}
    \item average quench velocity value to be used in the quench velocity-based model,
    \item reference model for a relative error.
\end{enumerate}

The benchmarking workflow, which compares the standard solution with a quench velocity-based method when no insulation is considered, is illustrated in Fig.~\ref{fig:block_diagram_benchmarking_methodology_no_insulation}. As it was recommended in \cite[p.~40]{paudel_thesis}, the mesh should be discretised in the scale of a millimetre when an adiabatic quench thermal problem is considered. Therefore, the input mesh size of $x=1~\text{mm}$ was set as the input parameter with a relatively large time step range of $t= [100, 1000]~\upmu \text{s}$. The workflow is divided into three main parts: 
\begin{enumerate}
    \item find the time step range,
    \item find the mesh size, 
    \item test quench velocity-based method with varying input parameters.
\end{enumerate}

The first two parts of the workflow are an iteration loop with a condition statement of a relative error with respect to an average quench velocity. The relative error range is calculated as 
\begin{equation}
    E_\text{r} = \frac{r_\text{i}-r_{i-1}}{r_\text{i}}~100\%,
\end{equation}
where $r_\text{i}$ -- reference analysis, $r_{i-1}$ -- analysis from the preceding iteration loop~$i$. In all the analyses, the condition statement imposes the absolute value of a relative error less or equal to 1\%. In this chapter, the quench velocity is calculated in a different manner with respect to Chapter \ref{chapter: 1d_quench_propagation_modelling} where the method was imposed by the STEAM-BBQ tool. The average quench velocity is calculated as
\begin{equation}
    v_\text{quench} = \frac{ \sum_{j=1}^{k-1} \frac{x_{j+1}-x_j}{\Delta t} }{k},
\end{equation}
where $x_j$ -- quench front position at time window~$j$, $\Delta t$ -- time increment between two time windows $j$ and $j+1$, $k$ -- number of time windows $j$. The time increment for the standard analyses was assumed to be equal to $\Delta t=10~\text{ms}$. In addition, all material properties are based on fits provided by NIST described in Appendix~\ref{appendix_material_properties_description}.

\begin{figure}[H]
    \centering
    \renewcommand{\baselinestretch}{0.75} 
    \begin{tikzpicture}[node distance = 1.5cm, auto]
        \tikzstyle{decision} = [diamond, draw, fill=blue!20, text width=3cm, text badly centered, node distance=3.0cm, inner sep=0pt, scale=0.8]
        \tikzstyle{block} = [rectangle, draw, fill=blue!20, text width=7.0cm, text centered, rounded corners, minimum height=0.5cm, node distance=1.15cm, scale=0.8]
        \tikzstyle{line} = [draw, -latex']
        \tikzstyle{cloud} = [draw, ellipse,fill=red!20, node distance=7cm, minimum height=2em, scale=0.8]
    
    \node [block] (initialisation) {Let initial simulation parameters be: \\ iteration $i=1$ \\ time stepping parameter $n=1$ \\ mesh size $x=1~\text{mm}$ \\ time step range $t \in [100, 1000] / 10^n ~\upmu \text{s}$};
    \node [block, below of=initialisation, node distance = 1.75cm] (solution 1) {Conduct analysis number $i$};
    \node [block, below of=solution 1] (increment 1) {$i=i+1$ \\ $n=n+1$};
    \node [block, below of=increment 1] (solution 2) {Conduct analysis number $i$};
    \node [block, below of=solution 2] (reference 1) {Set analysis number $i$ as a reference result};
    \node [decision, below of=reference 1] (decide 1) {check if $|E_\text{r}| \leq 1\%$ \\ for analysis $i-1$};
    \node [cloud, right of=decide 1] (explanation 1) {Time stepping loop};
    \node [block, below of=decide 1, node distance=2.75cm] (result 1) {Use analysis $i-1$ as a reference analysis~$j$};

    \path [line] (initialisation) -- (solution 1);
    \path [line] (solution 1) -- (increment 1);
    \path [line] (increment 1) -- (solution 2);
    \path [line] (solution 2) -- (reference 1);
    \path [line] (reference 1) -- (decide 1);
    \path [line] (decide 1) -| node [near start, scale=0.8] {no} (-5,-5) |- (increment 1);
    \path [line, dashed] (explanation 1) -- (decide 1);

    \node [block, below of=result 1] (increment 2) {$j=j+1$ \\ $x=x/2$};
    \node [block, below of=increment 2] (solution 3) {Conduct analysis number $j$};
    \node [block, below of=solution 3] (reference 2) {Set analysis number $j$ as a reference result};
    \node [decision, below of=reference 2] (decide 2) {check if $|E_\text{r}| \leq 1\%$ \\ for analysis $j-1$};
    \node [cloud, right of=decide 2] (explanation 2) {Mesh size loop};
    \node [block, below of=decide 2, node distance=2.95cm] (quench velocity) {Use average $v_\text{quench}$ based on solution $j-1$};
    \node [block, below of=quench velocity] (quench velocity model) {Conduct quench velocity-based analyses};
    \node [cloud, right of=quench velocity model] (explanation 3) {Benchmarking loop};
    
    \path [line] (decide 1) -- node [near start, scale=0.8] {yes} (result 1);
    \path [line] (result 1) -- (increment 2);
    \path [line] (increment 2) -- (solution 3);
    \path [line] (solution 3) -- (reference 2);
    \path [line] (reference 2) -- (decide 2);
    \path [line] (decide 2) -| node [near start, scale=0.8] {no} (-5,-10) |- (increment 2);
    \path [line] (decide 2) -- node [near start, scale=0.8] {yes} (quench velocity);
    \path [line] (quench velocity) -- (quench velocity model);
    \path [line, dashed] (explanation 2) -- (decide 2);
    \path [line, dashed] (explanation 3) -- (quench velocity model);
    \end{tikzpicture}
    \caption{Block diagram description of a benchmarking method without the insulation layer.}
    \label{fig:block_diagram_benchmarking_methodology_no_insulation}
\end{figure}

The next step of the benchmarking is the comparison conducted with an external insulation layer. The method workflow is presented in Fig.~\ref{fig:block_diagram_benchmarking_methodology_with_insulation}. In this case, the longitudinal mesh size as well as a time step range are taken from a reference model without the insulation layer. In the standard analysis, one searches for a mesh size across the insulation layer which allows for obtaining the solution within the precision of a relative error of 1\% with respect to the average quench velocity. When the condition statement of a relative error is fulfilled, the average quench velocity is taken from a reference model and set as an input parameter for the quench velocity-based analyses. 

\begin{figure}[H]
    \centering
    \begin{tikzpicture}[node distance = 1.5cm, auto]
        \renewcommand{\baselinestretch}{0.75} 
         \tikzstyle{decision} = [diamond, draw, fill=blue!20, text width=3cm, text badly centered, node distance=3.0cm, inner sep=0pt, scale=0.8]
        \tikzstyle{block} = [rectangle, draw, fill=blue!20, text width=7.0cm, text centered, rounded corners, minimum height=0.5cm, node distance=1.15cm, scale=0.8]
        \tikzstyle{line} = [draw, -latex']
        \tikzstyle{cloud} = [draw, ellipse,fill=red!20, node distance=7cm, minimum height=2em, scale=0.8]
    
    \node [block] (initialisation) {Let initial simulation parameters be: \\ iteration $i=1$ \\ number of nodes across the insulation $a=3$};
    \node [block, below of=initialisation, node distance = 1.75cm] (solution 1) {Conduct analysis number $i$};
    \node [block, right of=solution 1, text width=5.5cm, node distance=7cm] (initialisation2) {Reference model input from the case without the insulation layer: \\ mesh size~$x$, time step range~$t$};

    \node [block, below of=solution 1] (increment 1) {$i=i+1$ \\ $a=a+b$, where b is chosen by a user};
    \node [block, below of=increment 1] (solution 2) {Conduct analysis number $i$};
    \node [block, below of=solution 2] (reference 1) {Set analysis number $i$ as a reference result};
    \node [decision, below of=reference 1] (decide 1) {check if $|E_\text{r}| \leq 1\%$ \\ for analysis $i-1$};
    \node [cloud, right of=decide 1] (explanation 1) {Insulation mesh size loop};

    \path [line] (initialisation) -- (solution 1);
    \path [line] (initialisation2) -- (solution 1);
    \path [line] (solution 1) -- (increment 1);
    \path [line] (increment 1) -- (solution 2);
    \path [line] (solution 2) -- (reference 1);
    \path [line] (reference 1) -- (decide 1);
    \path [line] (decide 1) -| node [near start, scale=0.8] {no} (-5,-5) |- (increment 1);
    \path [line, dashed] (explanation 1) -- (decide 1);

    \node [block, below of=decide 1, node distance=2.95cm] (quench velocity) {Use average $v_\text{quench}$ based on solution $i-1$};
    \node [block, below of=quench velocity] (quench velocity model) {Conduct quench velocity-based analyses};
    \node [cloud, right of=quench velocity model] (explanation 3) {Benchmarking loop};

    \path [line] (decide 1) -- node [near start, scale=0.8] {yes} (quench velocity);
    \path [line] (quench velocity) -- (quench velocity model);
    \path [line, dashed] (explanation 3) -- (quench velocity model);
    \end{tikzpicture}
    \caption{Block diagram description of a benchmarking method with the insulation layer.}
    \label{fig:block_diagram_benchmarking_methodology_with_insulation}
\end{figure}