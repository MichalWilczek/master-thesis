
In this study, the standard ANSYS simulation of a bare strand is performed based on the LINK33 element. The geometric assumptions, the initial conditions, and the analysis settings related to ANSYS are the same as in Section~\ref{section: 1D_quench_propagation_no_insulation}. One can observe, that the initial Gaussian temperature profile (see Fig. \ref{fig: init_gauss_temp_distr}) is more approximated, if the mesh along the strand is relaxed. Therefore, the initial energy deposition in the strand depends on the mesh size as well. The initial energy in the discretised domain is calculated as
\begin{equation}
    E_\text{initial} = \sum_{i=1}^{n-1} V_{i,i+1}~c_\text{v, strand}\left(T = \frac{T_i+T_{i+1}}{2}\right)~\frac{T_i+T_{i+1}}{2},
\end{equation}
where $n$ -- number of nodes in the initially quenched zone, $V_{i,~i+1}$ -- volume of the domain between two neighbouring nodes, $c_\text{v, strand}$ -- volumetric heat capacity calculated for an average temperature, $T_i$ -- temperature of a node $i$, $T_{i+1}$ -- temperature of a neighbouring node $i+1$.

The relation between the initially deposited energy in the strand and the number of nodes over a 1 metre-long domain is presented in Fig. \ref{fig: q_vel_modelling_energy_deposition}. The exact energy deposition starts converging for approximately 1000 nodes which corresponds to the mesh size of 1~mm. A lower number of nodes would slow down the quench front propagation.

\begin{figure}[H]
    \centering
    \begin{tikzpicture}
        \begin{axis}[
          width=0.7\linewidth, 
          height = 4.0cm,
          xmode=log,
          xlabel={Number of nodes},
          ylabel={$E_\text{initial}$, $\text{J}$},
          xmin=50.0,
          xmax=5000.0
          ]
          \addplot[blue, mark=*] table[x=nodes,y=energy,col sep=comma] {sections/q_vel_modelling_benchmarking/figures/results_no_insulation/energy_deposition.csv};
        \end{axis}
    \end{tikzpicture}
    \caption{Initial energy deposition along the 1~metre-long strand as a function of number of nodes.}
    \label{fig: q_vel_modelling_energy_deposition}
\end{figure}

As illustrated in the block diagram of the analysis workflow corresponding to the bare strand in Fig.~\ref{fig:block_diagram_benchmarking_methodology_no_insulation}, the initial simulation is performed with the longitudinal mesh size of 1~mm and the time step range $t_\text{step range}=[0.1, 1]~\text{ms}$. The time step loop consists of two iterations, as shown in Table~\ref{table: 1d_qv_benchmarking_results_heat_balance_no_insulation}. The time step range $t_\text{step range}=[0.01, 0.1]~\text{ms}$ fulfills the condition statement related to the average quench velocity. In the next step, the mesh size loop consists of a single iteration in which the quality of the initial mesh size of 1~mm is confirmed by comparing the result with the analysis in which the number of nodes is doubled, as shown in Fig.~\ref{fig: q_vel_modelling_v_quench_rel_error_no_insulation}. 

\begin{table}[H]
    \caption{Quench results for analysed time step ranges.} 
    \vspace{-1.em} 
    \fontsize{10}{10}
    \selectfont 
    \renewcommand{\arraystretch}{1.5}
    \begin{center}
        \begin{tabular}{ cccc }  
        \hline
        $t_\text{step range}$, ms & [0.001, 0.01] & [0.01, 0.1] & [0.1, 1] \\
        \hline
        $v_\text{quench, average}$, m/s & 6.80 & 6.81 & 7.1 \\
        $E_\text{r}$ & - & 0.17\% & 4.26\% \\
        \hline 
        \end{tabular}
    \end{center}  
     \label{table: 1d_qv_benchmarking_results_heat_balance_no_insulation} 
 \end{table}

\begin{figure}[H]
\centering
    \begin{tikzpicture}
        \begin{axis}[
          width=0.7\linewidth, 
          height = 4.0cm,
          xlabel={$t$, $\text{s}$},
          ylabel={$E_\text{r}$, \%},
          xtick={0,0.02,0.04,...,0.1},
          xticklabel style={/pgf/number format/fixed},
          xmin=0.0,
          xmax=0.1
          ]
          \addplot[blue, mark=*] table[x=time,y=1000_nodes_rel_error,col sep=comma] {sections/q_vel_modelling_benchmarking/figures/results_no_insulation/v_quench_rel_error.csv};
          \addplot[blue, dashed] table[x=time,y=1000_nodes_av_rel_error,col sep=comma] {sections/q_vel_modelling_benchmarking/figures/results_no_insulation/v_quench_rel_error.csv};
        \end{axis}
    \end{tikzpicture}
    \caption{Relative error of the incremental and average (dashed line) quench velocity for the mesh size of 1~mm with respect to the mesh size of 0.5~mm; analysis conducted with the time step range of $t_\text{step range}=[0.01, 0.1]~\text{ms}$.}
    \label{fig: q_vel_modelling_v_quench_rel_error_no_insulation}
\end{figure}

In conclusion, the reference used for the verification loop in Fig.~\ref{fig:block_diagram_benchmarking_methodology_no_insulation} is the ANSYS solution with a mesh size of 1~mm and a time step range of $t_\text{step range}=[0.01, 0.1]~\text{ms}$. The parameters are summarised in Table~\ref{table: 1d_qv_benchmarking_reference_analysis_settings_no_insulation}. 

\begin{table}[H]
    \caption{Input parameters of the reference standard solution.} 
    \vspace{-1.em} 
    \fontsize{10}{10}
    \selectfont 
    \renewcommand{\arraystretch}{1.5}
    \begin{center}
        \begin{tabular}{ ccc }  
        \hline
        parameter & value & unit \\
        \hline
        mesh size & 1 & [mm] \\
        $t_\text{step range}$ & [0.01, 0.1] & [ms] \\
        $v_\text{quench, average}$ & 6.81 & [m/s] \\
        \hline 
        \end{tabular}
    \end{center}  
     \label{table: 1d_qv_benchmarking_reference_analysis_settings_no_insulation} 
 \end{table}

The reference solution is an acceptable compromise between accuracy and computing time. As presented in Fig. \ref{fig: q_vel_modelling_heat_balance_computing_time_no_insulation}, in a 1D quench propagation, computing time rises monotonically with the mesh refinement while keeping the time step range equal to $t_\text{step range}=[0.01, 0.1]~\text{ms}$\footnote{The analysis was performed on the following calculation unit: Intel(R) Xeon(R) CPU E5-2667 V4 @~3.20 GHz (2~processors) with RAM 128 GB.}.

\begin{figure}[H]
\centering
    \begin{tikzpicture}
        \begin{axis}[
          width=0.7\linewidth, 
          height = 4.0cm,
          xlabel={Number of nodes},
          ylabel={$t$, $\text{s}$},
          xmin=0,
          xtick={0,1000,2000,...,5000},
          xticklabel style={/pgf/number format/fixed},
          xmax=5000,
          legend pos=outer north east
          ]
          \addplot[blue, mark=*] table[x=nodes,y=time,col sep=comma] {sections/q_vel_modelling_benchmarking/figures/results_no_insulation/heat_balance_computing_time.csv};
          \addplot[red, dashed] table[x=nodes,y=linear_approx,col sep=comma] {sections/q_vel_modelling_benchmarking/figures/results_no_insulation/heat_balance_computing_time.csv};
          
          \legend{
          computing time,
          linear approximation
          }
          
        \end{axis}
    \end{tikzpicture}
    \caption{Computing time as a function of number of nodes.}
    \label{fig: q_vel_modelling_heat_balance_computing_time_no_insulation}
\end{figure}