
In this study, the standard thermal numerical analysis based on LINK33 element was conducted in ANSYS. The geometric assumptions as well as initial conditions were the same as in Section \ref{subsection: 1D_quench_propagation_no_insulation}. 

It is interesting to mention that the initial Gaussian profile of temperature over the domain (see Fig. \ref{fig: init_gauss_temp_distr}) stores a different value of energy in the strand depending on the mesh size. The initial energy in the discretised strand domain is calculated as
\begin{equation}
    E_\text{initial} = \sum_{i=1}^{n-1} V_{i,~i+1}~C_\text{v, strand}(\frac{T_i+T_{i+1}}{2})~\frac{T_i+T_{i+1}}{2},
\end{equation}
where $n$ -- number of nodes in the initially quenched zone, $V_{i,~i+1}$ -- volume of the domain between two neighbouring nodes, $C_\text{v, strand}$ -- volumetric heat capacity over an average temperature, $T_i$ -- temperature of node $i$, $T_{i+1}$ -- temperature of the neighbouring node $i+1$.

The relation between the initially deposited energy in the strand and the number of nodes over a 1 metre-long domain is presented in Fig. \ref{fig: q_vel_modelling_energy_deposition}. The exact energy deposition starts converging for the mesh refinement in the range of 500 and 1000 nodes. The lower number of nodes would cause the quench front being too slow with respect to the result obtained from a denser mesh.

\begin{figure}[H]
\centering
    \begin{tikzpicture}
        \begin{axis}[
          width=0.7\linewidth, 
          height = 4.5cm,
          xmode=log,
          xlabel={Number of nodes},
          ylabel={Deposited Energy, $\text{J}$},
          xmin=50.0,
          xmax=5000.0
          ]
          \addplot[blue, mark=*] table[x=nodes,y=energy,col sep=comma] {sections/q_vel_modelling_benchmarking/figures/results_no_insulation/energy_deposition.csv};
        \end{axis}
    \end{tikzpicture}
    \caption{Initial energy deposition along the strand as a function of number of nodes in a 1 metre-long domain.}
    \label{fig: q_vel_modelling_energy_deposition}
\end{figure}

The first analyses were conducted with the mesh size of 1 mm (1000 nodes), as it was recommended in Section~\ref{subsection: 1D_quench_propagation_conclusions}. Three different time step ranges, $t=\{[1, 10], [10, 100], [100, 1000]\}~\upmu \text{s}$ were tested to verify the maximum time step at which the analysis remained within an acceptable error. The results for quench velocity for the first two remained in the acceptable relative error range of~0.2\%. The largest time step range increased the quench front velocity by 5\%. The middle time step range of $t=[10, 100]~\upmu \text{s}$ is further analysed and compared for a different mesh size as it gives an error within specified tolerance. Moreover, it is a favourable choice for the sake of decreasing the computing needs for the analyses with respect to the time step range of $t= [1, 10]~\upmu \text{s}$. 

The study over a 1 metre-long domain was conducted with a varying number of nodes, $n=\{50, 100, 500, 1000, 2000, 5000\}$. The results from the analysis with 5000 nodes are used as a benchmark with a time step of $t=[10, 100]~\upmu \text{s}$. With the mesh size of 1000 nodes, as presented in Fig. \ref{fig: q_vel_modelling_v_quench_rel_error_no_insulation}, the relative error for an incremental quench velocity value during the analysis equals less than 1\%. The average quench velocity is equal to $v_\text{quench}=6.81~\frac{\text{m}}{\text{s}}$.

\begin{figure}[H]
\centering
    \begin{tikzpicture}
        \begin{axis}[
          width=0.7\linewidth, 
          height = 4.5cm,
          xlabel={Time, $\text{s}$},
          ylabel={Relative error, \%},
          xtick={0,0.02,0.04,...,0.1},
          xticklabel style={/pgf/number format/fixed},
          xmin=0.0,
          xmax=0.1
          ]
          \addplot[blue, mark=*] table[x=time,y=500_nodes,col sep=comma] {sections/q_vel_modelling_benchmarking/figures/results_no_insulation/v_quench_rel_error.csv};
          \addplot[red, mark=*] table[x=time,y=1000_nodes,col sep=comma] {sections/q_vel_modelling_benchmarking/figures/results_no_insulation/v_quench_rel_error.csv};
          \addlegendimage{/pgfplots/refstyle=plot_resistive_voltage}\addlegendentry{500 nodes}
          \addlegendimage{/pgfplots/refstyle=plot_resistive_voltage}\addlegendentry{1000 nodes}
        \end{axis}
    \end{tikzpicture}
    \caption{Quench velocity relative error for 500 and 1000 nodes.}
    \label{fig: q_vel_modelling_v_quench_rel_error_no_insulation}
\end{figure}

The analysis with 1000 nodes and a time step in the range of $t=[10, 100]~\upmu \text{s}$ is used as a further benchmark result for quench velocity simulation presented in the next section. It is an acceptable compromise between the accuracy and computing time. As presented in Fig. \ref{fig: q_vel_modelling_heat_balance_computing_time_no_insulation}, in 1D thermal quench propagation, computing time rises monotonically with mesh refinement if the insulation is not considered. 

\begin{figure}[H]
\centering
    \begin{tikzpicture}
        \begin{axis}[
          width=0.7\linewidth, 
          height = 4.5cm,
          xlabel={Number of nodes},
          ylabel={Computing time, $\text{s}$},
          xmin=0,
          xtick={0,1000,2000,...,5000},
          xticklabel style={/pgf/number format/fixed},
          xmax=5000,
          legend pos=north west
          ]
          \addplot[blue, mark=*] table[x=nodes,y=time,col sep=comma] {sections/q_vel_modelling_benchmarking/figures/results_no_insulation/heat_balance_computing_time.csv};
          \addplot[red, dashed] table[x=nodes,y=linear_approx,col sep=comma] {sections/q_vel_modelling_benchmarking/figures/results_no_insulation/heat_balance_computing_time.csv};
          
          \legend{
          computing time,
          linear approximation
          }
          
        \end{axis}
    \end{tikzpicture}
    \caption{Computing time as a function of number of nodes.}
    \label{fig: q_vel_modelling_heat_balance_computing_time_no_insulation}
\end{figure}