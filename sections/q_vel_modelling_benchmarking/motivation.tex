
As it is demonstrated in Chapter~\ref{chapter: 1d_quench_propagation_modelling}, in order to solve a time-dependent temperature distribution over a 1D strand, a refined mesh with a decreased time step is required. This section compares the standard ANSYS analysis of the quench propagation with the quench velocity-based approach presented in Chapter~\ref{chapter:quench_velocity_modelling}. The aim of this benchmark is to verify what is the computing time gain and a possible longitudinal mesh size relaxation with the quench velocity-based approach. Moreover, the relation between the mesh relaxation and the evolution of the error associated with the peak temperature is studied. The numerical analyses using the quench velocity-based approach are co-simulated with the external routine written in Python, as described in Chapter~\ref{chapter:python_implementation} with algorithms implemented as outlined in Chapter~\ref{chapter:algorithms}. The benchmark criteria for the standard analysis and the quench velocity-based approach are: 
\begin{enumerate}
    \item mesh size,
    \item time step,
    \item computation time,
    \item relative error with respect to the hot-spot temperature and resistive voltage.
\end{enumerate}