
As it was demonstrated in Chapter~\ref{chapter: 1d_quench_propagation_modelling}, in order to solve a time-dependent temperature distribution over a 1D strand, a refined mesh with a decreased time step is required. This section compares the standard solution explicitly solving the heat balance equation with a quench velocity-based approach presented in Chapter~\ref{chapter:quench_velocity_modelling}. The aim of this benchmark is to verify what is the computing time gain and a possible longitudinal mesh size relaxation when a quench velocity-based approach is applied. Moreover, the precision of the quench velocity-based approach is estimated. 

The benchmark criteria for the standard heat balance equation-based analysis and the quench velocity-based one are: 
\begin{enumerate}
    \item number of nodes,
    \item time step applied,
    \item computing time,
    \item relative error with respect to the hot spot temperature and resistive voltage.
\end{enumerate}

The analyses using the quench velocity-based approach are co-simulated with an external routine written in Python language, as described in Chapter~\ref{chapter:python_implementation} with algorithms implemented as outlined in Chapter~\ref{chapter:algorithms}.

% proven replace by... demonstrated, discussed, outlined
