
As it is demonstrated in Chapter~\ref{chapter: 1d_quench_propagation_modelling}, a refined mesh with a reduced time step is required in order to solve a time-dependent temperature distribution over a 1D strand. This section compares the standard ANSYS analysis of the quench propagation with the quench velocity-based approach presented in Chapter~\ref{chapter:quench_velocity_modelling}. The aim of this benchmark is to verify what the computing time gain is, and a possible longitudinal mesh size relaxation with the quench velocity-based approach. Moreover, the relation between the mesh relaxation, and the evolution of the error associated with the peak temperature is studied. The numerical analyses using the quench velocity-based approach are co-simulated with the external routine written in Python described in Chapter~\ref{chapter:python_implementation}, with algorithms implemented as outlined in Chapter~\ref{chapter:algorithms}. The benchmark criteria for the standard analysis and the quench velocity-based approach are: 
\begin{enumerate}
    \item Mesh size
    \item Time step
    \item Computation time
    \item Relative error with respect to the hot-spot temperature and resistive voltage
\end{enumerate}