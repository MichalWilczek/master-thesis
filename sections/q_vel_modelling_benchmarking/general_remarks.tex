The quench velocity-based analysis is a promising tool when the analysis of large thermal domains is simulated. If this method is used, a certain error should be assumed with respect to the evolving resistive voltage and hot spot temperature. It was demonstrated that the co-simulation results in underestimation of the quench from position with respect to the reference solution. As it was presented in this chapter, the total error can be estimated as a function of a mesh size. The mesh size across the insulation layer was chosen to obtain a solution in the assumed tolerance range of an average quench velocity equal to 1\%.

While simulating a multi-strand case with the quench velocity-based method, one should remember that the result will be less precise with respect to a standard solution. An estimation of an error for the case without an external insulation layer is presented in Table~\ref{table: 1d_qv_benchmarking_tolerance_range_without_insulation}. The tolerances are based on relative errors at $t=0.01~\text{s}$ of corresponding parameters as well as the final relative errors to which the parameters converged to.

 \begin{table}[H]
    \caption{Tolerance range for a quench velocity-based analysis without insulation.} 
    \vspace{-1.em} 
    \fontsize{10}{10}
    \selectfont 
    \renewcommand{\arraystretch}{1.5}
    \begin{center}
        \begin{tabular}{ cc | c | cc }  
        
        \hline
        \multicolumn{2}{c|}{mesh size} & \multirow{2}{*}{time step, \textmu s} & \multicolumn{2}{|c}{relative error tolerance} \\
        
        strand, m & insulation, \textmu m &  & $V_\text{res}$ & $T_\text{hot spot}$ \\
        \hline
        10 & 10 & [100, 1000] & -15\% & $>-5\%$ \\
        20 & 10 & [100, 1000] & -20\% & $>-5\%$ \\
        \hline 
        \end{tabular}
    \end{center}  
     \label{table: 1d_qv_benchmarking_tolerance_range_without_insulation} 
 \end{table}
 
The benchmark results with insulation are shown in Table~\ref{table: 1d_qv_benchmarking_tolerance_range_with_insulation}. Since the relative error of a hot spot temperature did not converge, its tolerance was assumed to be three times higher than its maximum value at $t=0.3~\text{s}$.

 \begin{table}[H]
    \caption{Tolerance range for a quench velocity-based analysis with insulation.} 
    \vspace{-1.em} 
    \fontsize{10}{10}
    \selectfont 
    \renewcommand{\arraystretch}{1.5}
    \begin{center}
        \begin{tabular}{ cc | c | cc }  
        
        \hline
        \multicolumn{2}{c|}{mesh size} & \multirow{2}{*}{time step, \textmu s} & \multicolumn{2}{|c}{relative error tolerance} \\
        
        strand, m & insulation, \textmu m &  & $V_\text{res}$ & $T_\text{hot spot}$ \\
        \hline
        10 & 10 & [100, 1000] & -5\% & -5\% \\
        20 & 10 & [100, 1000] & -20\% & -10\% \\
        \hline 
        \end{tabular}
    \end{center}  
     \label{table: 1d_qv_benchmarking_tolerance_range_with_insulation} 
 \end{table}
 
