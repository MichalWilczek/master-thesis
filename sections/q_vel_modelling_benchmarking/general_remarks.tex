
The quench velocity-based approach is a promising tool when a simulation of a large thermal domain is conducted. If this method is used, a certain error should be assumed with respect to an evolving resistive voltage and a hot spot temperature. In the benchmarking process presented in this chapter, the analysis settings for a standard solution were chosen to remain in an assumed absolute relative error of 1\% with respect to an average quench velocity. In the comparison of two methodologies, one can conclude that a quench velocity-based co-simulation results in an underestimation of the quench front position with respect to a standard solution. 

While simulating a multi-strand case with the quench velocity-based approach, one should remember that the results will be less precise with respect to a standard solution. A~relative error estimation for the case without an external insulation layer is presented in Table~\ref{table: 1d_qv_benchmarking_tolerance_range_without_insulation}. The tolerances are based on relative errors at $t=0.01~\text{s}$ of the corresponding parameters as well as the final relative errors to which the parameters converged to at $t=0.3~\text{s}$.

 \begin{table}[H]
    \caption{Tolerance range in a quench velocity-based approach without insulation.} 
    \vspace{-1.em} 
    \fontsize{10}{10}
    \selectfont 
    \renewcommand{\arraystretch}{1.5}
    \begin{center}
        \begin{tabular}{ cc | c | cc }  
        
        \hline
        \multicolumn{2}{c|}{mesh size} & \multirow{2}{*}{time step, \textmu s} & \multicolumn{2}{c}{relative error tolerance} \\
        
        strand, m & insulation, \textmu m &  & $V_\text{res}$ & $T_\text{hot spot}$ \\
        \hline
        10 & 10 & [100, 1000] & -15\% & $>-5\%$ \\
        20 & 10 & [100, 1000] & -20\% & $>-5\%$ \\
        \hline 
        \end{tabular}
    \end{center}  
     \label{table: 1d_qv_benchmarking_tolerance_range_without_insulation} 
 \end{table}
 
The results obtained from the benchmarking analysis with the insulation layer are shown in Table~\ref{table: 1d_qv_benchmarking_tolerance_range_with_insulation}. Similarly to the case without the insulation layer, the tolerances are based on relative errors at $t=0.01~\text{s}$ of the corresponding parameters as well as the final relative errors to which the parameters converged to at $t=0.5~\text{s}$. However, the relative error of a~hot spot temperature did not converge to one specific value. Thus, it is assumed that the error increases by 2\% per second during the simulation. It is a non-optimistic assumption which does not take into consideration a linearisation of the material properties at higher temperatures when their derivatives with respect to the temperature are more constant leading to a higher probability of convergence. 

 \begin{table}[H]
    \caption{Tolerance range in a quench velocity-based approach with insulation.} 
    \vspace{-1.em} 
    \fontsize{10}{10}
    \selectfont 
    \renewcommand{\arraystretch}{1.5}
    \begin{center}
        \begin{tabular}{ cc | c | cc }  
        
        \hline
        \multicolumn{2}{c|}{mesh size} & \multirow{2}{*}{time step, \textmu s} & \multicolumn{2}{c}{relative error tolerance} \\
        
        strand, m & insulation, \textmu m &  & $V_\text{res}$ & $T_\text{hot spot}$ \\
        \hline
        10 & 10 & [100, 1000] & -5\% & -1.5\% + (-2\%/s) \\
        20 & 10 & [100, 1000] & -20\% & -1.5\% + (-2\%/s) \\
        \hline 
        \end{tabular}
    \end{center}  
     \label{table: 1d_qv_benchmarking_tolerance_range_with_insulation} 
 \end{table}

Unfortunately, the lack of convergence for a hot spot temperature may have an influence on a~resistive voltage being a function of temperature due to the resistivity of copper. Therefore, two relative errors may be a coupled problem in a longer simulation lasting several seconds. The analysis with the insulation layer is definitely more ambiguous compared to the one without it. However, in the presented analysis the insulation has a large influence on the results because it occupies 56\% of the total volume of the strand. In most of the cases corresponding to the superconducting accelerator magnets, this fraction is much lower.

