
The quench velocity-based approach is a promising tool when a simulation of a large numerical domain is conducted exceeding 100 000 degrees of freedom for a standard solution. If this method is used, a certain error should be assumed with respect to the resistive voltage and the hot-spot temperature (both evolve in time). In the benchmarking process presented in this chapter, the analysis settings for a standard solution are chosen to remain in an assumed relative error of 1\% with respect to an average quench velocity. In the comparison of two methodologies, one can conclude that the quench velocity-based approach results in an underestimation of the quench front position with respect to the standard simulation performed in ANSYS. 

While simulating a multi-strand case with the quench velocity-based approach, one should remember that the results will be less precise with respect to a standard solution. An estimation of the~relative error is presented in Table~\ref{table: 1d_qv_benchmarking_tolerance_range_without_insulation} for the case of a bare strand. The tolerances are based on relative errors at $t=0.01~\text{s}$ of the corresponding parameters as well as the final relative errors to which the parameters converged to at $t=0.3~\text{s}$. As the convergence of all errors is clearly visible in this example, it is assumed that they do not change in further time steps of the performed simulation. However, this estimation should be carried out case by case.

 \begin{table}[H]
    \caption{Tolerance range in the quench velocity-based approach for the case of a bare strand.} 
    \vspace{-1.em} 
    \fontsize{10}{10}
    \selectfont 
    \renewcommand{\arraystretch}{1.5}
    \begin{center}
        \begin{tabular}{ c | c | cc }  
        
        \hline
        mesh size & \multirow{2}{*}{$t_\text{step range},~\upmu \text{s}$} & \multicolumn{2}{c}{tolerance, $E_\text{r}$} \\
        
        strand, m &  & $V_\text{res}$ & $T_\text{hot-spot}$ \\
        \hline
        10 & [100, 1000] & -15\% & $>-5\%$ \\
        20 & [100, 1000] & -20\% & $>-5\%$ \\
        \hline 
        \end{tabular}
    \end{center}  
     \label{table: 1d_qv_benchmarking_tolerance_range_without_insulation} 
 \end{table}
 
The results obtained from the analysis of a strand with insulation and epoxy resin are shown in Table~\ref{table: 1d_qv_benchmarking_tolerance_range_with_insulation}. Similarly to the case of a bare strand, the tolerances are based on relative errors at $t=0.01~\text{s}$ of the corresponding parameters as well as the final relative errors to which the parameters converged to at $t=0.5~\text{s}$. However, the relative error of the~hot-spot temperature does not converge to one specific value. Thus, it is assumed that the error corresponding to the hot-spot temperature increases by -2\% per second during the simulation by assuming a linear divergence with respect to the standard solution. 

 \begin{table}[H]
    \caption{Tolerance range in a quench velocity-based approach with insulation.} 
    \vspace{-1.em} 
    \fontsize{10}{10}
    \selectfont 
    \renewcommand{\arraystretch}{1.5}
    \begin{center}
        \begin{tabular}{ cc | c | cc }  
        
        \hline
        \multicolumn{2}{c|}{mesh size} & \multirow{2}{*}{$t_\text{step range},~\upmu \text{s}$} & \multicolumn{2}{c}{tolerance, $E_\text{r}$} \\
        
        strand, m & insulation, \textmu m &  & $V_\text{res}$ & $T_\text{hot-spot}$ \\
        \hline
        10 & 8 & [100, 1000] & -5\% & -1.5\% + (-2\%/s) \\
        20 & 8 & [100, 1000] & -20\% & -1.5\% + (-2\%/s) \\
        \hline 
        \end{tabular}
    \end{center}  
     \label{table: 1d_qv_benchmarking_tolerance_range_with_insulation} 
 \end{table}

It is expected that the lack of convergence of the hot-spot temperature has an influence on the evolution of the~resistive voltage being a function of temperature due to the resistivity of copper. Therefore, two relative errors can be a coupled problem in a longer simulation lasting several seconds. It is recommended to include the correction of the relative error as a function of time in the case of the resistive voltage as well. However, since the visible divergence does not occur in this study, the given step is omitted in this thesis.
