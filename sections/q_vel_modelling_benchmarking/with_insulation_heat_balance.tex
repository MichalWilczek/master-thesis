
As it was proven in Section \ref{subsection:quench_velocity_benchmarking_no_insulation_heat_balance}, the quench propagation analysis conducted with the mesh size of 1 mm and a time step in the range of $t=[10, 100]~\upmu \text{s}$ gives results remaining withing the tolerance of 1\%. Therefore, this section focuses on the mesh size across the insulation. The varying number of insulation nodes is set to $n=\{3, 5, 10, 20, 30\}$. The geometric assumptions as well as initial conditions were the same as in Section~\ref{subsection: 1D_quench_propagation_with_insulation}. 

As presented in Fig. \ref{fig: q_vel_modelling_v_quench_hot_spot_temp_with_insulation}, the hot spot temperature stabilises with 10-20 nodes used in the insulation zone. The solution with 30 nodes is used as a reference result. 

\begin{figure}[H]
\centering
    \begin{tikzpicture}
        \begin{axis}[
          width=0.7\linewidth, 
          height = 4.5cm,
          xlabel={Number of insulation elements},
          ylabel={Temperature, $\text{K}$},
          xtick={0,5,...,30},
          xticklabel style={/pgf/number format/fixed},
          ymin=20,
          ymax=28,
          xmin=0,
          xmax=30
          ]
          \addplot[red, mark=*] table[x=ins_elements,y=t_0_03,col sep=comma] {sections/q_vel_modelling_benchmarking/figures/results_with_insulation/hot_spot_temperature_vs_ins_elements.csv};
          \addplot[blue, mark=*] table[x=ins_elements,y=t_0_06,col sep=comma] {sections/q_vel_modelling_benchmarking/figures/results_with_insulation/hot_spot_temperature_vs_ins_elements.csv};
          \addplot[green, mark=*] table[x=ins_elements,y=t_0_1,col sep=comma] {sections/q_vel_modelling_benchmarking/figures/results_with_insulation/hot_spot_temperature_vs_ins_elements.csv};
          \addlegendentry{$t=0.03~\text{s}$}
          \addlegendimage{/pgfplots/refstyle=plot_resistive_voltage}\addlegendentry{$t=0.06~\text{s}$}
          \addlegendimage{/pgfplots/refstyle=plot_resistive_voltage}\addlegendentry{$t=0.1~\text{s}$}
        \end{axis}
    \end{tikzpicture}
    \caption{Hot spot temperature at $x = 0~\text{m}$ at three different time steps $t=\{0.03, 0.06, 0.1\}$ s with varying number of insulation elements.}
    \label{fig: q_vel_modelling_v_quench_hot_spot_temp_with_insulation}
\end{figure}

In Fig.~\ref{fig: q_vel_modelling_v_quench_rel_error_with_insulation}, the relative error for incremental quench velocity is presented. With 20 nodes across the insulation, the relative error of an average quench velocity remained withing the tolerance of~1\%.

\begin{figure}[H]
\centering
    \begin{tikzpicture}
        \begin{axis}[
          width=0.7\linewidth, 
          height = 4.5cm,
          xlabel={Time, $\text{s}$},
          ylabel={Relative error, \%},
          xtick={0,0.02,0.04,...,0.1},
          xticklabel style={/pgf/number format/fixed},
          xmin=0.0,
          xmax=0.1,
          legend pos=south east
          ]
          \addplot[blue, mark=*] table[x=time,y=10_ins_elems,col sep=comma] {sections/q_vel_modelling_benchmarking/figures/results_with_insulation/v_quench_rel_error.csv};
          \addplot[red, mark=*] table[x=time,y=20_ins_elems,col sep=comma] {sections/q_vel_modelling_benchmarking/figures/results_with_insulation/v_quench_rel_error.csv};
          
          \addplot[dashed, blue] table[x=time,y=10_ins_elems_average,col sep=comma] {sections/q_vel_modelling_benchmarking/figures/results_with_insulation/v_quench_rel_error.csv};
          \addplot[dashed, red] table[x=time,y=20_ins_elems_average,col sep=comma] {sections/q_vel_modelling_benchmarking/figures/results_with_insulation/v_quench_rel_error.csv};
          
          \addlegendimage{/pgfplots/refstyle=plot_resistive_voltage}\addlegendentry{10 nodes}
          \addlegendimage{/pgfplots/refstyle=plot_resistive_voltage}\addlegendentry{20 nodes}
        \end{axis}
    \end{tikzpicture}
    \caption{Quench velocity relative error for 10 and 20 nodes across the insulation layer; average relative error marked with dashed lines.}
    \label{fig: q_vel_modelling_v_quench_rel_error_with_insulation}
\end{figure}

20 nodes across the insulation layer accounts for the mesh size of $10~\upmu \text{m}$. Similarly to Section~\ref{subsection:quench_velocity_benchmarking_no_insulation_heat_balance}, this solution is used as a reference result for quench velocity-based analysis. The obtained average quench velocity is equal to $v_\text{quench}=2.18~\frac{\text{m}}{\text{s}}$.

In 1D thermal quench propagation with insulation layer, computing time rises monotonically with increase of number of nodes in the insulation layer, as presented in Fig. \ref{fig: q_vel_modelling_heat_balance_computing_time_with_insulation}. The analysis with 20 nodes lasted over 2 hours. Therefore, further refinement is not recommended.

\begin{figure}[H]
\centering
    \begin{tikzpicture}
        \begin{axis}[
          width=0.7\linewidth, 
          height = 4.5cm,
          xlabel={Number of nodes across insulation},
          ylabel={Computing time, $\text{s}$},
          xmin=0,
          xtick={0,3,5,10,20,30},
          xticklabel style={/pgf/number format/fixed},
          legend pos=north west
          ]
          \addplot[blue, mark=*] table[x=ins_elements,y=time,col sep=comma] {sections/q_vel_modelling_benchmarking/figures/results_with_insulation/heat_balance_computing_time.csv};
           \addplot[red, dashed] table[x=ins_elements,y=time_linear_approx,col sep=comma] {sections/q_vel_modelling_benchmarking/figures/results_with_insulation/heat_balance_computing_time.csv};
           
          \legend{
          computing time,
          linear approximation
          }
        \end{axis}
    \end{tikzpicture}
    \caption{Computing time vs. number of nodes.}
    \label{fig: q_vel_modelling_heat_balance_computing_time_with_insulation}
\end{figure}