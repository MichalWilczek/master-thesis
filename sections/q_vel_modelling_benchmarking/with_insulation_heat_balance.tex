
As described in the block diagram in Fig.~\ref{fig:block_diagram_benchmarking_methodology_with_insulation}, this section focuses on finding the mesh size across the insulation layer. The longitudinal mesh size and the time step range are taken from the reference solution obtained in the study of the bare strand. The geometric assumptions, the initial conditions, and the analysis settings are the same as in Section~\ref{section: 1D_quench_propagation_with_insulation}. In this analysis, the initial number of nodes accross the insulation is $n_\text{divisions, ins}=3$ which corresponds to the minimum number of nodes required to study the nonlinear material properties of the insulation. The iteration loop~$i$ is performed four times for the insulation mesh size with the increment of $\Delta n_\text{divisions, ins}$ in order to obtain the following number of insulation nodes $n_\text{divisions, ins}=\{3, 5, 10, 20, 30\}$. The number of nodes accross the insulation layer corresponds to the mesh size of $m=\{56, 34, 17, 8, 6\}~\upmu \text{m}$, respectively. Figure~\ref{fig: q_vel_modelling_v_quench_rel_error_with_insulation} shows that with 20 nodes across the insulation, the relative error of an average quench velocity remains within the tolerance of~1\% with respect to the analysis with 30 nodes. 

\begin{figure}[H]
\centering
    \begin{tikzpicture}
        \begin{axis}[
          width=0.7\linewidth, 
          height = 4.0cm,
          xlabel={$t$, $\text{s}$},
          ylabel={$E_\text{r}$, \%},
          xtick={0,0.02,0.04,...,0.1},
          xticklabel style={/pgf/number format/fixed},
          xmin=0.0,
          xmax=0.1,
          legend pos=south east
          ]
          \addplot[blue, mark=*] table[x=time,y=20_ins_elems,col sep=comma] {sections/q_vel_modelling_benchmarking/figures/results_with_insulation/v_quench_rel_error.csv};
          \addplot[dashed, blue] table[x=time,y=20_ins_elems_average,col sep=comma] {sections/q_vel_modelling_benchmarking/figures/results_with_insulation/v_quench_rel_error.csv};

        \end{axis}
    \end{tikzpicture}
    \caption{Incremental and average (dashed) quench velocity relative error for 20 nodes across the insulation layer.}
    \label{fig: q_vel_modelling_v_quench_rel_error_with_insulation}
\end{figure}

The stabilisation of the solution can also be observed in Fig. \ref{fig: q_vel_modelling_v_quench_hot_spot_temp_with_insulation}. Here, the evolution of the hot-spot temperature in time is presented as a function of a number of nodes across the insulation layer. The temperature clearly converges with 20 nodes.

\begin{figure}[H]
\centering
    \begin{tikzpicture}
        \begin{axis}[
          width=0.7\linewidth, 
          height = 4.0cm,
          xlabel={Number of insulation elements},
          ylabel={$T$, $\text{K}$},
          xtick={0,5,...,30},
          xticklabel style={/pgf/number format/fixed},
          ymin=20,
          ymax=28,
          xmin=0,
          xmax=30,
          legend pos=outer north east
          ]
          \addplot[blue, mark=*] table[x=ins_elements,y=t_0_03,col sep=comma] {sections/q_vel_modelling_benchmarking/figures/results_with_insulation/hot_spot_temperature_vs_ins_elements.csv};
          \addplot[red, mark=*] table[x=ins_elements,y=t_0_06,col sep=comma] {sections/q_vel_modelling_benchmarking/figures/results_with_insulation/hot_spot_temperature_vs_ins_elements.csv};
          \addplot[green, mark=*] table[x=ins_elements,y=t_0_1,col sep=comma] {sections/q_vel_modelling_benchmarking/figures/results_with_insulation/hot_spot_temperature_vs_ins_elements.csv};
          \addlegendentry{$t=0.03~\text{s}$}
          \addlegendimage{/pgfplots/refstyle=plot_resistive_voltage}\addlegendentry{$t=0.06~\text{s}$}
          \addlegendimage{/pgfplots/refstyle=plot_resistive_voltage}\addlegendentry{$t=0.1~\text{s}$}
        \end{axis}
    \end{tikzpicture}
    \caption{The evolution of the hot-spot temperature at three different time steps $t=\{0.03, 0.06, 0.1\}$ s as a function of a varying number of insulation elements.}
    \label{fig: q_vel_modelling_v_quench_hot_spot_temp_with_insulation}
\end{figure}

The parameters used for the benchmarking purposes with the quench velocity-based approach are summarised in Table~\ref{table: 1d_qv_benchmarking_reference_analysis_settings_with_insulation}. The reference analysis is that with a mesh size of 1~mm, a time step range of $t=[0.01, 0.1]~\text{ms}$, and 20 nodes across the insulation layer. 

\begin{table}[H]
    \caption{Analysis input parameters.} 
    \vspace{-1.em} 
    \fontsize{10}{10}
    \selectfont 
    \renewcommand{\arraystretch}{1.5}
    \begin{center}
        \begin{tabular}{ ccc }  
        \hline
        parameter & value & unit \\
        \hline
        longitudinal mesh size & 1 & [mm] \\
        insulation mesh size & 8 & [$\upmu \text{m}$] \\
        $t_\text{step range}$ & [0.01, 0.1] & [ms] \\
        $v_\text{quench, average}$ & 2.18 & [m/s] \\
        \hline 
        \end{tabular}
    \end{center}  
     \label{table: 1d_qv_benchmarking_reference_analysis_settings_with_insulation} 
 \end{table}

In the 1D quench analysis of a 1 metre-long strand with insulation and epoxy resin, the computing time rises monotonically when the number of nodes across the insulation layer increases. This is presented in Fig.~\ref{fig: q_vel_modelling_heat_balance_computing_time_with_insulation}. The analysis with 20 nodes lasts over 2 hours\footnote{The analysis was performed on the following calculation unit: Intel(R) Xeon(R) CPU E5-2667 V4 @~3.20 GHz (2~processors) with RAM 128 GB.}. Therefore, the further mesh refinement is not recommended.

\begin{figure}[H]
\centering
    \begin{tikzpicture}
        \begin{axis}[
          width=0.7\linewidth, 
          height = 4.0cm,
          xlabel={Number of nodes across insulation},
          ylabel={$t$, $\text{s}$},
          xmin=0,
          xtick={0,3,5,10,20,30},
          xticklabel style={/pgf/number format/fixed},
          legend pos=north west,
          legend pos=outer north east
          ]
          \addplot[blue, mark=*] table[x=ins_elements,y=time,col sep=comma] {sections/q_vel_modelling_benchmarking/figures/results_with_insulation/heat_balance_computing_time.csv};
           \addplot[red, dashed] table[x=ins_elements,y=time_linear_approx,col sep=comma] {sections/q_vel_modelling_benchmarking/figures/results_with_insulation/heat_balance_computing_time.csv};
           
          \legend{
          computing time,
          linear approximation
          }
        \end{axis}
    \end{tikzpicture}
    \caption{Computing time vs. number of insulation elements.}
    \label{fig: q_vel_modelling_heat_balance_computing_time_with_insulation}
\end{figure}