As presented in the benchmarking workflow in Fig.~\ref{fig:block_diagram_benchmarking_methodology_with_insulation} in Section~\ref{section:quench_velocity_benchmarking_benchmarking_methodology}, this section focuses on finding the mesh size across the insulation layer that gives results remaining within the tolerance of 1\% with respect to the average quench velocity. The input number of insulation nodes is set to $a=3$. The longitudinal mesh size as well as the time step range are taken from the reference solution without the insulation layer. The geometric assumptions as well as initial conditions were the same as in Section~\ref{section: 1D_quench_propagation_with_insulation}. 

The iteration loops~$i$ for the insulation mesh size was performed four times with the increment of $b$ (see Fig.~\ref{fig:block_diagram_benchmarking_methodology_with_insulation}) in order to obtain the following number of insulation nodes $a=\{3, 5, 10, 20, 30\}$. As shown in Fig.~\ref{fig: q_vel_modelling_v_quench_rel_error_with_insulation}, with 20 nodes across the insulation the relative error of an average quench velocity remained within the tolerance of~1\% with respect to the analysis with 30 nodes. 

\begin{figure}[H]
\centering
    \begin{tikzpicture}
        \begin{axis}[
          width=0.7\linewidth, 
          height = 4.5cm,
          xlabel={Time, $\text{s}$},
          ylabel={Relative error, \%},
          xtick={0,0.02,0.04,...,0.1},
          xticklabel style={/pgf/number format/fixed},
          xmin=0.0,
          xmax=0.1,
          legend pos=south east
          ]
          \addplot[blue, mark=*] table[x=time,y=20_ins_elems,col sep=comma] {sections/q_vel_modelling_benchmarking/figures/results_with_insulation/v_quench_rel_error.csv};
          \addplot[dashed, blue] table[x=time,y=20_ins_elems_average,col sep=comma] {sections/q_vel_modelling_benchmarking/figures/results_with_insulation/v_quench_rel_error.csv};

        \end{axis}
    \end{tikzpicture}
    \caption{Incremental and average (dashed) quench velocity relative error for 20 nodes across the insulation layer.}
    \label{fig: q_vel_modelling_v_quench_rel_error_with_insulation}
\end{figure}

As shown in Fig. \ref{fig: q_vel_modelling_v_quench_hot_spot_temp_with_insulation}, the hot spot temperature also stabilises with 20 nodes used in the insulation zone. 

\begin{figure}[H]
\centering
    \begin{tikzpicture}
        \begin{axis}[
          width=0.7\linewidth, 
          height = 4.5cm,
          xlabel={Number of insulation elements},
          ylabel={Temperature, $\text{K}$},
          xtick={0,5,...,30},
          xticklabel style={/pgf/number format/fixed},
          ymin=20,
          ymax=28,
          xmin=0,
          xmax=30
          ]
          \addplot[red, mark=*] table[x=ins_elements,y=t_0_03,col sep=comma] {sections/q_vel_modelling_benchmarking/figures/results_with_insulation/hot_spot_temperature_vs_ins_elements.csv};
          \addplot[blue, mark=*] table[x=ins_elements,y=t_0_06,col sep=comma] {sections/q_vel_modelling_benchmarking/figures/results_with_insulation/hot_spot_temperature_vs_ins_elements.csv};
          \addplot[green, mark=*] table[x=ins_elements,y=t_0_1,col sep=comma] {sections/q_vel_modelling_benchmarking/figures/results_with_insulation/hot_spot_temperature_vs_ins_elements.csv};
          \addlegendentry{$t=0.03~\text{s}$}
          \addlegendimage{/pgfplots/refstyle=plot_resistive_voltage}\addlegendentry{$t=0.06~\text{s}$}
          \addlegendimage{/pgfplots/refstyle=plot_resistive_voltage}\addlegendentry{$t=0.1~\text{s}$}
        \end{axis}
    \end{tikzpicture}
    \caption{Hot spot temperature at $x = 0~\text{m}$ at three different time steps $t=\{0.03, 0.06, 0.1\}$ s with varying number of insulation elements.}
    \label{fig: q_vel_modelling_v_quench_hot_spot_temp_with_insulation}
\end{figure}

The reference analysis for the benchmarking purposes with quench velocity-based simulations is the one with the mesh size of 1~mm and with the time step range of $t=[10, 100]~\upmu \text{s}$ and with 20 nodes across the insulation layer. 20 nodes across the insulation layer accounts for the mesh size of $10~\upmu \text{m}$. The analysis settings used for the benchmarking with the quench velocity-based method are summarised in Table~\ref{table: 1d_qv_benchmarking_reference_analysis_settings_with_insulation}. 

\begin{table}[H]
    \caption{Analysis input parameters.} 
    \vspace{-1.em} 
    \fontsize{10}{10}
    \selectfont 
    \renewcommand{\arraystretch}{1.5}
    \begin{center}
        \begin{tabular}{ ccc }  
        \hline
        parameter & value & unit \\
        \hline
        longitudinal mesh size & 1 & [mm] \\
        insulation mesh size & 10 & [$\upmu \text{m}$] \\
        time step range & [10, 100] & [\textmu s] \\
        $v_\text{quench, average}$ & 2.18 & [m/s] \\
        \hline 
        \end{tabular}
    \end{center}  
     \label{table: 1d_qv_benchmarking_reference_analysis_settings_with_insulation} 
 \end{table}

In 1D thermal quench propagation with insulation layer, computing time rises linearly with increase of number of nodes in the insulation layer, as presented in Fig. \ref{fig: q_vel_modelling_heat_balance_computing_time_with_insulation}. The analysis with 20 nodes lasted over 2 hours. Therefore, further refinement is not recommended.

\begin{figure}[H]
\centering
    \begin{tikzpicture}
        \begin{axis}[
          width=0.7\linewidth, 
          height = 4.5cm,
          xlabel={Number of nodes across insulation},
          ylabel={Computing time, $\text{s}$},
          xmin=0,
          xtick={0,3,5,10,20,30},
          xticklabel style={/pgf/number format/fixed},
          legend pos=north west
          ]
          \addplot[blue, mark=*] table[x=ins_elements,y=time,col sep=comma] {sections/q_vel_modelling_benchmarking/figures/results_with_insulation/heat_balance_computing_time.csv};
           \addplot[red, dashed] table[x=ins_elements,y=time_linear_approx,col sep=comma] {sections/q_vel_modelling_benchmarking/figures/results_with_insulation/heat_balance_computing_time.csv};
           
          \legend{
          computing time,
          linear approximation
          }
        \end{axis}
    \end{tikzpicture}
    \caption{Computing time vs. number of nodes.}
    \label{fig: q_vel_modelling_heat_balance_computing_time_with_insulation}
\end{figure}