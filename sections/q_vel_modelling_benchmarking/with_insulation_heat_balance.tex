
In this part of the section, the heat balance - based analysis of a one metre-long strand was conducted with insulation. The created 1D+1D geometry does not vary with respect to Section \ref{subsection: 1D_quench_propagation_with_insulation} and is presented in Fig. \ref{fig: 1d_strand_geometry_with_insulation}. As it was proven in Section \ref{subsection:quench_velocity_benchmarking_no_insulation_heat_balance}, the longitudinal mesh of 1000 nodes with time step range of $t=[10, 100]~\mu \text{s}$ is sufficient to simulate a longitudinal quench propagation with a relative error lower than 1\%. Therefore, in this section, the attention is paid to the number of elements across the insulation which would assure the proper temperature distribution in the strand along its length. The analysis was conducted with 3, 5, 10, 20 and 30 elements across the insulation layer.

In Fig. \ref{fig: q_vel_modelling_v_quench_hot_spot_temp_with_insulation}, the hot spot temperature was presented a variable of a number of elements across the insulation. One can notice that starting from 20 elements, the temperature reaches a stable value for each time step. As presented in Fig. \ref{fig: q_vel_modelling_v_quench_rel_error_with_insulation}, the change of relative error for quench velocity was compared taking the solution with 30 elements as a benchmark. The average quench velocity for 20 insulation elements did not vary by more than 1\%. 

\begin{figure}[h!]
\centering
    \begin{tikzpicture}
        \begin{axis}[
          width=0.7\linewidth, 
          height = 4.5cm,
          xlabel={Number of insulation elements},
          ylabel={Temperature, $\text{K}$},
          xtick={0,5,...,30},
          xticklabel style={/pgf/number format/fixed},
          ymin=20,
          ymax=28,
          xmin=0,
          xmax=30
          ]
          \addplot[red, mark=*] table[x=ins_elements,y=t_0_03,col sep=comma] {sections/q_vel_modelling_benchmarking/figures/results_with_insulation/hot_spot_temperature_vs_ins_elements.csv};
          \addplot[blue, mark=*] table[x=ins_elements,y=t_0_06,col sep=comma] {sections/q_vel_modelling_benchmarking/figures/results_with_insulation/hot_spot_temperature_vs_ins_elements.csv};
          \addplot[green, mark=*] table[x=ins_elements,y=t_0_1,col sep=comma] {sections/q_vel_modelling_benchmarking/figures/results_with_insulation/hot_spot_temperature_vs_ins_elements.csv};
          \addlegendentry{$t=0.03~\text{s}$}
          \addlegendimage{/pgfplots/refstyle=plot_resistive_voltage}\addlegendentry{$t=0.06~\text{s}$}
          \addlegendimage{/pgfplots/refstyle=plot_resistive_voltage}\addlegendentry{$t=0.1~\text{s}$}
        \end{axis}
    \end{tikzpicture}
    \caption{Hot spot temperature at $x = 0~\text{m}$ at three different time steps $t=\{0.03, 0.06, 0.1\}$ s with varying number of insulation elements.}
    \label{fig: q_vel_modelling_v_quench_hot_spot_temp_with_insulation}
\end{figure}

\begin{figure}[h!]
\centering
    \begin{tikzpicture}
        \begin{axis}[
          width=0.7\linewidth, 
          height = 4.5cm,
          xlabel={Time, $\text{s}$},
          ylabel={Relative error, \%},
          xtick={0,0.02,0.04,...,0.1},
          xticklabel style={/pgf/number format/fixed},
          xmin=0.0,
          xmax=0.1
          ]
          \addplot[blue, mark=*] table[x=time,y=10_ins_elems,col sep=comma] {sections/q_vel_modelling_benchmarking/figures/results_with_insulation/v_quench_rel_error.csv};
          \addplot[red, mark=*] table[x=time,y=20_ins_elems,col sep=comma] {sections/q_vel_modelling_benchmarking/figures/results_with_insulation/v_quench_rel_error.csv};
          
          \addplot[dashed, blue] table[x=time,y=10_ins_elems_average,col sep=comma] {sections/q_vel_modelling_benchmarking/figures/results_with_insulation/v_quench_rel_error.csv};
          \addplot[dashed, red] table[x=time,y=20_ins_elems_average,col sep=comma] {sections/q_vel_modelling_benchmarking/figures/results_with_insulation/v_quench_rel_error.csv};
        \end{axis}
    \end{tikzpicture}
    \caption{Quench velocity relative error for 10 (in blue) and 20 (in red) elements across the insulation layer; average error marked with dashed lines.}
    \label{fig: q_vel_modelling_v_quench_rel_error_with_insulation}
\end{figure}

It can be concluded that, if the simulation is supposed to be analysed, 20 insulation elements should be applied. It accounts for approximately $10~\mu\text{m}$ long insulation element. Similarly to Section \ref{subsection:quench_velocity_benchmarking_no_insulation_heat_balance}, the solution with 20 insulation elements will be used as a further benchmark result for quench velocity simulation presented in next section. In 1D thermal quench propagation with insulation layer, computing time rises linearly too, as presented in Fig. \ref{fig: q_vel_modelling_heat_balance_computing_time_with_insulation}.

\begin{figure}[h!]
\centering
    \begin{tikzpicture}
        \begin{axis}[
          width=0.7\linewidth, 
          height = 4.5cm,
          xlabel={Number of elements across insulation},
          ylabel={Computing time, $\text{s}$},
          xmin=0,
          xtick={0,3,5,10,20,30},
          xticklabel style={/pgf/number format/fixed},
          ]
          \addplot[blue, mark=*] table[x=ins_elements,y=time,col sep=comma] {sections/q_vel_modelling_benchmarking/figures/results_with_insulation/heat_balance_computing_time.csv};
        \end{axis}
    \end{tikzpicture}
    \caption{Computing time vs. number of nodes.}
    \label{fig: q_vel_modelling_heat_balance_computing_time_with_insulation}
\end{figure}