From the magnet design standpoint, the key parameters are:
\begin{enumerate}
    \item peak voltage to ground,
    \item maximum hot-spot temperature.
\end{enumerate}

Temperature is rarely measured in experiments because there is no space to mount temperature sensors inside the magnets. It is physically deduced from resistive voltage measured across the magnet as quench propagates. The place where a quench starts propagating is usually characterised by the highest temperature in the entire magnet during quench (also referred as the hot-spot temperature). The rise of resistive voltage inside the magnet must also remain within the allowable voltage-to-ground limits as well as turn-to-turn voltage limits imposed by electrical properties of the insulation. Exceeding these limits may lead to:

\begin{itemize}
    \item short-circuit between different windings of the coil across the internal windings' insulation,
    \item short-circuit between the winding and the ground across the ground insulation of the magnet.
\end{itemize}

The hot spot temperature and the voltage-to-ground value allow for defining whether the materials reach their safety limits at which they loose mechanical, thermal or electro-magnetic properties leading to the magnet destruction.
