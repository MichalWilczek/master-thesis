
During the measurements, the spot heater was attached to the external ground insulation of one of the four coils of the magnet. The contact area of the heater was 4~$\text{mm}^2$ and it was centered at the layer 13 on the aperture side of the magnet, as presented in Fig.~\ref{fig:spot_heater_placement}.

\begin{figure}[H]
\centering
    \begin{tikzpicture}
        \begin{axis}[
          width=0.5\linewidth, 
          height=0.5\linewidth,
          xtick={0.0, 24.46},
          ytick={0.0, 27.30},
          xlabel={$x,~\text{mm}~\text{(layers direction)}$},
          ylabel={$y,~\text{mm}~\text{(turns direction)}$},
          xmajorgrids=true,
          ymajorgrids=true,
          xmin=-5.0,
          xmax=29.47,
          ymin=-5.0,
          ymax=32.29,
          ]
          \addplot[blue, only marks, mark size=1pt] table[x=x,y=y,col sep=comma] {sections/skew_quad_q_det/figures/skew_quad_design/winding_location_cross_section.csv};
        \end{axis}
        \draw[scale=0.172, black, ultra thick] (3.0,4.5) -- (32.0,4.5); 
        \node[black, scale=0.8] at (4.9,0.5) {ground insulation};
        \filldraw[scale=0.172, red] (0.471+5.35+0.941*12,3.7) circle (13pt);
        \node[red, scale=0.8] at (3.0,0.3) {spot heater};

    \end{tikzpicture}
    \caption{Location of the spot heater with respect to the coil cross-section~\cite{marco_prioli_mails}.}
    \label{fig:spot_heater_placement}
\end{figure}

The coil is quenched by the heat generated in the spot-heater. The heat is a~result of a~current discharge in the~\nth{1} order RC circuit as shown in Fig.~\ref{fig:spot_heater_capacitor_discharge}. Its capacitance was equal to $C=14~\text{mF}$.

\begin{figure}[H]
	\centering
	\begin{tikzpicture}[american resistors, scale = 0.8] 
	\draw[semithick] 
	(-2,0)
	to[C,l^=$C$] (-2,2) -- (2,2);
	\draw[semithick] 
	(-2,0) -- (2,0)
	to[R,l^=$R_\text{spot heater}$] (2,2);
	\draw[semithick, red] (3,-0.5) -- (3,2.5) -- (6,2.5) -- (6,-0.5) -- (3,-0.5);
	\node[red, scale=0.8] at (4.5,1) {magnet domain};
	\end{tikzpicture}
	\caption{Schematic of heating the magnet by means of a spot heater.}
	\label{fig:spot_heater_capacitor_discharge}
\end{figure}

The spot heater resistance is calculated as
\begin{equation}
    R_\text{spot heater} = \frac{\tau}{C},
\end{equation}
where $\tau$ -- time constant, $C$ -- capacitance. The electric field of the capacitor is directly dissipated in the spot-heater. Therefore, the power generation at the spot heater is calculated as
\begin{equation}
    P(t)_\text{spot heater} = -P(t)_\text{capacitor} = -\frac{1}{\text{dt}} (E_\text{C}) = -\frac{1}{\text{dt}} (\frac{1}{2} \text{C}V_\text{C}^2),
    \label{eqn:power_dissipation}
\end{equation}
where $C$ -- capacitance, $V_\text{C}$ -- voltage across the capacitor, $E_\text{C}$ -- energy stored in the capacitor, $P(t)$ -- power being a function of time. Fig.~\ref{fig:capacitor_discharge} presents the measured drop of a~capacitive voltage when the capacitor was discharging. The same Figure also presents the~deduced power dissipation at the~spot heater based on~(\ref{eqn:power_dissipation}). The sampling frequency during the measurements was equal to $f=5000~\text{Hz}$. Based on the power curve in Fig.~\ref{fig:capacitor_discharge}, one can notice that most of energy was transmitted to the coil during the first $t=5~\text{ms}$ and the capacitor was fully discharged at $t=10~\text{ms}$.

\begin{figure}[H]
\centering
\begin{tikzpicture}
\pgfplotsset{
    scale only axis,
    width=0.7\linewidth, 
    height = 3.5cm,
    compat=1.3,
    xmin=0, xmax=20.0,
    xticklabel style={/pgf/number format/fixed},
    xtick={0,5,10,15,20},
    legend pos=north east
}
\begin{axis}[
  axis y line*=left,
  ymin=0, ymax=20,
  xlabel={Time, ms},
  ylabel={Capacitive voltage,~V},
]
\addplot[smooth, red] table[x=time,y=v_capacitor,col sep=comma] {sections/skew_quad_q_det/figures/measurements/capacitor_discharge.csv};
\label{plot_voltage_discharge_capacitor}
\end{axis}

\begin{axis}[
  axis y line*=right,
  axis x line=none,
  ymin=0, ymax=2200,
  ylabel={Power, W},
]
\addplot[smooth, blue] table[x=time,y=p_capacitor,col sep=comma] {sections/skew_quad_q_det/figures/measurements/capacitor_discharge.csv}; 
\label{plot_power_deposition_resitor}

\addlegendentry{$P_\text{spot heater}$}
\addlegendimage{/pgfplots/refstyle=plot_voltage_discharge_capacitor}\addlegendentry{$V_\text{C}$}

\end{axis}
\end{tikzpicture}
\caption{Voltage drop across the capacitor and heating power curve at the spot heater.}
\label{fig:capacitor_discharge}
\end{figure}

With the shape of the voltage drop, the time constant and spot heater resistance were deduced, as shown in Table~\ref{table:rc_circuit_characteristics}. Moreover, the entire capacitive energy deposited in the~magnet was approximately equal to $E_\text{C}=2.5~\text{J}$.

 \begin{table}[H]
    \caption{RC heating circuit characteristics.} 
    \vspace{-1.em} 
    \fontsize{10}{10}
    \selectfont 
    \renewcommand{\arraystretch}{1.5}
    \begin{center}
        \begin{tabular}{ ccc } 
        \hline
        parameter & value & unit \\
        \hline
        $\tau$ & 3.5 & [ms] \\
        $R_\text{spot heater}$ & 0.25 & [\textOmega] \\
        $E_\text{C}$ & 2.5 & [J] \\
        \hline 
        \end{tabular}
    \end{center}  
     \label{table:rc_circuit_characteristics} 
 \end{table}

The schematic electrical circuit of the skew quadrupole is presented in Fig.~\ref{fig:skew_quad_electrical_scheme}. Four coils are connected in series. Each of them is characterised by the same inductance $L_\text{coil}$ that are a function of current. During the measurements, the heat was not transmitted to other coils of the skew quadrupole, i.e. only one coil out of four quenched. $R_\text{coil}$ in Fig.~\ref{fig:skew_quad_electrical_scheme} represents the internal resistance of the coil where the spot heater was placed. When the skew quadrupole is in the superconducting state, $R_\text{coil}=0$. After the quench is initiated by depositing heat through the spot heater, $R_\text{coil}$ represented by a varying resistance starts increasing. In order to detect a quench, the voltages $V_1$ and $V_2$ are compared. If their difference exceeds a threshold value, $V_\text{th}$ during a certain period of time, the quench is detected. The time after which the quench is detected after exceeding the threshold $V_\text{th}$ is referred as a validation time, $t_\text{validation}$. When the quench is detected, the power converter $P_\text{converter}$ is switched off and the energy extraction switch $E_\text{switch}$ is closed. At this moment the magnet starts discharging.

\begin{figure}[H]
	\centering
	\begin{tikzpicture}[american currents, american inductors, american resistors, scale = 0.8] 
	\draw[semithick] 
	(0,-0.5) -- (0,2)
	to[vL,l^=$L_\text{coil}$, o-] (2,2)
	to[vL,l^=$L_\text{coil}$, -o] (4,2)
	to[vL,l^=$L_\text{coil}$] (6,2)
	to[vL,l^=$L_\text{coil}$] (8,2)
	to[vR,l^=$R_\text{coil}$, -o] (10,2) -- (10,-0.5)
	(0,-0.5) to[I, l_=$P_\text{converter}$, i^=$I$]  (10,-0.5);
	\draw[semithick]
	(4,2) -- (4,4)
	to[voltmeter, l=$V_1$] (10,4) -- (10,2);
    \draw[semithick]
    (0,2) -- (0,4)
    to[voltmeter, l=$V_2$] (4,4);
    \draw[semithick]
    (3,-0.5) -- (3,0.5)
    to[closing switch, l=$E_\text{switch}$] (5,0.5) 
    to[R,l^=$R_\text{extraction}$] (7,0.5) -- (7,-0.5);
    \draw[semithick]
    (0,-0.5) -- (0,-0.5) node[ground]{}; 
	\end{tikzpicture}
	\caption{Circuit scheme of the skew quadrupole with energy extraction system.}
	\label{fig:skew_quad_electrical_scheme}
\end{figure}

The quench detection system parameters are presented in Table~\ref{table:qds_characteristics}. The voltage threshold $V_\text{th}$ should be exceeded during the validation time of $t_\text{validation}=20~\text{ms}$ in order to switch off the power converter connected to the the magnet.

\begin{table}[H]
    \caption{Quench detection system characteristics.} 
    \vspace{-1.em} 
    \fontsize{10}{10}
    \selectfont 
    \renewcommand{\arraystretch}{1.5}
    \begin{center}
        \begin{tabular}{ ccc } 
        \hline
        parameter & value & unit \\
        \hline
        $V_\text{th}$ threshold & 0.2 & [V] \\
        $t_\text{validation}$ & 20 & [ms] \\
        \hline 
        \end{tabular}
    \end{center}  
     \label{table:qds_characteristics} 
\end{table}

After the quench is detected, the circuit is represented, as shown in Fig.~\ref{fig:skew_quad_discharge_electrical_scheme}. The total magnet inductance $L_\text{magnet}$ is equal to the sum of four equal inductances $L_\text{coil}$. The initial discharge current $I_0$ is equal to the operating current of the magnet during the measurements.

\begin{figure}[H]
	\centering
	\begin{tikzpicture}[american currents, american inductors, american resistors, scale = 0.8] 
	\draw[semithick] 
	(0,0) -- (0,2)
	to[vL,l^=$L_\text{magnet}$, i_=$I_0$] (6,2)
	to[vR,l^=$R_\text{coil}$] (6,0)
	to[R,l^=$R_\text{extraction}$] (0,0);
    \draw[semithick]
    (0,0) -- (0,0) node[ground]{}; 
	\end{tikzpicture}
	\caption{Simplified circuit scheme of the skew quadrupole after the quench is detected.}
	\label{fig:skew_quad_discharge_electrical_scheme}
\end{figure}

If the magnet is self-protected, $R_\text{extraction}=0$. In case of a skew quadrupole, $R_\text{extraction}=1.557$~\textOmega. One should remember that $\tau$ is not constant during the discharge due to the nonlinear inductance as a function of current and increasing $R_\text{coil}$ as the quench propagates.The time constant of this circuit is calculated as 
\begin{equation}
    \tau = \frac{L_\text{magnet}}{R_\text{coil}+R_\text{extraction}}.
    \label{eqn:variable_time_constant}
\end{equation}

The quench measurements were conducted for $I=86~\text{A}$ which is a lower value than the designed nominal operating current of the skew quadrupole. As presented in Fig.~\ref{fig:skew_quad_discharge_before_quench_detection}, the quench resistive voltage reaches the threshold value for the quench detection system at $t=130~\text{ms}$. The quench is detected at $t=150~\text{ms}$.

\begin{figure}[H]
\centering
\begin{tikzpicture}
\pgfplotsset{
    scale only axis,
    height=3.5cm,
    width=8cm,
    compat=1.3,
    scaled x ticks=base 10:3,
    xmin=0, xmax=0.15,
    legend pos=south west
}
\begin{axis}[
  axis y line*=left,
  ymin=0, ymax=0.35,
  xlabel={Time, s},
  ylabel={Resistive Voltage, V},
]
\addplot[smooth, red] table[x=Time,y=Resistive_Voltage,col sep=comma] {sections/skew_quad_q_det/figures/measurements/current_voltage_quench_detection.csv}; 
\label{plot_resistive_voltage_rise_quench_detection}
\addplot[smooth, black, dashed] table[x=Time,y=QDS_v_threshold,col sep=comma] {sections/skew_quad_q_det/figures/measurements/current_voltage_quench_detection.csv};
\label{plot_resistive_voltage_qds_threshold}
\end{axis}

\begin{axis}[
  axis y line*=right,
  axis x line=none,
  ymin=0, ymax=100,
  ylabel={Current, A},
]
\addplot[smooth, blue] table[x=Time,y=Current,col sep=comma] {sections/skew_quad_q_det/figures/measurements/current_voltage_quench_detection.csv}; 
\label{plot_current_constant_quench_detection}
\addlegendentry{$I$}
\addlegendimage{/pgfplots/refstyle=plot_resistive_voltage_qds_threshold}\addlegendentry{$V_\text{th}$}
\addlegendimage{/pgfplots/refstyle=plot_resistive_voltage_rise_quench_detection}\addlegendentry{$V_\text{res}$}

\end{axis}
\end{tikzpicture}
\caption{Resistive voltage rise as a function of time at constant current before quench detection.}
\label{fig:skew_quad_discharge_before_quench_detection}
\end{figure}

Fig.~\ref{fig:skew_quad_discharge} presents the current profile as well as the change of resistive voltage across the~magnet during the discharge of the skew quadrupole. Resistive voltage $V_\text{res}$ is a function of coil resistance $R_\text{coil}$ and the value of current in the circuit. Until $t=1.5~\text{s}$, the rise of voltage up to approximately 16~V is observed due to the increase of coil resistance mainly because of a turn-to-turn propagation. After this time, the voltage drops down because the~current decreases rapidly in the circuit. The smooth shape of a current curve indicates that the quench back did not occur during the discharge which can be explained by a~relatively long discharge time. The quench measurements ended at $t=9~\text{s}$.

\begin{figure}[H]
\centering
\begin{tikzpicture}
\pgfplotsset{
    scale only axis,
    width=0.7\linewidth, 
    height = 3.5cm,
    compat=1.3,
    scaled x ticks=base 10:0,
    xmin=0, xmax=9
}
\begin{axis}[
  axis y line*=left,
  ymin=0, ymax=20,
  xlabel={Time, s},
  ylabel={Resistive Voltage, V},
]
\addplot[smooth, red] table[x=Time,y=Resistive_Voltage,col sep=comma] {sections/skew_quad_q_det/figures/measurements/current_voltage_discharge.csv}; 
\label{plot_resistive_voltage_entire_discharge_skew_quad}
\end{axis}
\begin{axis}[
  axis y line*=right,
  axis x line=none,
  ymin=0, ymax=100,
  ylabel={Current, A}
]
\addplot[smooth, blue] table[x=Time,y=Current,col sep=comma] {sections/skew_quad_q_det/figures/measurements/current_voltage_discharge.csv}; 
\label{plot_current_entire_discharge_skew_quad}
\addlegendentry{$I$}
\addlegendimage{/pgfplots/refstyle=plot_resistive_voltage_entire_discharge_skew_quad}\addlegendentry{$V_\text{res}$}
\end{axis}
\end{tikzpicture}
\caption{Current and Resistive Voltage change during the discharge of skew quadrupole}
\label{fig:skew_quad_discharge}
\end{figure}

