
During the experiments, the spot heater was attached to the external grounding insulation of the magnet. The contact area of the heater was 4~$\text{mm}^2$ and it was centered at the layer 13 on the aperture side of the magnet, as presented in Fig.~\ref{fig:spot_heater_placement}. 

\begin{figure}[H]
\centering
    \begin{tikzpicture}
        \begin{axis}[
          width=0.5\linewidth, 
          height=0.5\linewidth,
          xtick={0.0, 24.46},
          ytick={0.0, 27.30},
          xlabel={$x,~\text{mm}~\text{(layers direction)}$},
          ylabel={$y,~\text{mm}~\text{(turns direction)}$},
          xmajorgrids=true,
          ymajorgrids=true,
          xmin=-5.0,
          xmax=29.47,
          ymin=-5.0,
          ymax=32.29,
          ]
          \addplot[blue, only marks, mark size=1pt] table[x=x,y=y,col sep=comma] {sections/skew_quad_q_det/figures/winding_location_cross_section.csv};
        \end{axis}
        \draw[scale=0.172, black, ultra thick] (3.0,4.5) -- (32.0,4.5); 
        \node[black, scale=0.8] at (4.9,0.5) {ground insulation};
        \filldraw[scale=0.172, red] (0.471+5.35+0.941*12,3.7) circle (13pt);
        \node[red, scale=0.8] at (3.0,0.3) {spot heater};

    \end{tikzpicture}
    \caption{Location of the spot heater with respect to the cross-section of a  half-quadrant~\cite{marco_prioli_mails}.}
    \label{fig:spot_heater_placement}
\end{figure}

As presented in Fig.~\ref{fig:spot_heater_capacitor_discharge}, quenching the magnet occurs by discharging the capacitor being the RC system of the \nth{1}~order. Its capacitance was equal to $C=14~\text{mF}$.

\begin{figure}[H]
	\centering
	\begin{tikzpicture}[american resistors, scale = 0.8] 
	\draw[semithick] 
	(-2,0)
	to[C,l^=$C$] (-2,2) -- (2,2);
	\draw[semithick] 
	(-2,0) -- (2,0)
	to[R,l^=$R_\text{spot heater}$] (2,2);
	\draw[semithick, red] (3,-0.5) -- (3,2.5) -- (6,2.5) -- (6,-0.5) -- (3,-0.5);
	\node[red, scale=0.8] at (4.5,1) {magnet domain};
	\end{tikzpicture}
	\caption{Schematic of heating the magnet by means of a spot heater.}
	\label{fig:spot_heater_capacitor_discharge}
\end{figure}

The spot heater resistance is calculated as
\begin{equation}
    R_\text{spot heater} = \frac{\tau}{C},
\end{equation}
where $\tau$ -- time constant, $C$ -- capacitance. The energy stored in the capacitor is directly dissipated in the spot heater. Therefore, the power generation at the spot heater is calculated as
\begin{equation}
    P(t)_\text{spot heater} = -P(t)_\text{capacitor} = -\frac{1}{\text{dt}} (\frac{1}{2} \text{C}V_\text{C}^2),
\end{equation}
where $C$ -- constant capacitance, $V_\text{C}$ -- voltage across the capacitor, $P(t)$ -- power being a function of time. Fig.~\ref{fig:capacitor_discharge} presents the measured drop of capacitive voltage during quenching of the magnet as well as the power dissipation at the spot heater. The sampling frequency during the measurements was equal to $f=5000~\text{Hz}$. It can be concluded that the energy is sent to the system in approximately $t=5~\text{ms}$.

\begin{figure}[H]
\centering
\begin{tikzpicture}
\pgfplotsset{
    scale only axis,
    width=0.7\linewidth, 
    height = 3.5cm,
    compat=1.3,
    xmin=0, xmax=0.02,
    xticklabel style={/pgf/number format/fixed},
    xtick={0,0.005,0.01,0.015,0.02},
    legend pos=north east
}
\begin{axis}[
  axis y line*=left,
  ymin=0, ymax=20,
  xlabel={Time, s},
  ylabel={Capacitive voltage,~V},
]
\addplot[smooth, red] table[x=time,y=v_capacitor,col sep=comma] {sections/skew_quad_q_det/figures/capacitor_discharge.csv};
\label{plot_voltage_discharge_capacitor}
\end{axis}

\begin{axis}[
  axis y line*=right,
  axis x line=none,
  ymin=0, ymax=2200,
  ylabel={Power, W},
]
\addplot[smooth, blue] table[x=time,y=p_capacitor,col sep=comma] {sections/skew_quad_q_det/figures/capacitor_discharge.csv}; 
\label{plot_power_deposition_resitor}

\addlegendentry{$P_\text{spot heater}$}
\addlegendimage{/pgfplots/refstyle=plot_voltage_discharge_capacitor}\addlegendentry{$V_\text{C}$}

\end{axis}
\end{tikzpicture}
\caption{Voltage drop across the capacitor and heating power curve at the spot heater.}
\label{fig:capacitor_discharge}
\end{figure}

With the shape of the voltage drop, the time constant and spot heater resistance were deduced, as shown in Table~\ref{table:rc_circuit_characteristics}. Moreover, the entire energy deposited in the magnet was approximately equal to $E=2.5~\text{J}$.

 \begin{table}[H]
    \caption{RC heating circuit characteristics.} 
    \vspace{-1.em} 
    \fontsize{10}{10}
    \selectfont 
    \renewcommand{\arraystretch}{1.5}
    \begin{center}
        \begin{tabular}{ ccc }  
        \hline
        time constant & 3.5 & [ms] \\
        spot heater resistance & 0.25 & [\textOmega] \\
        energy deposition & 2.5 & [J] \\
        \hline 
        \end{tabular}
    \end{center}  
     \label{table:rc_circuit_characteristics} 
 \end{table}

The schematic electrical circuit of the skew quadrupole is presented in Fig.~\ref{fig:skew_quad_electrical_scheme}. Four quadrants are connected in series. Each of them is characterised by the same inductance $L_\text{quad}$. When the entire magnet is in the superconducting state, $R_\text{quench}=0$. In order to detect the quench, the voltages $V_1$ and $V_2$ are compared. If their difference exceeds a threshold value during a certain period of time, the quench detection system (QDS) is switched on. It cuts of the power source and allows the magnet to discharge by itself as an RL circuit. During the discharge, the resistance $R_\text{quench}$ grows because the temperature of one of the quadrants rises with time. Fig.~\ref{fig:skew_quad_electrical_scheme} shows the principles of work of the quench detection system if only one quadrant quenches. This is the case of the quench measurements for the skew quadrupole. During the entire duration of the discharge only one quadrant was in a normal conducting state.

\begin{figure}[H]
	\centering
	\begin{tikzpicture}[american currents, american inductors, american resistors, scale = 0.8] 
	\draw[semithick] 
	(0,-0.5) -- (0,2)
	to[L,l^=$L_\text{quad}$, o-] (2,2)
	to[L,l^=$L_\text{quad}$, -o] (4,2)
	to[L,l^=$L_\text{quad}$] (6,2)
	to[L,l^=$L_\text{quad}$] (8,2)
	to[R,l^=$R_\text{quench}$, -o] (10,2) -- (10,-0.5)
	(0,-0.5) to[I, i^=$I$]  (10,-0.5);
	\draw[semithick]
	(4,2) -- (4,4)
	to[voltmeter, l=$V_1$] (10,4) -- (10,2);
    \draw[semithick]
    (0,2) -- (0,4)
    to[voltmeter, l=$V_2$] (4,4);
    \draw[semithick]
    (3.5,-0.5) -- (3.5,0.5)
    to[closing switch, l=QDS] (6.5,0.5) -- (6.5,-0.5);
	\end{tikzpicture}
	\caption{Circuit scheme of the skew quadrupole without energy extraction system.}
	\label{fig:skew_quad_electrical_scheme}
\end{figure}

QDS characteristics is presented in Table~\ref{table:qds_characteristics}. The voltage threshold $V_\text{r}$ should be measured during the period of $=20~\text{ms}$ in order to switch of the power supply in the magnet.

\begin{table}[H]
    \caption{QDS characteristics.} 
    \vspace{-1.em} 
    \fontsize{10}{10}
    \selectfont 
    \renewcommand{\arraystretch}{1.5}
    \begin{center}
        \begin{tabular}{ ccc }  
        \hline
        $V_\text{r}$ threshold & 0.2 & [V] \\
        time delay & 20 & [ms] \\
        \hline 
        \end{tabular}
    \end{center}  
     \label{table:qds_characteristics} 
\end{table}

The quench measurements were conducted for $I=86~\text{A}$ which is a lower value than the designed nominal operating current of the skew quadrupole. As presented in Fig.~\ref{fig:skew_quad_discharge_before_quench_detection}, the quench resistive voltage reaches the threshold value for the QDS at $t=130~\text{ms}$. The quench is detected at $t=150~\text{ms}$.

\begin{figure}[H]
\centering
\begin{tikzpicture}
\pgfplotsset{
    scale only axis,
    height=4cm,
    width=8cm,
    compat=1.3,
    scaled x ticks=base 10:3,
    xmin=0, xmax=0.15,
    legend pos=south west
}
\begin{axis}[
  axis y line*=left,
  ymin=0, ymax=0.35,
  xlabel={Time, s},
  ylabel={Resistive Voltage, V},
]
\addplot[smooth, red] table[x=Time,y=Resistive_Voltage,col sep=comma] {sections/skew_quad_q_det/figures/current_voltage_quench_detection.csv}; 
\label{plot_resistive_voltage_rise_quench_detection}
\addplot[smooth, black, dashed] table[x=Time,y=QDS_v_threshold,col sep=comma] {sections/skew_quad_q_det/figures/current_voltage_quench_detection.csv};
\label{plot_resistive_voltage_qds_threshold}
\end{axis}

\begin{axis}[
  axis y line*=right,
  axis x line=none,
  ymin=0, ymax=100,
  ylabel={Current, A},
]
\addplot[smooth, blue] table[x=Time,y=Current,col sep=comma] {sections/skew_quad_q_det/figures/current_voltage_quench_detection.csv}; 
\label{plot_current_constant_quench_detection}
\addlegendentry{$I$}
\addlegendimage{/pgfplots/refstyle=plot_resistive_voltage_qds_threshold}\addlegendentry{$V_\text{r, QDS}$}
\addlegendimage{/pgfplots/refstyle=plot_resistive_voltage_rise_quench_detection}\addlegendentry{$V_\text{r}$}

\end{axis}
\end{tikzpicture}
\caption{Resistive voltage rise as a function of time at constant current before quench detection.}
\label{fig:skew_quad_discharge_before_quench_detection}
\end{figure}

Fig.~\ref{fig:skew_quad_discharge} presents the current profile as well as the change of resistive voltage across the magnet during the discharge of the skew quadrupole. Being a function of current and resistance, the resistive voltage is a result of the quench propagation inside the magnet as well as the decreasing current with time. The discharge ends at $t=9~\text{s}$.

\begin{figure}[H]
\centering
\begin{tikzpicture}
\pgfplotsset{
    scale only axis,
    width=0.7\linewidth, 
    height = 3.5cm,
    compat=1.3,
    scaled x ticks=base 10:0,
    xmin=0, xmax=9
}
\begin{axis}[
  axis y line*=left,
  ymin=0, ymax=20,
  xlabel={Time, s},
  ylabel={Resistive Voltage, V},
]
\addplot[smooth, red] table[x=Time,y=Resistive_Voltage,col sep=comma] {sections/skew_quad_q_det/figures/current_voltage_discharge.csv}; 
\label{plot_resistive_voltage_entire_discharge_skew_quad}
\end{axis}
\begin{axis}[
  axis y line*=right,
  axis x line=none,
  ymin=0, ymax=100,
  ylabel={Current, A}
]
\addplot[smooth, blue] table[x=Time,y=Current,col sep=comma] {sections/skew_quad_q_det/figures/current_voltage_discharge.csv}; 
\label{plot_current_entire_discharge_skew_quad}
\addlegendentry{$I$}
\addlegendimage{/pgfplots/refstyle=plot_resistive_voltage_entire_discharge_skew_quad}\addlegendentry{$V_\text{r}$}
\end{axis}
\end{tikzpicture}
\caption{Current and Resistive Voltage change during the discharge of skew quadrupole}
\label{fig:skew_quad_discharge}
\end{figure}