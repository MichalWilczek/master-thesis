
This chapter describes the thermal quench analysis of a skew quadrupole developed by LASA laboratories of INFN-Milano. The skew quadrupole belongs to the group of high-order corrector magnets designed for the upgrade of High-Luminosity LHC. Their design assumes using no quench protection devices such as quench heaters or CLIQ except for crowbars \footnote{Describe what crowbars are...}. Therefore, when any of these magnet quenches, the discharge of energy stored in their magnetic field occurs merely by their own rise in resistivity. i.e. they are self-protected.

Quench velocity approach presented in chapter \ref{section:quench_velocity_modelling} is used to analyze the magnet as a 3-dimensional thermal domain. It is verified by comparing the available measurements of a quenched skew quadrupole with the simulation results conducted in ANSYS. One has to remember that the skew quadrupole is the only high-order corrector which has quench protection system installed, i.e. it is not self-protected. However, it has been used for illustration purposes because of its available quench measurements when quench heaters were not used. 