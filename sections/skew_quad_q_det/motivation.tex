As it is demonstrated in the previous chapter, the quench velocity-based approach results in a faster solution of a one-dimensional thermal quench propagation with insulation and epoxy resin while introducing an error in nodal temperature and resistive voltage. This chapter aims at validating the quench velocity-based approach by using the real case of a skew quadrupole magnet.
As described in Section~\ref{section: 1d_quench_propagation_geometry}, the skew quadrupole is one of the high-order corrector magnets developed within the scope of the HL-LHC project. This magnet was selected for two reasons: 
\begin{itemize}
    \item The skew quadrupole was tested in LASA laboratories. The resistive voltage and current drop were measured during a provoked quench of the magnet. Therefore, with the quench velocity-based approach one can simulate a real case of a magnet and validate the results against available measurements. 
    \item The geometry of the skew quadrupole is similar to other high-order corrector magnets which are self-protected. One has to remember that the skew quadrupole does have the energy extraction system, i.e. it is not self-protected. However, the analysis based on the skew quadrupole will allow for examining other self-protected corrector magnets in the future. A 3D thermal study is required for a self-protectability analysis. Such studies reduce the safety margin corresponding to the magnet design parameters. 
\end{itemize}
