
As it was proven, the quench velocity-based analysis results in a faster solution of a one-dimensional thermal quench propagation with an external insulation layer. This chapter aims at verifying the quench velocity methodology using the real case of a magnet. In this case, a skew quadrupole is used being one of the high-order corrector magnets designed for the upgrade of High-Luminosity LHC, as described in Section~\ref{subsection: 1d_quench_propagation_geometry}. This magnet was used for two reasons: 
\begin{itemize}
    \item High-order corrector magnets are meant to be self-protected, i.e. their design assumes using no quench protection devices such as quench heaters or CLIQ. When quench occurs, the dissipation of energy stored in a magnet occurs merely by its own rise in resistivity. A 3D thermal study is required for a self-protectability case.
    \item Quench measurements were conducted for this magnet in LASA laboratories. Therefore, the quench velocity methodology can be compared with a real case of a magnet.
\end{itemize}

One has to remember that the skew quadrupole does have the energy extraction system, i.e. it is not entirely self-protected. Nevertheless, the magnet discharge is not dominated by the energy extraction system. The rise of magnet resistance also plays an important role in the drop of current during the discharge.
