As it was demonstrated, the quench velocity-based approach results in a faster solution of a one-dimensional thermal quench propagation with an external insulation layer while introducing an error in the temperature estimation. This chapter aims at verifying the quench velocity method using the real case of a skew quadrupole magnet.
As described in Section~\ref{section: 1d_quench_propagation_geometry}, the skew quadrupole is one of the high-order corrector magnets developed within the scope of HL-LHC project. This magnet was used for two reasons: 
\begin{itemize}
    \item The skew quadrupole was tested in LASA laboratories. The resistive voltage and current drop were measured during a forced discharge of the magnet. Therefore, within the quench velocity-based approach one can simulate a real case of a magnet and be compared with available measurements. 
    \item The geometry of the skew quadrupole is very similar to other high-order corrector magnets which are self-protected, i.e. their design assumes using no quench protection devices such as quench heaters or CLIQ. When quench occurs, the dissipation of energy stored in a magnet occurs merely by its own rise in resistivity. A 3D thermal study is required for a self-protectability case. One has to remember that the skew quadrupole does have the energy extraction system, i.e. it is not self-protected. However, the analysis based on the skew quadrupole will allow for examining other self-protected corrector magnets.
\end{itemize}
