
\subsection{Additional Input Parameters}

The quench is detected 20~ms after the threshold voltage $V_\text{th}$ is reached (see Table~\ref{table:qds_characteristics}). At that moment, ANSYS solves a discharge of an RL-circuit with an initially applied current of $I=86~\text{A}$. The model takes into account the nonlinear inductance of  the magnet presented in Fig.~\ref{fig:differential_inductance}.

\begin{figure}[H]
    \centering
    \begin{tikzpicture}
        \begin{axis}[
          no markers,
          legend style={at={(1,0)},anchor=south east},
          grid=both, 
          grid style={dashed,gray!30},
          width=0.7\linewidth, 
          height = 5cm,
          xlabel={$I,~\text{A}$},
          ylabel={$L_\text{d},~\text{H}$},
          xlabel style={below right},
          ylabel style={above left},
          xmin=0.0,
          xmax=100.0,
          ymin=0.0,
          ymax=6.0
          ]
          \addplot table[x=current,y=inductance,col sep=comma] {sections/skew_quad_q_det/figures/skew_quad_analysis/differential_inductance.csv}; 
        \end{axis}
    \end{tikzpicture}
    \caption{Differential inductance as a function of current for the skew quadrupole~\cite{marco_prioli_mails}.}
    \label{fig:differential_inductance}
\end{figure}

The circuit represented in Fig.~\ref{fig:skew_quad_discharge_electrical_scheme} is solved with additional electrical elements in ANSYS -- CIRCU124. Resistors and inductors modelled by CIRCU124 elements have voltage as a~single degree of freedom~\cite{ansys_element_manual}. The non-linear inductor is modelled by updating its inductance at each communication point between Python and ANSYS. Table~\ref{table: skew_quad_discharge_input_params} depicts additional simulation parameters for the discharge phase of the magnet. 

\begin{table}[H]
    \caption{Input parameters in the analysis of the skew quadrupole discharge.} 
    \vspace{-1.em} 
    \fontsize{10}{10}
    \selectfont 
    \renewcommand{\arraystretch}{1.5}
    \begin{center}
        \begin{tabular}{ ccc }  
        \hline
        parameter & value & unit \\
        \hline
        communication time step before the entire coil quenches & 0.0025 & [s] \\
        communication time step after the entire coil quenches & 0.2 & [s] \\ 
        material properties update criterion & 10 & [A] \\ 
        magnet discharge criterion & 8 & [A] \\
        \hline 
        \end{tabular}
    \end{center}  
     \label{table: skew_quad_discharge_input_params} 
 \end{table}

The quench velocity-based approach is only needed until the entire coil quenches. Therefore, after that moment, the communication time step is increased to $t=0.2~\text{s}$. The remaining communication time steps are required for updating the inductance of a magnet and extracting the value of current. Updating varying material properties, being a function of magnetic field, for 754 windings is a time-consuming process. Thus, it is only conducted if the current in the magnet drops by $I=10~\text{A}$ meaning that the magnetic field also decreases. It is assumed that the magnet is fully discharged if the transport current drops below $I=8~\text{A}$, i.e. if the current decreased by 90\% of its initial value.

\subsection{Results}

Fig.~\ref{fig: magnet_discharge_v_res} presents the evolution of resistive voltage $V_\text{res}$ in time. The higher volume ratio is considered in the model, the lower peak resistive voltage is obtained. Moreover, the peak voltage occurs faster with lower $c_\text{volume}$. In principle, the~peak voltage corresponds to a~moment when both longitudinal and turn-to-turn quench propagation result in a quench of the entire coil. In Fig.~\ref{fig: magnet_discharge_v_res}, one can also notice sharp changes of resistive voltage which occur in gradually increasing time periods. These changes are due to an update of material properties at each $I=10~\text{A}$.

\begin{figure}[H]
    \centering
    \begin{tikzpicture}
        \begin{axis}[
          no markers,
          width=0.8\linewidth, 
          height = 6.5cm,
          xlabel={$\text{Time},~\text{s}$},
          ylabel={$V_\text{res},~\text{V}$},
          xmin=0.0,
          ymin=0.0,
          ymax=30.0,
          legend pos = north east
          ]
          
          \addplot[green] table[x=time_delayed,y=V_res,col sep=comma] {sections/skew_quad_q_det/figures/skew_quad_analysis/results_case1.csv}; 
          
          \addplot[blue] table[x=time_delayed,y=V_res,col sep=comma] {sections/skew_quad_q_det/figures/skew_quad_analysis/results_case2.csv}; 
          
          \addplot[red] table[x=time_delayed,y=V_res,col sep=comma] {sections/skew_quad_q_det/figures/skew_quad_analysis/results_case3.csv}; 
          
          \addplot[black, dashed] table[x=time,y=V_res,col sep=comma] {sections/skew_quad_q_det/figures/skew_quad_analysis/measurements.csv}; 
          
          \legend{
          $c_\text{volume}=0.0$,
          $c_\text{volume}=0.5$,
          $c_\text{volume}=1.0$,
          measurements}
          
        \end{axis}
    \end{tikzpicture}
    \caption{Coil resistive voltage change in time.}
    \label{fig: magnet_discharge_v_res}
\end{figure}

Every simulation results in an overestimation of the peak resistive voltage with respect to the measurements. It is directly connected to the change of coil resistance during the~discharge presented in Fig.~\ref{fig: magnet_discharge_resistance}. The higher resin volume is included in the model, the~lower coil resistance is obtained. It is an expected result because resin is an additional thermal capacitance meaning that the coil remains at a lower temperature. 

\begin{figure}[H]
    \centering
    \begin{tikzpicture}
        \begin{axis}[
          no markers,
          width=0.8\linewidth, 
          height = 6.5cm,
          xlabel={$\text{Time},~\text{s}$},
          ylabel={$R_\text{coil},~\Upomega$},
          xmin=0.0,
          ymin=0.0,
          ymax=1.0,
          legend pos = south east
          ]

          \addplot[green] table[x=time_delayed,y=Resistance,col sep=comma] {sections/skew_quad_q_det/figures/skew_quad_analysis/results_case1.csv}; 
          
          \addplot[blue] table[x=time_delayed,y=Resistance,col sep=comma] {sections/skew_quad_q_det/figures/skew_quad_analysis/results_case2.csv}; 
          
          \addplot[red] table[x=time_delayed,y=Resistance,col sep=comma] {sections/skew_quad_q_det/figures/skew_quad_analysis/results_case3.csv}; 
          
          \addplot[black, dashed] table[x=time,y=Resistance,col sep=comma] {sections/skew_quad_q_det/figures/skew_quad_analysis/measurements.csv}; 
          
          \legend{
          $c_\text{volume}=0.0$,
          $c_\text{volume}=0.5$,
          $c_\text{volume}=1.0$,
          measurements}
          
        \end{axis}
    \end{tikzpicture}
    \caption{Coil resistance evolution in time.}
    \label{fig: magnet_discharge_resistance}
\end{figure}

The final temperature of the coil for every simulated case is shown in Fig.~\ref{fig: magnet_discharge_final_temperature}. The~temperature profile inside of a coil can be compared to a sinusoidal signal with a maximum amplitude of approximately $T=15~\text{K}$. The initially quenched winding is the only place with a distinct temperature difference. The initially quenched zone remained the place in the coil characterised by the highest temperature.

\begin{figure}[H]
    \centering
    \begin{tikzpicture}
        \begin{axis}[
          no markers,
          width=0.8\linewidth, 
          height = 7.0cm,
          xlabel={$L_\text{coil},~\text{m}$},
          ylabel={$T,~\text{K}$},
          xmin=0.0,
          xmax=812.0,
          ymin=0.0,
          ymax=50.0,
          legend pos = south east
          ]

          \addplot[green] table[x=METERS,y=case1,col sep=comma] {sections/skew_quad_q_det/figures/skew_quad_analysis/final_magnet_temperature.csv}; 
          
          \addplot[blue] table[x=METERS,y=case2,col sep=comma] {sections/skew_quad_q_det/figures/skew_quad_analysis/final_magnet_temperature.csv}; 
          
          \addplot[black, dashed] table[x=METERS,y=T_init,col sep=comma] {sections/skew_quad_q_det/figures/skew_quad_analysis/final_magnet_temperature.csv}; 
          
        %   \addplot[red] table[x=time,y=Hot_spot,col sep=comma] {sections/skew_quad_q_det/figures/skew_quad_analysis/results_case3.csv}; 
          
          \legend{
          $c_\text{volume}=0.0$,
          $c_\text{volume}=0.5$,
          $T_\text{init}$
        %   $c_\text{volume}=1.0$
          }
          
        \end{axis}
    \end{tikzpicture}
    \caption{Final temperature of the coil after the discharge.}
    \label{fig: magnet_discharge_final_temperature}
\end{figure}

The evolution of the hot spot temperature temperature in time is presented in Fig.~\ref{fig: magnet_discharge_hot_spot}. It can be concluded that the higher resin volume is taken into consideration, the lower hot spot temperature is obtained in the model. 

\begin{figure}[H]
    \centering
    \begin{tikzpicture}
        \begin{axis}[
          no markers,
          width=0.8\linewidth, 
          height = 5.0cm,
          xlabel={$\text{Time},~\text{s}$},
          ylabel={$T_\text{hot spot},~\text{K}$},
          xmin=0.0,
          ymin=0.0,
          ymax=60.0,
          legend pos = south east
          ]

          \addplot[green] table[x=time,y=Hot_spot,col sep=comma] {sections/skew_quad_q_det/figures/skew_quad_analysis/results_case1.csv}; 
          
          \addplot[blue] table[x=time,y=Hot_spot,col sep=comma] {sections/skew_quad_q_det/figures/skew_quad_analysis/results_case2.csv}; 
          
        %   \addplot[red] table[x=time,y=Hot_spot,col sep=comma] {sections/skew_quad_q_det/figures/skew_quad_analysis/results_case3.csv}; 
          
          \legend{
          $c_\text{volume}=0.0$,
          $c_\text{volume}=0.5$,
        %   $c_\text{volume}=1.0$
          }
          
        \end{axis}
    \end{tikzpicture}
    \caption{Hot spot temperature evolution in time.}
    \label{fig: magnet_discharge_hot_spot}
\end{figure}

The current discharge curves are presented in Fig.~\ref{fig: magnet_discharge_current}. The discharge occurs faster in all simulated cases with respect to the measurements. It is a result of a higher coil resistance compared to the measurements. Thus, the time constant (actually varying) in (\ref{eqn:variable_time_constant}) is higher and the current drop is quicker. 

There is one additional analysis added to the plot in which the coil resistance is omitted. The communication time step, at which the inductance is updated, equals $t=0.0025~\text{s}$. The~ANSYS time step in this case is also lower and equal to $t=0.1~\text{ms}$. The higher precision is due to the fact that such a simulation (with already more accurate input parameters) is much faster with respect to the analysis of an entire coil domain. This analysis with no coil resistance represents a case when the magnet discharge is only dependent on the dump resistor of the energy extraction system.

\begin{figure}[H]
    \centering
    \begin{tikzpicture}
        \begin{axis}[
          no markers,
          width=0.8\linewidth, 
          height = 7.5cm,
          xlabel={$\text{Time},~\text{s}$},
          ylabel={$I,~\text{A}$},
          xmin=0.0,
          ymin=0.0,
          legend pos = north east
          ]

          \addplot[green] table[x=time_delayed,y=Current,col sep=comma] {sections/skew_quad_q_det/figures/skew_quad_analysis/results_case1.csv}; 
          
          \addplot[blue] table[x=time_delayed,y=Current,col sep=comma] {sections/skew_quad_q_det/figures/skew_quad_analysis/results_case2.csv}; 
          
          \addplot[red] table[x=time_delayed,y=Current,col sep=comma] {sections/skew_quad_q_det/figures/skew_quad_analysis/results_case3.csv}; 
          
          \addplot[black] table[x=t_translated,y=I_coil,col sep=comma] {sections/skew_quad_q_det/figures/skew_quad_analysis/results_case0.csv};
                    
          \addplot[black, dashed] table[x=time,y=Current,col sep=comma] {sections/skew_quad_q_det/figures/skew_quad_analysis/measurements.csv}; 

          \legend{
          $c_\text{volume}=0.0$,
          $c_\text{volume}=0.5$,
          $c_\text{volume}=1.0$,
          $R_\text{coil} = 0.0$,
          measurements}
          
        \end{axis}
    \end{tikzpicture}
    \caption{Current discharge curve of the skew quadrupole.}
    \label{fig: magnet_discharge_current}
\end{figure}



