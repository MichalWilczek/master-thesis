
Three simulated cases of the magnet discharge allowed for predicting the final temperature profile of the coil, including the evolution of the hot-spot temperature. In principle, when a turn-to-turn propagation is considered, the temperature rise in the hot-spot is less steep than in case of a 1D longitudinal heat propagation, because a part of the accumulated heat is evacuated to neighbouring windings. Therefore, the multi-strand analysis allows for lowering the safety coefficients with respect to the peak hot-spot temperature in case of a quench. On the other hand, the skew quadrupole model overestimates the temperature distribution in the coil regardless of the analysis case conducted with or without resin. Such a conclusion can be made by observing the curves of coil resistance being higher than in case of the measurements. An overestimation of the temperature in the coil is contradictory to the assumptions related to the quench velocity-based approach in which the temperature is underestimated as outlined in Chapter~\ref{chapter:quench_velocity_benchmarking}. The discrepancy in the results with respect to the measurements can be explained as follows: 

\begin{enumerate}
    \item The mesh size across the insulation is relatively large when the entire skew quadrupole is simulated. As discussed in Section~\ref{section:quench_velocity_benchmarking_with_insulation_heat_balance}, a coarser mesh across the insulation results in an overestimation of the hot-spot temperature (see Fig.~\ref{fig: q_vel_modelling_v_quench_hot_spot_temp_with_insulation}).
    \item No influence of helium is considered. In principle, helium is an important additional heat capacity in the system that should reduce the coil temperature during the discharge.
    \item The insulation and resin are homogenised and modelled with G10. The material properties of G10 are assumed to be similar to S2-glass and CTD-101K epoxy resin. By comparing the simulation results with and without resin, it can be concluded that the heat capacity of the coil elements not belonging to the composite strand have a considerable influence on the temperature evolution of the system during the discharge.
\end{enumerate}

By taking the aforementioned arguments into account, it can be concluded, that the model is validated against available measurements of the magnet discharge while assuming a given accuracy. The overestimation of the quench velocity-based approach for the skew quadrupole remains in the range of 50-120\% with respect to the peak resistive voltage. In fact, an overestimation of the resistive voltage or the hot-spot temperature increases the safety margin from the magnet design standpoint. Therefore, it is a favourable case of the discrepancy between the model and the measurements. 