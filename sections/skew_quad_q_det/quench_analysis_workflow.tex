
As illustrated in Fig.~\ref{fig:block_diagram_skew_quad_analysis_workflow}, there are four steps required to conduct a multi-strand quench analysis of a 3D magnet. The first one aims at creating a 2D magnetic field map as a function of current. As it was presented in Section~\ref{section:multidimensional_mapping_algorithm}, a multi-dimensional mapping algorithm requires a varying magnetic field assigned to each winding of a magnet separately. The average quench velocity also varies with magnetic field and current. Therefore, the quench velocity map is necessary if a magnet discharge is simulated.
The third step of the workflow aims at finding the the minimum length of an insulated strand at critical temperature that leads to the propagation of quench. In literature, this length is referred as a minimum propagating zone (MPZ). When all three steps are completed, one can start simulating a multi-strand 3D geometry of a superconducting magnet.

\begin{figure}[H]
    \centering
    \begin{tikzpicture}[node distance = 1.5cm, auto]
        \renewcommand{\baselinestretch}{0.75} 
        \tikzstyle{decision} = [diamond, draw, fill=blue!20, text width=3cm, text badly centered, node distance=2.5cm, inner sep=0pt, scale=0.8]
        \tikzstyle{block} = [rectangle, draw, fill=blue!20, text width=7.0cm, text centered, rounded corners, minimum height=0.5cm, node distance=1.5cm, scale=0.8]
        \tikzstyle{line} = [draw, -latex']
        \tikzstyle{cloud} = [draw, ellipse,fill=red!20, node distance=7cm, minimum height=2em, scale=0.8]
        
        \node [block] (magnetic_map) {Create a 2D Magnetic Map as a function of current.};
        \node [block, below of=magnetic_map] (v_quench_map) {Create a quench velocity map as a function of current and magnetic field};
        \node [block, below of=v_quench_map] (mpz_analysis) {Find minimum propagating zone for the quench initiation};
        \node [block, below of=mpz_analysis] (skew_quad_analysis) {Conduct a multi-strand analysis of the skew quadrupole};
    
        \path [line] (magnetic_map) -- (v_quench_map);
        \path [line] (v_quench_map) -- (mpz_analysis);
        \path [line] (mpz_analysis) -- (skew_quad_analysis);

    \end{tikzpicture}
    \caption{Analysis workflow for the analysis of a skew quadrupole.}
    \label{fig:block_diagram_skew_quad_analysis_workflow}
\end{figure}