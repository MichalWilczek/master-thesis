
As illustrated in Fig.~\ref{fig:block_diagram_skew_quad_analysis_workflow}, there are four steps required to conduct a multi-strand quench analysis of the skew quadrupole. The first one creates a 2D magnetic field map as a function of current. The assignment of a magnetic map to the windings is presented in Section~\ref{section:multidimensional_mapping_algorithm}. The~average quench velocity also depends on magnetic field and current. The third step of the workflow aims at finding the minimum propagation zone of an insulated strand. It is required because the~spot heater is not included in the simulation. Therefore, one should set an initial temperature profile that leads to a natural quench initiation. When all three steps are completed, one can start simulating the discharge of the skew quadrupole.

\begin{figure}[H]
    \centering
    \begin{tikzpicture}[node distance = 1.5cm, auto]
        \renewcommand{\baselinestretch}{0.75} 
        \tikzstyle{decision} = [diamond, draw, fill=blue!20, text width=3cm, text badly centered, node distance=2.5cm, inner sep=0pt, scale=0.8]
        \tikzstyle{block} = [rectangle, draw, fill=blue!20, text width=7.0cm, text centered, rounded corners, minimum height=0.5cm, node distance=1.25cm, scale=0.8]
        \tikzstyle{line} = [draw, -latex']
        \tikzstyle{cloud} = [draw, ellipse,fill=red!20, node distance=7cm, minimum height=2em, scale=0.8]
        
        \node [block] (magnetic_map) {Create a 2D Magnetic Map as a function of current.};
        \node [block, below of=magnetic_map] (v_quench_map) {Create a quench velocity map as a function of current and magnetic field.};
        \node [block, below of=v_quench_map] (mpz_analysis) {Find minimum propagating zone for the quench initiation.};
        \node [block, below of=mpz_analysis] (skew_quad_analysis) {Conduct a multi-strand analysis of the skew quadrupole.};
    
        \path [line] (magnetic_map) -- (v_quench_map);
        \path [line] (v_quench_map) -- (mpz_analysis);
        \path [line] (mpz_analysis) -- (skew_quad_analysis);

    \end{tikzpicture}
    \caption{Analysis workflow for the study of a skew quadrupole.}
    \label{fig:block_diagram_skew_quad_analysis_workflow}
\end{figure}