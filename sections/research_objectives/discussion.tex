
The implementation of the quench velocity-based approach leads to a considerable decrease of the number of degrees of freedom in a numerical model. The lower number of degrees of freedom allows for simulating a full-scale coil of a magnet. Nevertheless, the simulation remains an approximation in which a certain error should be considered with respect to the standard analysis without the quench velocity-based approach. It is recommended to take the error into account because the key magnet design parameters such as the hot-spot temperature as well as the resistive voltage are underestimated in the quench velocity-based approach with respect to a standard numerical solution. In the quench velocity-based approach, the improvement rate in computation time is the highest when an external insulation layer (optionally including the epoxy resin) with a relatively high number of nodes is simulated. In this case, a high mesh relaxation in the longitudinal direction of the coil allows for a considerable decrease of insulation nodes with respect to the standard numerical analysis. Based on an example of the skew quadrupole, the discharge of the magnet, in which the quench occurs in one coil, can be simulated within five-seven days. Unfortunately, even the quench velocity-based approach did not allow for simulating the skew quadrupole with a refined insulation layer due to an exceeding computation time. The mesh refinement of 20 elements across the insulation would make the model too heavy to compute for ANSYS (in the order of 4 million elements). Therefore, the high increase in computation time is not visible when the quench velocity-based approach is used. In principle, the more accurate discretisation of the insulation layer is applied in the model, the higher gain in the computation time is reached with the quench velocity-based approach. On can conclude that the multi-strand quench simulation of full coils, relying on the quench velocity-based approach, will remain a tool implemented in the final design stage of the quench protection study rather than in its beginning when short iteration loops are required.

The quench velocity-based approach is successfully implemented in ANSYS by using an external routine developed in Python imposing the position of the quenched zone as the simulation proceeds. The ANSYS model allows for studying the current discharge in an RL-circuit with a constant dump resistance and inductance being a function of the transport current. Since the dump resistor is not present in self-protected magnets, is neglected in this case. The temperature evolution inside of the coil relies on the longitudinal and the turn-to-turn quench propagation. Moreover, the material properties are a function of temperature and magnetic field during the quench propagation in a magnet. The~simulation of the full magnet discharge requires using the following ANSYS elements: 

\begin{itemize}
    \item LINK68 used for the composite strand.
    \item LINK33 applied to simulate the insulation layer. 
    \item CIRCU124 required for connecting the coil to an external power source (independent current source), resistor and inductor.
    \item MASS71 optionally used to take resin into consideration when fully impregnated coils are simulated. Moreover, a part of the insulation layer may also be compressed in the given elements.
\end{itemize}
