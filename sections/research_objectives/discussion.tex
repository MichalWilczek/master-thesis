
First of all, the physics implemented in ANSYS models was cross-checked in this thesis with STEAM-BBQ, being a standard tool at TE-MPE-PE Section to simulate a 1D quench propagation. Then, the quench velocity-based approach was introduced, which aims at reducing the number of degrees of freedom in the quench problem allowing for simulating a full-scale coil of a magnet. The ANSYS models and the algorithms governing the quench velocity-based approach were described. The quench velocity-based approach was verified with standard ANSYS simulations, which do not apply the proposed method (instead, they are based on the heat balance equation). The aim of this study was to estimate the error associated with relaxing the mesh density in the quench velocity-based approach. The verification between the standard ANSYS analysis and the quench velocity-based approach was conducted with simplified models only simulating the longitudinal quench propagation. The quench velocity-approach allowed for simulating the skew quadrupole discharge being one of the high-order corrector magnets for the HL-LHC upgrade. One coil of the magnet was simulated in which both the longitudinal and the turn-to-turn propagation were analysed and validated against available measurements.

The implementation of the quench velocity-based approach leads to a considerable decrease in the number of degrees of freedom of the numerical model. The lower number of degrees of freedom allows for simulating a full-scale coil of a magnet. Nevertheless, the simulation remains an approximation, in which a certain error should be considered. It is recommended to take the error into account because the key magnet design parameters, such as the hot-spot temperature as well as the voltage-to-ground, are underestimated in the quench velocity-based approach with respect to a standard numerical solution. In the quench velocity-based approach, the improvement rate in computation time is the highest when an external insulation layer (optionally including the epoxy resin) with a relatively high number of nodes is simulated. In this case, a high mesh relaxation in the longitudinal direction of the coil allows for a considerable decrease in the number of insulation nodes nodes with respect to the standard numerical analysis. Based on the example of the skew quadrupole, the discharge of the magnet, in which the quench occurs in one coil, can be simulated within five-seven days. Unfortunately, even the quench velocity-based approach did not allow for simulating the skew quadrupole with a refined insulation layer due to an exceeding computation time. The mesh refinement of 20 elements across the insulation would make the model too heavy to compute for ANSYS (in the order of 4 million elements). Therefore, the high improvement of the computation time is not visible when the quench velocity-based approach is used. In principle, the more accurate a discretisation of the insulation layer is applied in the model, the higher the gain in the computation time that is reached with the quench velocity-based approach as compared to a standard analysis with the same mesh across the insulation. Overall, the multi-strand quench simulation of full coils, relying on the quench velocity-based approach, will remain a tool implemented in the final design stage of the quench protection study, rather than in its beginning when short iteration loops are required.

The quench velocity-based approach is successfully implemented in ANSYS by using an external routine developed in Python imposing the position of the quenched zone as the simulation proceeds. The ANSYS model allows for studying the current discharge in an RL-circuit with a constant dump resistance and the inductance being a function of the transport current. It is possible to neglect the dump resistance in case of future quench simulations of self-protected magnets. The temperature evolution inside of the coil relies on the longitudinal and the turn-to-turn quench propagation. Moreover, the material properties are a function of temperature and magnetic field strength during the quench propagation in a magnet. The~simulation of the full magnet discharge requires using the following ANSYS elements: 

\begin{enumerate}
    \item LINK68 used for the composite strand.
    \item LINK33 applied to simulate the insulation layer.
    \item CIRCU124 required for connecting the coil to an external power source (independent current source), resistor and inductor.
    \item MASS71 optionally used to take resin into account when fully impregnated coils are simulated. Moreover, a part of the insulation layer may also be compressed in the given elements.
\end{enumerate}
