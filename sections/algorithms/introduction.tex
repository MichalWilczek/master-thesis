
In numerical thermal 3D problem, there are two phenomena to be analysed important for the quench simulations: $(i)$~longitudinal propagation of quench, $(ii)$~transverse propagation of quench across the insulation between separate windings of the coil. Only the \nth{1} part is solved by means of a quench velocity - based thermal model. In order to conduct a 3D thermal analysis using quench velocity method, the simulation tools listed below need to be developed: 
\begin{enumerate}
\item electro-thermal model simulating longitudinal quench propagation
\item thermal model simulating transverse thermal propagation across insulation
\item algorithm mapping multidimensional winding geometry onto 1D coil
\item algorithm calculating quench velocity
\item algorithm assigning node numbers to quench position varying in time
\item algorithm detecting new quenches across multiple windings
\end{enumerate}

ANSYS environment was chosen to perform the electro-thermal and thermal analyses. The external algorithms are prepared in Python. This chapter describes all the aforementioned algorithms as well as their implementation in Python-ANSYS coupling.