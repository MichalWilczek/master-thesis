
There are two important phenomena in thermal quench simulations: $(i)$~longitudinal quench propagation, $(ii)$~transverse heat flow through the insulation layer that causes a~quench in neighbouring windings. The quench velocity-based approach only allows for estimating the longitudinal quench propagation corresponding to the \nth{1} case. The heat flow across the insulation layer should be handled by means of a standard numerical solution. In order to conduct a multi-strand thermo-electric analysis with the insulation layer using a quench velocity-based approach, the following models were developed:

\begin{enumerate}
    \item Electro-thermal model calculating a longitudinal quench propagation with quench onset based on a priori known quench velocity map.
    \item Thermal model allowing for an estimation of a transverse heat propagation across the~insulation.
\end{enumerate}

In addition, the following algorithms were implemented in the developed models in order to meet the requirements of a quench velocity-based approach:

\begin{enumerate}
    \item Quench velocity algorithm. It calculates a time-dependent quench position based on a~quench velocity function.
    \item Multidimensional mapping algorithm. It maps multi-dimensional magnet geometries onto a one-dimensional cable.
    \item Quench detection algorithm. It detects new quenches in a multi-strand domain if a~transverse heat propagation results in a quench of windings directly neighbouring with a quenched zone.
    \item Quenched and non-quenched zone assignment algorithm. It assigns nodes in a~discretised and time-dependent space.
\end{enumerate}

This chapter presents how two aforementioned models are implemented in ANSYS APDL. It also describes all algorithms except for the quench velocity algorithm which was depicted in Section~\ref{section:quench_velocity_cosimulation}.