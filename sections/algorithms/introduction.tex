
There are two important phenomena in thermal quench simulations: $(i)$~longitudinal propagation of quench, $(ii)$~transverse heat flow inside the insulation layer. The quench velocity-based methodology can only be used in the \nth{1} case. The heat flow across the insulation is an important factor if a multi-strand model is analysed. In order to conduct a multi-strand thermal analysis using quench velocity-based approach, the following simulation tools and algorithms ought to be developed.

\begin{enumerate}
\item Electro-thermal model simulating longitudinal quench propagation.
\item Thermal model simulating transverse thermal propagation across insulation.
\item Algorithm mapping multi-strand and multi-dimensional magnet geometry onto 1D coil.
\item Algorithm calculating quench position in time which is based on a quench velocity function.
\item Algorithm which would assign the nodes to the quenched and non-quenched zones in a discretised domain in time.
\item Algorithm detecting new quenches in a multi-strand domain if the quenched winding heats up the neighbouring ones across the insulation.
\end{enumerate}

This chapter describes all the aforementioned algorithms except for the quench velocity algorithm which was described in Section~\ref{section:quench_velocity_cosimulation}.