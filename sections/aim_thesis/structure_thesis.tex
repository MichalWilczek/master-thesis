
The thesis is divided as follows. Chapter~\ref{chapter: introduction} is a general introduction to the quench analysis in superconducting magnets that allows the reader to smoothly acquaint with the problematics of the thesis. Chapter~\ref{chapter: aim_thesis} depicts the main objectives of this work. Since at TE-MPE-PE section\footnote{TE-MPE-PE is an abbreviation for Performance Evaluation Section belonging to Machine Protection and Electrical Integrity Group being part of Technology Department at CERN.}, COMSOL is used as the standard software for solving thermal quench problems, in Chapter~\ref{chapter: 1d_quench_propagation_modelling}, 1D quench propagation is solved in both ANSYS and COMSOL to cross-check the results. The analyses of a 1D strand are conducted with and without an insulation layer and resin. Chapter~\ref{chapter:quench_velocity_modelling} explains in detail the quench velocity-based approach to solve multi-dimensional thermal problems in superconducting magnets.
Chapter~\ref{chapter:algorithms} describes the algorithms created to perform a multi-dimensional thermal analysis. Chapter~\ref{chapter:python_implementation} deals with how the foregoing algorithms are implemented in the simulations through Python scripts and as well as in ANSYS APDL scripts. Chapter~\ref{chapter:quench_velocity_benchmarking} compares the quench velocity-based approach with standard analyses performed in Chapter~\ref{chapter: 1d_quench_propagation_modelling}. In Chapter~\ref{chapter:skew_quadrupole_quench_detection_analysis}, the quench velocity-based approach is applied to solve a multi-dimensional thermal problem of a skew quadrupole, being one of the high-order corrector magnets for the High-Luminosity LHC upgrade. The analysis is performed: $(i)$ at constant current to simulate the quench propagation until the quench is detected, $(ii)$ at varying current during the magnet discharge after the quench detection. The simulation results are compared with available measurements. Chapter~\ref{chapter:research_questions_discussions} summarises the covered topics as well provides an answer to the research questions proposed in this thesis.

