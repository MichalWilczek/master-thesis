
The thesis has been divided as follows. Section \ref{section: 1d_quench_propagation_modelling} describes the basic assumptions related to 1D quench propagation in which numerical examples are presented in two widely available commercial software: ANSYS and COMSOL. The analyses are conducted with and without insulation layers. Section \ref{section:quench_velocity_modelling} explains quench velocity modelling approach and compares this methodology with similar simulations performed in section \ref{section: 1d_quench_propagation_modelling}. Section \ref{section:algorithms} describes the algorithms created to perform a multi-dimensional thermo-electric numerical analysis in general as well as the quench velocity - based one. Section \ref{section:python_implementation} describes how the presented algorithms are implemented in the simulations through Python scripts and then in ANSYS software. In section \ref{section:skew_quadrupole_quench_detection_analysis}, the quench velocity modelling approach is applied to simulate the quench propagation in the skew quadrupole, being one of the high-order corrector magnets for the HL-LHC upgrade.