
The thesis is divided as follows. Since at TE-MPE-PE section\footnote{TE-MPE-PE is an abbreviation for Performance Evaluation Section belonging to Machine Protection and Electrical Integrity Group being part of Technology Department at CERN.}, COMSOL is used as a standard software for solving thermal quench problem, in Chapter \ref{section: 1d_quench_propagation_modelling}, 1D quench propagation is solved in both ANSYS and COMSOL for bench-marking purposes. The analyses of a 1D strand are conducted with and without insulation layer. Chapter \ref{section:quench_velocity_modelling} explains in detail quench velocity modelling approach to solve multi-dimensional thermal problems of superconducting magnets.
Chapter \ref{section:algorithms} describes the algorithms created to perform a multi-dimensional thermal analysis. Chapter \ref{section:python_implementation} describes how the foregoing algorithms are implemented in the simulations through Python scripts and then in ANSYS software. Chapter \ref{section:quench_velocity_benchmarking} compares quench velocity modelling with standard analyses performed in Chapter \ref{section: 1d_quench_propagation_modelling}. In Chapter \ref{section:skew_quadrupole_quench_detection_analysis}, the quench velocity modelling is applied to solve a multi-dimensional thermal problem of a skew quadrupole, being one of the high-order corrector magnets for the High-Luminosity LHC upgrade. The analysis is performed at constant current to simulate the quench propagation until the quench is detected in the magnet. The simulation results are compared with available quench measurements.

