
Solving temperature distribution in a superconducting magnet in 3D where quench is considered is a challenging task. The~reasons standing behind this are twofold: $(i)$~nonlinear material properties at cryogenic temperatures; $(ii)$~high temperature gradients at the quench front. When a numerical solver handles the problem characterised as follows, it decreases its time step in order to reach the convergence. The solver also requires a finer mesh to accurately represent the steep temperature change at the quench front. Both implications make the simulation computationally demanding. A~good example of the level of non-linearities at cryogenic temperatures is the thermal conductivity of copper. Its value changes from 250 to 1700 $\frac{\text{W}}{\text{m K}}$ for $T \in (1.9, 20)~\text{K}$ at $B=3~\text{T}$ according to a fit provided by NIST\footnote{NIST -- National Institute of Standards and Technology} \cite[p.~9-13]{material_properties_roxie}. 

To sum up, a simulation of the 3D quench propagation in a superconducting magnet requires high temporal and spatial resolution, which translate into considerable computation times. Since the 3D thermal models are computationally demanding, they are not suitable for a direct use during a magnet design which is an intrinsically iterative process. In this thesis, there are two scientific questions asked:

\begin{enumerate}
\item Can a multidimensional thermal analysis in superconducting accelerator magnets be more efficient computationally?
\item  Can such an optimised modelling for superconducting accelerator magnets be conducted in ANSYS, whose interface allows for creating multi-dimensional geometries, which is an important asset during a magnet design?
\end{enumerate}

One can ask a question whether there are methods for approximating the quench problem which would allow for obtaining a quick and reliable solution. If the quench position, being a function of operating current and magnetic field, was known a priori in time and estimated beforehand, the numerical solver would solve the temperature distribution faster over the coil domain. In fact, such an approach has been already undertaken by ITER's Magnet Division approach to quench modelling of toroidal superconducting magnets in fusion applications~\cite{iter_presentation_qualified_analysis, iter_fault_case_study}. In this thesis, this method, called quench velocity modelling, is applied to study self-protectability of the superconducting accelerator magnets. 
