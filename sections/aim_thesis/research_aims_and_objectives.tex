
Solving the heat balance equation in 3D in a superconducting magnet, where quench is considered, is a challenging task. There are two primary reasons for that: $(i)$~nonlinear material properties at cryogenic temperatures; $(ii)$~high temperature gradients at the quench front. A numerical solver can handle the problem characterised as follows if the time step and mesh size are relatively small. Both implications serve for attaining the desired convergence in the solution. However, they make the simulation computationally demanding for a full-scale magnet. A~good example of the level of non-linearities at cryogenic temperatures is the thermal conductivity of copper. Its value changes from 250 to 1700 $\frac{\text{W}}{\text{m K}}$ for $T \in (1.9, 20)~\text{K}$ at $B=3~\text{T}$ according to a fit provided by NIST\footnote{NIST -- National Institute of Standards and Technology.} \cite[p.~9-13]{material_properties_roxie}. 

To sum up, a simulation of the 3D quench propagation in a superconducting magnet requires high temporal and spatial resolution, which translates into considerable computation times. Since the 3D thermal models are computationally demanding, they are not suitable for a direct use during a magnet design which is an intrinsically iterative process. In this thesis, there are two scientific questions asked:

\begin{enumerate}
\item Can a multidimensional thermal analysis in superconducting accelerator magnets be more efficient computationally?
\item Can such an optimised modelling for superconducting accelerator magnets be conducted in ANSYS? ANSYS allows for creating multi-dimensional geometries being an important asset in magnet design. Moreover, the majority of mechanical-related studies in superconducting magnets are conducted in this software. ANSYS usage for the quench protection analyses would allow for unifying the tools in multiple design stages.
\end{enumerate}

One can ask a question whether there are methods for approximating the quench problem which would allow for obtaining a quick and reliable solution. If the quench position, being a function of operating current and magnetic field, was known a priori in time and estimated beforehand, the numerical solver would solve the temperature distribution faster over the coil domain. In fact, such an approach has been already undertaken by ITER's Magnet Division approach to quench modelling of toroidal superconducting magnets in fusion applications~\cite{iter_presentation_qualified_analysis, iter_fault_case_study}. In this thesis, the following method called "quench velocity-based approach", is applied to study self-protectability of the superconducting accelerator magnets. 
