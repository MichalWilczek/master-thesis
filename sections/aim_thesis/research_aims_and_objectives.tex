
Solving temperature distribution in a superconducting magnet in 3D where quench is considered is a challenging task. The~reasons standing behind this are twofold: $(i)$~nonlinear material properties at cryogenic temperatures; $(ii)$~high temperature gradients at the quench front. When a numerical solver deals with the problem characterised as follows, it decreases its time step in order to reach convergence. It also requires a finer mesh to represent the steep temperature change at the quench front. Both implications make the simulation computationally demanding. A good example of the level of non-linearities at cryogenic temperatures is thermal conductivity of copper. Its value changes from 250 to 1700 $\frac{\text{W}}{\text{m K}}$ at $T \in (1.9, 20)~\text{K}$ for $B=3~\text{T}$ according to NIST\footnote{NIST -- National Institute of Standards and Technology} \cite[p.~9-13]{material_properties_roxie}.

Since the 3D thermal models are computationally demanding, they are not suitable for use in magnet design which is an intrinsically iterative process. In this thesis, there are two scientific questions asked:

\begin{enumerate}
\item Can a multidimensional thermal analysis in superconducting accelerator magnets be more efficient computationally?
\item  Can such an optimised modelling for superconducting accelerator magnets be conducted in ANSYS which is nowadays one of the most widespread commercial tools for FEM numerical simulations on the market?
\end{enumerate}

One can ask a question whether there exist methods for approximating the quench problem which would allow for obtaining quick and reliable solution. If the quench position in time was known a priori and estimated beforehand, the numerical solver would solve the temperature distribution faster over the coil domain. Such an approach has been inspired by ITER's Magnet Division approach to quench modelling of toroidal superconducting magnets in fusion applications~\cite{iter_presentation_qualified_analysis, iter_fault_case_study}. In this thesis, this method, called quench velocity modelling, is applied for the case of accelerator superconducting magnets. 
