
The European Organisation for Nuclear Research (CERN) is one of the largest organisations performing scientific research in fundamental particle physics. CERN aims at providing particle accelerator facilities for numerous experiments conducted in a strong international collaboration. The Organisation is situated in the French-Swiss border in the Geneva area. Since 2008, CERN has been operating the largest particle accelerator ever built, called Large Hadron Collider (LHC). LHC has become the last component in the CERN accelerator complex. The  accelerator consists of a 27 km-long ring with two beam pipes in which particles travel in opposite directions and accelerate over multiple accelerator turns. The~schematic of the LHC is illustrated in Fig.~\ref{fig:schematic_representation_lhc}.

\begin{figure}[H]
    \centering
    \begin{tikzpicture}
    \node at (0,0) {\includegraphics[width=.45\textwidth]{sections/introduction/figures/LHC_accelerator_view.png}};
    \end{tikzpicture}
    \caption{Schematic representation of the LHC~\cite{schematic_representation_lhc}.}
    \label{fig:schematic_representation_lhc}
\end{figure}

In the last operation of the LHC lasting in the period of 2015-2018, called "Run 2", the LHC was able to accelerate protons to the energy of $E=6.5~\text{TeV}$~\cite{cern_main_webpage} in each of its two beams. The higher energy is attained, the deeper one can study the particle showers obtained during the collisions. They allow for unravelling the fundamentals of particles and interactions between them. There are four collision points in the LHC with different colliders installed that analyse the data from particle showers: ATLAS\footnote{A Toroidal LHC ApparatuS.}, CMS\footnote{CMS -- Compact Muon Solenoid.}, ALICE\footnote{ALICE - A Large Ion Collider Experiment.} and LHCb\footnote{Large Hadron Collider beauty experiment.}. ATLAS and CMS are two large general-purpose detectors whereas ALICE is specialised in heavy-ion physics and LHCb -- in matter-antimatter asymmetry~\cite[p.~3-21]{evans_marvel_of_technology}.

As presented in Fig.~\ref{fig:schematic_representation_lhc}, the LHC machine is divided into eight sectors. Detectors are installed in four of them. The sectors two and eight serve for injecting the preliminary accelerated beam from other CERN accelerators. In sector four, RF cavities accelerate the~beam at every turn until the particles reach the desired energy. The LHC is also equipped with a beam dump system which extracts the beam from the tunnel in case of a machine failure or if the beam quality deteriorates~\cite[p.~1-4]{maciejewski_cosimulation_transient_effects_in_magnets}. 

The LHC is currently being upgraded to the High-Luminosity LHC. The project aims at increasing the frequency of particle collisions, also referred as luminosity, which will improve the machine performance. Moreover, the particles are planned to be accelerated to the energy of $E=7~\text{TeV}$ within the HL-LHC~\cite{cern_main_webpage}.