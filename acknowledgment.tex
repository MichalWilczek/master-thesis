
With the end of this work, I would like to sincerely thank my CERN supervisor, Dr~Michał Maciejewski from the STEAM group. Our multiple discussions over the phenomena governing the simulations, the algorithms related to them, and programming in general, allowed me to develop professionally in the given field. Moreover, I would like to thank him for many comments on writing, based on which I learnt how to ask the right research questions.

I would like to express words of gratitude to my supervisor from Lodz University of Technology, dr hab. inż. Artur Gutkowski. His flexible attitude to organise meetings in a convenient time for me were very helpful, as I was preparing this thesis abroad, far from my home university.

I would also like to mention the former and current STEAM members, the STEAM leader Dr~Arjan Verweij, as well as Lorenzo Bortot, Dr~Matthias Mentink, and Dr~Emanuele Ravaioli who participated in my presentations and gave me many comments on the possibility of improving my work.

I am grateful to the researchers from INFN, Dr~Marco Prioli and Samuele Mariotto, for our collaboration concerning the analysis of the skew quadrupole. Moreover, I would also like to mention Simon McIntosh from ITER Magnet Division for the introduction to the problematics of the quench velocity-based approach. 

Besides, I would like to thank my colleagues Andreas Will and Dr~Thomas Cartier-Michaud for their help in Python scripting as well as the cheerful atmosphere in the office, thanks to which my daily work at CERN was a pure pleasure. Last but not least, I am grateful to my friends, Douglas and Jonathan whose numerous comments on my writing allowed me to correct this thesis to be light and easy to read.
